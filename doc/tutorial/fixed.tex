\section{Fixed size matrices and vectors} \label{fixed-size-sec}
\subsection{Fixed size vectors} \label{fixed-size-vec-sec}
Vectors of sizes two, three and four are represented by specific structures
in Gandalf. The structure and function definitions are nearly identical,
so we shall only describe the workings of 3-vectors. To use 3-vectors
include the header file
\begin{verbatim}
      #include <gandalf/linalg/3vector.h>
\end{verbatim}
for double precision, or
\begin{verbatim}
      #include <gandalf/linalg/3vectorf.h>
\end{verbatim}
for single precision 3-vectors.
A double precision 3-vector is defined as
\begin{verbatim}
      typedef struct Gan_Vector3
      {
         double x, y, z;
      } Gan_Vector3, Gan_Vector3_d
\end{verbatim}
and a single precision 3-vector as
\begin{verbatim}
      typedef struct Gan_Vector3_f
      {
         float x, y, z;
      } Gan_Vector3_f;
\end{verbatim}

Most of the routines below return a pointer to the result vector/matrix.
This may be used as an argument to another routine, although care must
be taken with macros as regards multiple evaluation. The routines are very
safe, because everything using fixed size vectors \& matrices can be written
to involve only automatic variables with no dynamic allocation, and the only
failure modes are arithmetic overflow (Gandalf does not check for this).
The few exceptions are noted in the text.

\subsubsection{Creating and accessing fixed size vectors}
Single fixed size vectors are such simple objects, it makes sense to
normally use declare structure variables directly, rather than use pointers
to structures created by {\tt malloc()}. So to create a double precision
3-vector, use the declaration
\begin{verbatim}
      Gan_Vector3 v3x;
\end{verbatim}
From now on, we shall describe the functions for double precision vectors
only. Single precision functions are very similar and will be explained below.
Setting the coordinates of a 3-vector can be achieved by one of
\begin{enumerate}
  \item Initialising the 3-vector when it is created, as in
\begin{verbatim}
      Gan_Vector3 v3x = {1.0, 2.0, 3.0};
\end{verbatim}

  \item Accessing the structure elements directly:
\begin{verbatim}
      v3x.x = 1.0; v3x.y = 2.0; v3x.z = 3.0;
\end{verbatim}
  \item Using the macro call
\begin{verbatim}
      gan_vec3_fill_q ( &v3x, 1.0, 2.0, 3.0 );
\end{verbatim}
Note that the Gnu C compiler prints a warning when the above call is compiled,
and also for most other similar calls in the linear algebra package.
This warning can be avoided by inserting an initial {\tt (void)} cast:
\begin{verbatim}
      (void)gan_vec3_fill_q ( &v3x, 1.0, 2.0, 3.0 );
\end{verbatim}
We omit this cast in the following to keep the exposition simple.
  \item The equivalent function call
\begin{verbatim}
      v3x = gan_vec3_fill_s ( 1.0, 2.0, 3.0 );
\end{verbatim}
\end{enumerate}
Setting a 3-vector to zero can be accomplished using one of the calls
\begin{verbatim}
      gan_vec3_zero_q ( &v3x ); /* macro, OR */
      v3x = gan_vec3_zero_s(); /* function call */
\end{verbatim}

Copying 3-vectors can be accomplished either by direct assignment
\begin{verbatim}
      Gan_Vector3 v3y;

      v3y = v3x;
\end{verbatim}
or by use of the one of the routines
\begin{verbatim}
      gan_vec3_copy_q ( &v3x, &v3y ); /* macro, OR */
      v3y = gan_vec3_copy_s ( &v3x ); /* function call */
\end{verbatim}

\subsubsection{Fixed size vector addition}
To add two 3-vectors use either the macro
\begin{verbatim}
      Gan_Vector3 v3z;

      gan_vec3_add_q ( &v3x, &v3y, &v3z ); /* macro */
\end{verbatim}
or the function
\begin{verbatim}
      v3z = gan_vec3_add_s ( &v3x, &v3y ); /* function call */
\end{verbatim}
See the discussion of ``quick'' and ``slow'' versions of the same operation,
identified by the {\tt ...\_q} and {\tt ...\_s} suffices,
in Section~\ref{conventions-sec}. In this case, the ``slow'' version
{\tt gan\_vec3\_add\_s()} has the overhead of a function call relative
to the ``quick'' version {\tt gan\_vec3\_add\_q()}, so the latter should
be used unless the input vectors are not simple variables (i.e. they might
be elements of arrays), in which case the repeated evaluation required
by the macro version might be slower.

There are also in-place versions of the add operation, which overwrite one
of the input vectors with the result. The macro operations
\begin{verbatim}
      gan_vec3_add_i1 ( &v3x, &v3y ); /* result in-place in v3x */
\end{verbatim}
and
\begin{verbatim}
      gan_vec3_add_i2 ( &v3x, &v3y ); /* result in-place in v3y */
\end{verbatim}
produce the same result but overwrite respectively the first {\tt v3x}
and the second {\tt v3y} argument with the result. There is also a
more explicit macro
\begin{verbatim}
      gan_vec3_increment ( &v3x, &v3y ); /* result in-place in v3x */
\end{verbatim}
which increments {\tt v3x} by {\tt v3y}, i.e. identical to
{\tt gan\_vec3\_add\_i1()}. Note that if one of the input arguments is
a non-trivial expression, and the result is being overwritten on the other,
use the function {\tt gan\_vec3\_add\_s()}, as in
\begin{verbatim}
      Gan_Vector3 av3x[100];

      /* ... fill av3x array ... */
      v3x = gan_vec3_add_s ( &v3x, &av3x[33] );
\end{verbatim}

\subsubsection{Fixed size vector subtraction}
To subtract 3-vectors use the equivalent macros and functions
\begin{verbatim}
      gan_vec3_sub_q ( &v3x, &v3y, &v3z ); /* macro */
      v3z = gan_vec3_sub_s ( &v3x, &v3y ); /* function call */
      gan_vec3_sub_i1 ( &v3x, &v3y ); /* result in-place in v3x */
      gan_vec3_sub_i2 ( &v3x, &v3y ); /* result in-place in v3y */
      gan_vec3_decrement ( &v3x, &v3y ); /* result in-place in v3x */
\end{verbatim}

\subsubsection{Rescaling a fixed size vector}
There are similar functions to multiply a 3-vector by a scalar
\begin{verbatim}
      double ds;

      gan_vec3_scale_q ( &v3x, ds, &v3z ); /* macro */
      v3z = gan_vec3_scale_s ( &v3x, ds ); /* function call */
      gan_vec3_scale_i ( &v3x, ds ); /* macro, result in-place in v3x */
\end{verbatim}
to divide a 3-vector by a (non-zero) scalar
\begin{verbatim}
      gan_vec3_divide_q ( &v3x, ds, &v3z ); /* macro */
      v3z = gan_vec3_divide_s ( &v3x, ds ); /* function call */
      gan_vec3_divide_i ( &v3x, ds ); /* macro, result in-place in v3x */
\end{verbatim}
to negate a 3-vector
\begin{verbatim}
      gan_vec3_negate_q ( &v3x, &v3z ); /* macro */
      v3z = gan_vec3_negate_s ( &v3x ); /* function call */
      gan_vec3_negate_i ( &v3x ); /* macro, result in-place in v3x */
\end{verbatim}
and to scale a 3-vector to unit length (2-norm)
\begin{verbatim}
      gan_vec3_unit_q ( &v3x, &v3z ); /* macro */
      v3z = gan_vec3_unit_s ( &v3x ); /* function call */
      gan_vec3_unit_i ( &v3x ); /* macro, result in-place in v3x */
\end{verbatim}
The last guarantees than the total squared element of the vector
returned by {\tt gan\_vec3\_sqrlen()} will be one; see below.

{\tt Error detection:} If zero is passed as the scalar value to the
{\tt ...\_divide\_[qi]()} routines, {\tt NULL} will be returned, while the
{\tt ...\_divide\_s()} routines will abort the program. You should add tests
for division by zero {\em before} calling any of the {\tt ...\_divide\_[qsi]()}
routines. Similarly, the {\tt ...\_unit\_[qsi]()} routines will fail if
the input vector contains only zeros. If this is a possibility then the
program should check beforehand.

\subsubsection{Fixed size vector products}
Vector dot product (scalar product) is compute using the alternatives
\begin{verbatim}
      ds = gan_vec3_dot_q ( &v3x, &v3y ); /* macro, or */
      ds = gan_vec3_dot_s ( &v3x, &v3y ); /* function call */
\end{verbatim}
Similarly, to compute the squared length of a 3-vector, use
\begin{verbatim}
      ds = gan_vec3_sqrlen_q ( &v3x ); /* macro, or */
      ds = gan_vec3_sqrlen_s ( &v3x ); /* function call */
\end{verbatim}

For 3-vectors only we also have the cross product (vector product)
\begin{verbatim}
      gan_vec3_cross_q ( &v3x, &v3y, &v3z ); /* macro */
      v3z = gan_vec3_cross_s ( &v3x, &v3y ); /* function call */
\end{verbatim}

There are also outer products formed by two vectors, producing a matrix.
These functions are described in Section~\ref{fixed-vector-outer-sec}.

\subsubsection{Fixed size vector file I/O}
Gandalf supports both ASCII and binary format file I/O of vectors and
matrices. Both formats use standard {\tt FILE~*} file streams.
ASCII format is obviously more convenient to use, while binary format is
more compact and guarantees no loss of precision when the data is read.
To print a 3-vector in ASCII format, use
\begin{verbatim}
      Gan_Bool gan_vec3_fprint ( FILE *fp, Gan_Vector3 *p,
                                 const char *prefix, int indent, const char *fmt );
\end{verbatim}
{\tt prefix} is a prefix string to print before the vector itself, {\tt indent}
is the number of spaces to indent the vector by, and {\tt fmt} is a format
string to use when printing the vector, e.g. {\tt "\%f"}. So for example
\begin{verbatim}
      FILE *pfFile;

      pfFile = fopen ( "/tmp/vectors", "w" );
      gan_vec3_fill_q ( &v3x, 1.0, 2.0, 3.0 );
      gan_vec3_fprint ( pfFile, &v3x, "Example vector", 3, "%f" );
\end{verbatim}
will print the output
\begin{verbatim}
         Example vector: 1.000000 2.000000 3.000000
\end{verbatim}
to the file {\tt "/tmp/vectors"}.
There is also a version {\tt gan\_vec3\_print()} for printing to
standard output:
\begin{verbatim}
      Gan_Bool gan_vec3_print ( Gan_Vector3 *p,
                                const char *prefix, int indent, const char *fmt );
\end{verbatim}
The corresponding input function is
\begin{verbatim}
      Gan_Bool gan_vec3_fscanf ( FILE *fp, Gan_Vector3 *p,
                                 char *prefix, int prefix_len )
\end{verbatim}
which reads the vector from the file stream {\tt fp} into the 3-vector
pointer {\tt p}. It also reads the prefix string (up to the specified maximum
length {\tt prefix\_len}), which can be compared with the expected prefix
string to check for consistency.

Binary file I/O is handled by the functions {\tt gan\_vec3\_fwrite()}
and {\tt gan\_vec3\_fread()}. To write a 3-vector in binary format use
\begin{verbatim}
      Gan_Bool gan_vec3_fwrite ( FILE *fp, Gan_Vector3 *p, gan_ui32 magic_number );
\end{verbatim}
The {\tt magic\_number} takes the same role as the {\tt prefix} string
in {\tt gan\_vec3\_fprint()}, and is written into the file so that it can be
used later to identify the vector when it is read back using
\begin{verbatim}
      Gan_Bool gan_vec3_fread ( FILE *fp, Gan_Vector3 *p, gan_ui32 *magic_number );
\end{verbatim}

{\tt Error detection:} The I/O routines return a boolean value, returning
{\tt GAN\_TRUE} on success, {\tt GAN\_FALSE} on failure, invoking the Gandalf
error handle in the latter case.

\subsubsection{Conversion from general to fixed size vector}
Functions are provided to convert a general vector {\tt Gan\_Vector}
{\tt Gan\_Vector3}, provided that the general vector has actually been
created with size three. So for instance
\begin{verbatim}
      Gan_Vector *pvx;

      pvx = gan_vec_alloc(3);
      gan_vec_fill_va ( pvx, 3, 2.0, 3.0, 4.0 );
      gan_vec3_from_vec_q ( pvx, &v3x );
\end{verbatim}
or
\begin{verbatim}
      v3x = gan_vec3_from_vec_s ( pvx );
\end{verbatim}
fill {\tt v3x} with the same values as the general vector.
Calling these functions with a general vector {\tt pvx} not having
the same size as the fixed size vector is an error.

{\tt Error detection:} The conversion routines return the pointer to the
filled fixed size vector, or {\tt NULL} on failure, invoking the Gandalf
error handle in the latter case.

\subsubsection{Single precision fixed size vectors}
\begin{verbatim}
      #include <gandalf/linalg/3vectorf.h>
\end{verbatim}
There is an identical set of functions for handling single precision
floating point vectors, the names of which are obtained by replacing
{\tt "gan\_vec3\_..."} in the above functions with {\tt "gan\_vec3f\_..."}.
For example, to add two single precision 3-vectors use
\begin{verbatim}
      Gan_Vector3_f v3x, v3y, v3z;

      /* ... fill v3x and v3y ... */
      gan_vec3f_add_q ( &v3x, &v3y, &v3z );
\end{verbatim}

\subsubsection{Other types of fixed size vector}
As well as floating point coordinates, integer vectors are often useful.
Although Gandalf does not currently provide sets of functions handling
vectors with integer coordinates, structures are defined as follows
\begin{verbatim}
      /* structure definition for unsigned character 3-vector */
      typedef struct Gan_Vector3_uc
      {
         unsigned char x, y, z;
      } Gan_Vector3_uc;

      /* structure definition for short integer 3-vector */
      typedef struct Gan_Vector3_s
      {
         short x, y, z;
      } Gan_Vector3_s;

      /* structure definition for unsigned short integer 3-vector */
      typedef struct Gan_Vector3_us
      {
         unsigned short x, y, z;
      } Gan_Vector3_us;

      /* structure definition for integer 3-vector */
      typedef struct Gan_Vector3_i
      {
         int x, y, z;
      } Gan_Vector3_i;

      /* structure definition for unsigned integer 3-vector */
      typedef struct Gan_Vector3_ui
      {
         unsigned int x, y, z;
      } Gan_Vector3_ui;
\end{verbatim}

\subsubsection{Other sizes of fixed size vector}
Gandalf supports fixed size vectors with sizes two, three and four.
The functions described above for size three vectors are repeated
for sizes two and four, both single and double precision, in the header
files
\begin{verbatim}
      #include <gandalf/linalg/2vector.h>   /* double precision */
      #include <gandalf/linalg/2vectorf.h>  /* single precision */
      #include <gandalf/linalg/4vector.h>   /* double precision */
      #include <gandalf/linalg/4vectorf.h>  /* single precision */
\end{verbatim}

\subsubsection{Setting/extracting parts of fixed size vectors}
Apart from the cross product routines {\tt gan\_vec3\_cross\_[qs]()}
defined only for 3-vectors, there are a few other miscellaneous routines which
apply to a subset of the fixed size vectors. These routines enable setting
or extracting parts of a fixed size vector using another fixed size with
a different size. The most comprehensive set of such routines is for
vectors of size four. So for instance to extract the first three elements
of a 4-vector and write them into a 3-vector, use
\begin{verbatim}
      Gan_Vector3 v3x;
      Gan_Vector4 v4x;

      gan_vec4_fill_q ( &v4x, 1.0, 2.0, 3.0, 4.0 );
      gan_vec4_get_v3t_q ( &v4x, &v3x ); /* macro */
\end{verbatim}
or alternatively
\begin{verbatim}
      v3x = gan_vec4_get_v3t_s ( &v4x ); /* function */
\end{verbatim}
both of which set {\tt v3x} to {\tt \{1.0, 2.0, 3.0\}}.
To build a 4-vector from a 3-vector and a scalar use
\begin{verbatim}
      gan_vec3_fill_q ( &v3x, 1.0, 2.0, 3.0 );
      gan_vec4_set_parts_q ( &v4x, &v3x, 4.0 ); /* macro */
\end{verbatim}
or alternatively
\begin{verbatim}
      v4x = gan_vec4_set_parts_s ( &v3x, 4.0 ); /* function */
\end{verbatim}
both of which set {\tt v4x} to {\tt \{1.0, 2.0, 3.0, 4.0\}}.
To build a 4-vector from two 2-vectors use
\begin{verbatim}
      Gan_Vector3 v2xt, v2xb;
      Gan_Vector4 v4x;

      gan_vec2_fill_q ( &v2xt,  1.0, 2.0 );
      gan_vec2_fill_q ( &v2xb, 3.0, 4.0 );
      gan_vec4_set_blocks_q ( &v4x, &v2xt, &v2xb ); /* macro */
\end{verbatim}
(note that the ``t'' and ``b'' in {\tt v2xt} and {\tt v2xb} stand for the
``top'' and ``bottom'' parts of vector $\xvec$), or alternatively
\begin{verbatim}
      v4x = gan_vec4_set_blocks_s ( &v2xt, &v2xb ); /* function */
\end{verbatim}
both of which again set {\tt v4x} to {\tt \{1.0, 2.0, 3.0, 4.0\}}.

For 3-vectors the equivalent set of functions involves splitting the
3-vector into the {\tt x,y} coordinates as a 2-vector and {\tt z} as the
scalar. Then we have
\begin{verbatim}
      Gan_Vector2 v2xt;
      Gan_Vector3 v3x;

      gan_vec3_fill_q ( &v3x, 1.0, 2.0, 3.0 );
      gan_vec3_get_v2t_q ( &v3x, &v2xt ); /* macro, or */
      v2xt = gan_vec3_get_v2t_s ( &v3x ); /* function */
\end{verbatim}
the last two lines of which both set {\tt v2xt} to {\tt \{1.0, 2.0\}}.
To build a 3-vector from a 2-vector and a scalar use
\begin{verbatim}
      gan_vec2_fill_q ( &v2xt, 1.0, 2.0 );
      gan_vec3_set_parts_q ( &v3x, &v2xt, 3.0 ); /* macro, or */
      v3x = gan_vec3_set_parts_s ( &v2xt, 3.0 ); /* function */
\end{verbatim}
both of which set {\tt v3x} to {\tt \{1.0, 2.0, 3.0\}}.

\subsection{Fixed size matrices} \label{fixed-size-mat-sec}
Matrices of sizes $2\times 2$, $2\times 3$, $2\times 4$, $3\times 3$,
$3\times 4$ and $4\times 4$ are represented by specific structures
in Gandalf. There is a large set of functions which is repeated across
every size of matrix. There is also a set of functions specific to square
matrices. In both cases we will choose a single size of matrix and describe
the functions available for that size. The sizes we will use are $3\times 4$
matrices for the functions available to every size of matrix, and $3\times 3$
matrices for the functions specific to square matrices.

\subsubsection{Definitions of fixed size matrice}\label{fixed34-sec}
To use $3\times 4$ matrices or $3\times 3$ matrices
include the header files
\begin{verbatim}
      #include <gandalf/linalg/3x4matrix.h> /* OR */
      #include <gandalf/linalg/3x3matrix.h>
\end{verbatim}
respectively. This is for double precision matrix elements.
The files to include for single precision elements are
\begin{verbatim}
      #include <gandalf/linalg/3x4matrixf.h> /* OR */
      #include <gandalf/linalg/3x3matrixf.h>
\end{verbatim}
A double precision $3\times 4$ matrix is defined as
\begin{verbatim}
      typedef struct Gan_Matrix34
      {
         double xx, xy, xz, xw,
                yx, yy, yz, yw,
                zx, zy, zz, zw;
      } Gan_Matrix34;
\end{verbatim}
A $3\times 3$ matrix is similarly defined as
\begin{verbatim}
      typedef struct Gan_Matrix33
      {
         double xx, xy, xz,
                yx, yy, yz,
                zx, zy, zz;
      } Gan_Matrix33;
\end{verbatim}
For square matrices there is also a specific structure to handle symmetric
and triangular matrix structures, as follows:
\begin{verbatim}
      #ifndef NDEBUG
      /* square matrix type, for setting and checking in debug mode */
      typedef enum { GAN_SYMMETRIC_MATRIX33, GAN_LOWER_TRI_MATRIX33 }
       Gan_SquMatrix33Type;
      #endif /* #ifndef NDEBUG */

      /* structure definition for square 3x3 matrix */
      typedef struct Gan_SquMatrix33
      {
      #ifndef NDEBUG
         /* square matrix type, for setting and checking in debug mode */
         Gan_SquMatrix33Type type;
      #endif /* #ifndef NDEBUG */

         /* matrix data */
         double xx,
                yx, yy,
                zx, zy, zz;
      } Gan_SquMatrix33;
\end{verbatim}
Note that the matrix type field {\tt Gan\_SquMatrix33} is only used in
debug compilation mode ({\tt NDEBUG} not defined). The {\tt type} field is
actually invisible to the programmers' interface to the Gandalf functions,
and is used merely for internal checking. Note also that Gandalf does not
provide explicit support for {\em upper} triangular fixed size matrices
(although it does for general size matrices;
see Section~\ref{general-size-sec}). Any operations involving upper triangular
matrices can be implemented using implicit transpose of a lower triangular
matrix.

Single precision structures are defined similarly, with the names changed
to {\tt Gan\_Matrix34\_f}, {\tt Gan\_SquMatrix33\_f} etc.

\subsubsection{Creating and accessing fixed size matrices}
Single fixed size matrices are such simple objects, it makes sense to
normally use declare structure variables directly, rather than use pointers
to structures created by {\tt malloc()}. So to create a double precision
$3\times 4$ matrix, use the declaration
\begin{verbatim}
      Gan_Matrix34 m34A;
\end{verbatim}
From now on, we shall describe the routines for double precision matrices
only. Single precision functions are very similar and will be explained below.
Setting the coordinates of a $3\times 4$ matrix can be achieved by one of
\begin{enumerate}
  \item Initialising the $3\times 4$ matrix when it is created, as in
\begin{verbatim}
      Gan_Matrix34 m34A = {1.0,  2.0,  3.0,  4.0,
                           5.0,  6.0,  7.0,  8.0,
                           9.0, 10.0, 11.0, 12.0};
\end{verbatim}

  \item Accessing the structure elements directly:
\begin{verbatim}
      m34A.xx = 1.0; m34A.xy = 2.0; m34A.xz = 3.0; /* etc. */
\end{verbatim}
  \item Using the macro call
\begin{verbatim}
      gan_mat34_fill_q ( &m34A, 1.0,  2.0,  3.0,  4.0,
                                5.0,  6.0,  7.0,  8.0,
                                9.0, 10.0, 11.0, 12.0 );
\end{verbatim}
Note that the Gnu C compiler prints a warning when the above call is compiled,
and also for most other similar calls in the linear algebra package.
This warning can be avoided by inserting an initial {\tt (void)} cast:
\begin{verbatim}
      (void)gan_mat34_fill_q ( &m34A, 1.0,  2.0,  3.0,  4.0,
                                      5.0,  6.0,  7.0,  8.0,
                                      9.0, 10.0, 11.0, 12.0 );
\end{verbatim}
We omit this cast in the following to keep the exposition simple.
  \item The equivalent function call
\begin{verbatim}
      m34A = gan_mat34_fill_s ( 1.0,  2.0,  3.0,  4.0,
                                5.0,  6.0,  7.0,  8.0,
                                9.0, 10.0, 11.0, 12.0 );
\end{verbatim}
\end{enumerate}

The methods of initialising a $3\times 3$ matrix follow those listed above
for $3\times 4$ matrices, for instance
\begin{verbatim}
      Gan_Matrix33 m33A;

      gan_mat33_fill_q ( &m33A, 1.0, 2.0, 3.0,
                                4.0, 5.0, 6.0,
                                7.0, 8.0, 9.0 ); /* OR */
      m33A = gan_mat33_fill_s ( 1.0, 2.0, 3.0,
                                4.0, 5.0, 6.0,
                                7.0, 8.0, 9.0 );
\end{verbatim}
For a symmetric or lower triangular
{\tt Gan\_SquMatrix33} matrix, direct initialisation (options 1 and 2 above) is
not advisable, because of the {\tt type} field of the structure whose
presence depends on {\tt NDEBUG}, Instead use the macro calls
\begin{verbatim}
      Gan_SquMatrix33 sm33S, sm33L;

      /* symmetric matrix */
      gan_symmat33_fill_q ( &sm33S, 1.0,
                                    2.0, 3.0,
                                    4.0, 5.0, 6.0 );

      /* lower triangular matrix */
      gan_ltmat33_fill_q ( &sm33L, 1.0,
                                   2.0, 3.0,
                                   4.0, 5.0, 6.0 );
\end{verbatim}
The first of these fills the matrix without specifying the values above the
diagonal, and actually builds the matrix
\[ S = \beginm{ccc} 1 & 2 & 4 \\ 2 & 3 & 5 \\ 4 & 5 & 6\endm
\]
The second builds the lower triangular matrix
\[ L = \beginm{ccc} 1 & 0 & 0 \\ 2 & 3 & 0 \\ 4 & 5 & 6\endm
\]
Setting a fixed-size matrix to zero can be accomplished using one of the calls
\begin{verbatim}
      gan_mat34_zero_q ( &m34A ); /* OR */ m34A = gan_mat34_zero_s();
      gan_mat33_zero_q ( &m33A ); /* OR */ m33A = gan_mat33_zero_s();
      gan_symmat33_zero_q ( &sm33S ); /* OR */ sm33S = gan_symmat33_zero_s();
      gan_ltmat33_zero_q ( &sm33L ); /* OR */ sm33L = gan_ltmat33_zero_s();
\end{verbatim}

Setting a square matrix to identity is achieved using
\begin{verbatim}
      gan_mat33_ident_q ( &m33A ); /* OR */ m33A = gan_mat33_ident_s();
      gan_symmat33_ident_q ( &sm33S ); /* OR */ sm33S = gan_symmat33_ident_s();
      gan_ltmat33_ident_q ( &sm33L ); /* OR */ sm33L = gan_ltmat33_ident_s();
\end{verbatim}

Copying $3\times 4$ matrices can be accomplished either by direct assignment
\begin{verbatim}
      Gan_Matrix34 m34B;

      m34B = m34A;
\end{verbatim}
or by use of one of the routines
\begin{verbatim}
      gan_mat34_copy_q ( &m34A, &m34B ); /* macro, OR */
      m34B = gan_mat34_copy_s ( &m34A ); /* function call */
\end{verbatim}
The methods of copying general, symmetric and lower triangular matrices follow
that of $3\times 4$ matrices.

\subsubsection{Fixed size matrix addition}
To add two $3\times4$ matrices use either the macro
\begin{verbatim}
      Gan_Matrix34 m34C;

      /* ... set up m34A, m34B using e.g. gan_mat34_fill_q() ... */
      gan_mat34_add_q ( &m34A, &m34B, &m34C ); /* macro */
\end{verbatim}
or the function
\begin{verbatim}
      m34C = gan_mat34_add_s ( &m34A, &m34B ); /* function call */
\end{verbatim}
See the discussion of ``quick'' and ``slow'' versions of the same operation,
identified by the {\tt ...\_q} and {\tt ...\_s} suffices,
in Section~\ref{conventions-sec}. In this case, the ``slow'' version
{\tt gan\_mat34\_add\_s()} has the overhead of a function call relative
to the ``quick'' version {\tt gan\_mat34\_add\_q()}, so the latter should
be used unless the input matrices are not simple variables (i.e. they might
be elements of arrays), in which case the repeated evaluation required
by the macro version might be slower.

There are also in-place versions of the add operation, which overwrite one
of the input matrices with the result. The macro operations
\begin{verbatim}
      gan_mat34_add_i1 ( &m34A, &m34B ); /* result in-place in m34A */
\end{verbatim}
and
\begin{verbatim}
      gan_mat34_add_i2 ( &m34A, &m34B ); /* result in-place in m34B */
\end{verbatim}
produce the same result but overwrite respectively the first {\tt m34A}
and the second {\tt m34B} argument with the result. There is also a
more explicit macro
\begin{verbatim}
      gan_mat34_increment ( &m34A, &m34B ); /* result in-place in m34A */
\end{verbatim}
which increments {\tt m34A} by {\tt m34B}, i.e. identical to
{\tt gan\_mat34\_add\_i1()}. Note that if one of the input arguments is
a non-trivial expression, and the result is being overwritten on the other,
use the function {\tt gan\_mat34\_add\_s()}, as in
\begin{verbatim}
      Gan_Matrix34 am34A[100];

      /* ... fill am34A array ... */
      m34A = gan_mat34_add_s ( &m34A, &am34A[33] );
\end{verbatim}

For general, symmetric and lower triangular $3\times 3$ matrices the
addition routines follow those for $3\times 4$ matrices. So for general
$3\times 3$ matrices we have the options
\begin{verbatim}
      Gan_Matrix33 m33A, m33B, m33C;

      /* ... set up m33A, m33B using e.g. gan_mat33_fill_q() ... */
      gan_mat33_add_q ( &m33A, &m33B, &m33C ); /* macro, OR */
      m33C = gan_mat33_add_s ( &m33A, &m33B ); /* function call */
      gan_mat33_add_i1 ( &m33A, &m33B ); /* macro, result in-place in m33A */
      gan_mat33_add_i2 ( &m33A, &m33B ); /* macro, result in-place in m33B */
      gan_mat33_increment ( &m33A, &m33B ); /* equivalent to gan_mat33_add_i1() */
\end{verbatim}
For symmetric $3\times 3$ matrices we have
\begin{verbatim}
      Gan_SquMatrix33 sm33Sa, sm33Sb, sm33Sc;

      /* ... set up sm33Sa, sm33Sb using e.g. gan_symmat33_fill_q() ... */
      gan_symmat33_add_q ( &sm33Sa, &sm33Sb, &sm33Sc ); /* macro, OR */
      sm33Sc = gan_symmat33_add_s ( &sm33Sa, &sm33Sb ); /* function call */
      gan_symmat33_add_i1 ( &sm33Sa, &sm33Sb ); /* macro, result in-place in sm33Sa */
      gan_symmat33_add_i2 ( &sm33Sa, &sm33Sb ); /* macro, result in-place in sm33Sb */
      gan_symmat33_increment ( &sm33Sa, &sm33Sb ); /* equivalent to gan_symmat33_add_i1() */
\end{verbatim}
Finally for lower triangular $3\times 3$ matrices we have
\begin{verbatim}
      Gan_SquMatrix33 sm33La, sm33Lb, sm33Lc;

      /* ... set up sm33La, sm33Lb using e.g. gan_ltmat33_fill_q() ... */
      gan_ltmat33_add_q ( &sm33La, &sm33Lb, &sm33Lc ); /* macro, OR */
      sm33Lc = gan_ltmat33_add_s ( &sm33La, &sm33Lb ); /* function call */
      gan_ltmat33_add_i1 ( &sm33La, &sm33Lb ); /* macro, result in-place in sm33La */
      gan_ltmat33_add_i2 ( &sm33La, &sm33Lb ); /* macro, result in-place in sm33Lb */
      gan_ltmat33_increment ( &sm33La, &sm33Lb ); /* equivalent to gan_ltmat33_add_i1() */
\end{verbatim}

For general square matrices there are routines to implement the operation
\[ S = A+A\tr,
\]
obtaining a symmetric matrix $S$ by adding together a general square matrix $A$
and its transpose. The routines for $3\times 3$ matrices are
\begin{verbatim}
      Gan_Matrix33 m33A;
      Gan_SquMatrix33 sm33S;

      /* set up m33A using e.g. gan_mat33_fill_q() */
      gan_mat33_saddT_q ( &m33A, &sm33S ); /* S = A+A^T, macro */
      sm33S = gan_mat33_saddT_s ( &m33A ); /* S = A+A^T, function call */
\end{verbatim}

\subsubsection{Fixed size matrix subtraction}
To subtract $3\times4$ matrices use the equivalent macros and functions
\begin{verbatim}
      gan_mat34_sub_q ( &m34A, &m34B, &m34C ); /* macro */
      m34C = gan_mat34_sub_s ( &m34A, &m34B ); /* function call */
      gan_mat34_sub_i1 ( &m34A, &m34B ); /* result in-place in m34A */
      gan_mat34_sub_i2 ( &m34A, &m34B ); /* result in-place in m34B */
      gan_mat34_decrement ( &m34A, &m34B ); /* result in-place in m34A */
\end{verbatim}
For general $3\times 3$ matrices we have the options
\begin{verbatim}
      Gan_Matrix33 m33A, m33B, m33C;

      /* ... set up m33A, m33B using e.g. gan_mat33_fill_q() ... */
      gan_mat33_sub_q ( &m33A, &m33B, &m33C ); /* macro, OR */
      m33C = gan_mat33_sub_s ( &m33A, &m33B ); /* function call */
      gan_mat33_sub_i1 ( &m33A, &m33B ); /* macro, result in-place in m33A */
      gan_mat33_sub_i2 ( &m33A, &m33B ); /* macro, result in-place in m33B */
      gan_mat33_decrement ( &m33A, &m33B ); /* equivalent to gan_mat33_sub_i1() */
\end{verbatim}
For symmetric $3\times 3$ matrices we have
\begin{verbatim}
      Gan_SquMatrix33 sm33Sa, sm33Sb, sm33Sc;

      /* ... set up sm33Sa, sm33Sb using e.g. gan_symmat33_fill_q() ... */
      gan_symmat33_sub_q ( &sm33Sa, &sm33Sb, &sm33Sc ); /* macro, OR */
      sm33Sc = gan_symmat33_sub_s ( &sm33Sa, &sm33Sb ); /* function call */
      gan_symmat33_sub_i1 ( &sm33Sa, &sm33Sb ); /* macro, result in-place in sm33Sa */
      gan_symmat33_sub_i2 ( &sm33Sa, &sm33Sb ); /* macro, result in-place in sm33Sb */
      gan_symmat33_decrement ( &sm33Sa, &sm33Sb ); /* equivalent to gan_symmat33_sub_i1() */
\end{verbatim}
Finally for lower triangular $3\times 3$ matrices we have
\begin{verbatim}
      Gan_SquMatrix33 sm33La, sm33Lb, sm33Lc;

      /* ... set up sm33La, sm33Lb using e.g. gan_ltmat33_fill_q() ... */
      gan_ltmat33_sub_q ( &sm33La, &sm33Lb, &sm33Lc ); /* macro, OR */
      sm33Lc = gan_ltmat33_sub_s ( &sm33La, &sm33Lb ); /* function call */
      gan_ltmat33_sub_i1 ( &sm33La, &sm33Lb ); /* macro, result in-place in sm33La */
      gan_ltmat33_sub_i2 ( &sm33La, &sm33Lb ); /* macro, result in-place in sm33Lb */
      gan_ltmat33_decrement ( &sm33La, &sm33Lb ); /* equivalent to gan_ltmat33_sub_i1() */
\end{verbatim}

\subsubsection{Rescaling a fixed size matrix}
There are similar functions to multiply a $3\times 4$ matrix by a scalar
\begin{verbatim}
      double ds;

      gan_mat34_scale_q ( &m34A, ds, &m34C ); /* macro */
      m34C = gan_mat34_scale_s ( &m34A, ds ); /* function call */
      gan_mat34_scale_i ( &m34A, ds ); /* macro, result in-place in m34A */
\end{verbatim}
to divide a $3\times 4$ matrix by a (non-zero) scalar
\begin{verbatim}
      gan_mat34_divide_q ( &m34A, ds, &m34C ); /* macro */
      m34C = gan_mat34_divide_s ( &m34A, ds ); /* function call */
      gan_mat34_divide_i ( &m34A, ds ); /* macro, result in-place in m34A */
\end{verbatim}
to negate a $3\times 4$ matrix
\begin{verbatim}
      gan_mat34_negate_q ( &m34A, &m34C ); /* macro */
      m34C = gan_mat34_negate_s ( &m34A ); /* function call */
      gan_mat34_negate_i ( &m34A ); /* macro, result in-place in m34A */
\end{verbatim}
and to scale a $3\times 4$ matrix to unit unit Frobenius norm
\begin{verbatim}
      gan_mat34_unit_q ( &m34A, &m34C ); /* macro */
      m34C = gan_mat34_unit_s ( &m34A ); /* function call */
      gan_mat34_unit_i ( &m34A ); /* macro, result in-place in m34A */
\end{verbatim}
The Frobenius norm of a matrix is the square-root of the sum of squares
of the matrix elements. The Gandalf functions for computing it are
described in~\ref{fixed-det-trace-norms-sec}.

Equivalent routiness to the above for multiplying/dividing a matrix by a
scalar, negating a matrix and scaling a matrix to unit Frobenius norm are
available for square fixed size matrices. Without listing the routines
exhaustively, some examples are
\begin{verbatim}
      gan_mat33_scale_q ( &m33A, ds, &m33C ); /* macro */
      sm33Sc = gan_symmat33_divide_s ( &sm33Sa, ds ); /* function call */
      gan_ltmat33_negate_i ( &sm33La ); /* macro, result in-place in sm33La */
      m33C = gan_mat33_unit_s ( &m33A ); /* function call */
\end{verbatim}

{\tt Error detection:} If zero is passed as the scalar value to the
{\tt ...\_divide\_[qi]()} routines, {\tt NULL} will be returned, while the
{\tt ...\_divide\_s()} routines will abort the program. You should add tests
for division by zero {\em before} calling any of the {\tt ...\_divide\_[qsi]()}
routines. Similarly, the {\tt ...\_unit\_[qsi]()} routines will fail if
the input matrix contains only zeros. If this is a possibility then the
program should check beforehand.

\subsubsection{Transposing a fixed size matrix}
Explicit matrix transposition is not often required in Gandalf, because of
the extensive support for implicit transpose in other matrix operations.
However where necessary, computing the transpose can be achieved using
\begin{verbatim}
      Gan_Matrix33 m33A, m33B;

      /* set up m33A using e.g. gan_mat33_fill_q() */
      gan_mat33_tpose_q ( &m33A, &m33B ); /* B = A^T */
      m33B = gan_mat33_tpose_s ( &m33A ); /* B = A^T */
      gan_mat33_tpose_i ( &m33A ); /* A = A^T, result in-place in A */
\end{verbatim}

\subsubsection{Fixed size vector outer products} \label{fixed-vector-outer-sec}
The outer product operation on two fixed size vectors $\xvec$ and $\yvec$
is the matrix
\[ A = \xvec \yvec\tr
\]
In Gandalf, any pair of vectors can be combined in this way\footnote{So long as
$\xvec$ and $\yvec$ are ordered so that the size of $\xvec$ is less than or
equal to the size of $\yvec$.}. The operation
on a 3-vector and a 4-vector produces a $3\times 4$ matrix, using either of
the routines
\begin{verbatim}
      Gan_Vector3 v3x;
      Gan_Vector4 v4y;
      Gan_Matrix34 m34A;

      /* ... set up v3x and v4y using e.g. gan_vec[34]_fill_q() ... */
      gan_vec34_outer_q ( &v3x, &v4y, &m34A ); /* macro, or */
      m34A = gan_vec34_outer_s ( &v3x, &v4y ); /* function call */
\end{verbatim}
Similar routines {\tt gan\_vec33\_outer\_[qs]()} enable the computation
of the outer product of two 3-vectors, resulting in a $3\times 3$ matrix.
If $\xvec=\yvec$, the result of the outer product is a symmetric matrix
\[ S = \xvec \xvec\tr
\]
Gandalf has special routines for this case:
\begin{verbatim}
      Gan_Vector3 v3x;
      Gan_SquMatrix33 sm33S;

      /* ... set up v3x using e.g. gan_vec3_fill_q() ... */
      gan_vec33_outer_sym_q ( &v3x, &sm33S ); /* macro, or */
      sm33S = gan_vec33_outer_sym_s ( &v3x ); /* function call */
\end{verbatim}

\subsubsection{Fixed size matrix/vector multiplication}
Gandalf supports matrix/vector multiplication with the matrix optionally
being (implicitly) transposed. If the matrix is triangular, Gandalf also
supports multiplication by the inverse of the matrix, computed implicitly
as described in the introduction to this chapter.
These operations can be written as
\begin{eqnarray}
 \yvec &\!\!=\!\!& A\xvec\;\;\;\;\mbox{OR}\;\;\;\; \yvec = A\tr\xvec\;\;\;\mbox{OR} \nonumber \\
 \yvec &\!\!=\!\!& A\inv\xvec\;\;\;\;\mbox{OR}\;\;\;\; \yvec = A\trinv\xvec\;\;\;\mbox{(triangular $A$ only)} \nonumber
\end{eqnarray}
The Gandalf routines involving a $3\times 4$ matrix for the first of these
(no transpose of $A$) is
\begin{verbatim}
      Gan_Vector4 v4x;
      Gan_Vector3 v3y;
      Gan_Matrix34 m34A;

      /* ... set up m34A and v4x ... */
      gan_mat34_multv4_q ( &m34A, &v4x, &v3y ); /* macro, or */
      v3y = gan_mat34_multv4_s ( &m34A, &v4x ); /* function call */
\end{verbatim}
while if $A$ is to be transposed then the routines are
\begin{verbatim}
      Gan_Vector3 v3x;
      Gan_Vector4 v4y;
      Gan_Matrix34 m34A;

      /* ... set up m34A and v3x ... */
      gan_mat34T_multv3_q ( &m34A, &v3x, &v4y ); /* macro, or */
      v4y = gan_mat34T_multv3_s ( &m34A, &v3x ); /* function call */
\end{verbatim}
There are similar routines {\tt gan\_mat33\_multv3\_[qs]()} and
{\tt gan\_mat33T\_multv3\_[qs]()} for $A$ being a general $3\times 3$ matrix.
For symmetric matrices, there is only one pair of routines:
\begin{verbatim}
      Gan_Vector3 v3x, v3y;
      Gan_SquMatrix33 sm33S;

      /* ... set up sm33S using e.g. gan_symmat33_fill_q(), and v3x ... */
      gan_symmat33_multv3_q ( &sm33S, &v3x, &v3y ); /* macro, or */
      v3y = gan_symmat33_multv3_s ( &sm33S, &v3x ); /* function call */
\end{verbatim}
In the case that matrix $A$ is triangular, Gandalf also supports
multiplication of vectors by the inverse of $A$. In this case the result
may be computed in-place in the input vector, So the complete set of
matrix/vector multiplication routines for triangular matrices is
\begin{verbatim}
      Gan_Vector3 v3x, v3y;
      Gan_SquMatrix33 sm33L;

      /* ... set up sm33L using e.g. gan_ltmat33_fill_q(), and v3x ... */

      /* multiply vector by lower triangular matrix */
      gan_ltmat33_multv3_q ( &sm33L, &v3x, &v3y ); /* macro, or */
      v3y = gan_ltmat33_multv3_s ( &sm33L, &v3x ); /* function call, or */
      gan_ltmat33_multv3_i ( &sm33L, &v3x ); /* macro, in-place in v3x */

      /* multiply vector by upper triangular matrix */
      gan_ltmat33T_multv3_q ( &sm33L, &v3x, &v3y ); /* macro, or */
      v3y = gan_ltmat33T_multv3_s ( &sm33L, &v3x ); /* function call, or */
      gan_ltmat33T_multv3_i ( &sm33L, &v3x ); /* macro, in-place in v3x */

      /* multiply vector by inverse of lower triangular matrix */
      gan_ltmat33I_multv3_q ( &sm33L, &v3x, &v3y ); /* macro, or */
      v3y = gan_ltmat33I_multv3_s ( &sm33L, &v3x ); /* function call, or */
      gan_ltmat33I_multv3_i ( &sm33L, &v3x ); /* macro, in-place in v3x */

      /* multiply vector by inverse of upper triangular matrix */
      gan_ltmat33IT_multv3_q ( &sm33L, &v3x, &v3y ); /* macro, or */
      v3y = gan_ltmat33IT_multv3_s ( &sm33L, &v3x ); /* function call, or */
      gan_ltmat33IT_multv3_i ( &sm33L, &v3x ); /* macro, in-place in v3x */
\end{verbatim}

{\bf Error detection:} If implicit inverse is used (the {\tt ...I\_multv...()}
or {\tt ...IT\_multv...()} routines), the matrix must
be non-singular, which for triangular matrices means that none of the
diagonal elements should be zero. If the matrix was produced
by successful Cholesky factorisation of a symmetric matrix
(see Section~\ref{fixed-decomp-sec}) the matrix is guaranteed to be
non-singular. This is the normal method of creating a triangular matrix,
and Gandalf uses {\tt assert()} to check for the singularity of the matrix.

\subsubsection{Fixed size matrix/matrix multiplication}
Most useful matrix product combinations are supported by Gandalf.
Here we describe all the combinations involving $3\times 4$ matrices.
The first functions to describe are those which involve multiplication
by a $3\times 3$ on the left or $4\times 4$ matrix on the right,
the square matrix optionally being (implicitly) transposed,
the product producing another $3\times 4$ matrix.
The operator combinations are
\[ D_{3\times 4} = B_{3\times 3} A_{3\times 4},\;\;\;
   D_{3\times 4} = B_{3\times 3}\tr A_{3\times 4},\;\;\;
   D_{3\times 4} = A_{3\times 4} C_{4\times 4},\;\;\;
   D_{3\times 4} = A_{3\times 4} C_{4\times 4}\tr
\]
which are implemented in Gandalf using the macros
\begin{verbatim}
      Gan_Matrix34 m34A, m34D;
      Gan_Matrix33 m33B;
      Gan_Matrix44 m44C;

      /* ... set up m34A, m33B and m44C ... */
      gan_mat34_lmultm33_q  ( &m34A, &m33B, &m34D ); /* D = B*A   */
      gan_mat34_lmultm33T_q ( &m34A, &m33B, &m34D ); /* D = B*A^T */
      gan_mat34_rmultm44_q  ( &m34A, &m44C, &m34D ); /* D = A*C   */
      gan_mat34_rmultm44T_q ( &m34A, &m44C, &m34D ); /* D = A*C^T */
\end{verbatim}
Equivalent function calls are available:
\begin{verbatim}
      m34D = gan_mat34_lmultm33_s  ( &m34A, &m33B ); /* D = B*A   */
      m34D = gan_mat34_lmultm33T_s ( &m34A, &m33B ); /* D = B*A^T */
      m34D = gan_mat34_rmultm44_s  ( &m34A, &m44C ); /* D = A*C   */
      m34D = gan_mat34_rmultm44T_s ( &m34A, &m44C ); /* D = A*C^T */
\end{verbatim}
Note that although by and large the functions described here for $3\times 4$
matrices are repeated for square matrices, there is redundancy because
in the case of $3\times 3$ matrices the routines
\begin{verbatim}
      m33D = gan_mat33_lmultm33_s  ( &m33A, &m33B ); /* D = B*A   */
      m33D = gan_mat33_rmultm33_s  ( &m33B, &m33A ); /* D = B*A   */
\end{verbatim}
would be equivalent, so in fact only the routines
{\tt gan\_mat33\_rmultm33\_[qs]()} are defined.

The square matrix may be symmetric or triangular, for which cases there
are specific Gandalf functions. Firstly for multiplying by symmetric matrices
we have the routines
\begin{verbatim}
      Gan_Matrix34 m34A, m34B;
      Gan_SquMatrix33 sm33S;
      Gan_SquMatrix44 sm44S;

      /* ... set up m34A, symmetric sm33S and sm44S ... */
      gan_mat34_lmults33_q ( &m34A, &sm33S, &m34B ); /* B = S*A, macro */
      gan_mat34_rmults44_q ( &m34A, &sm44S, &m34B ); /* B = A*S, macro */
\end{verbatim}
with equivalent function calls
\begin{verbatim}
      m34B = gan_mat34_lmults33_q ( &m34A, &sm33S ); /* B = S*A, function call */
      m34B = gan_mat34_rmults44_q ( &m34A, &sm44S ); /* B = A*S, function call */
\end{verbatim}
When multiplying by a triangular matrix, there are also options of implicit
transpose and inverse, as described in the introduction to this chapter.
Gandalf also supports in-place operations in this case. So there is a whole
family of functions covering multiplication of a matrix by a triangular
matrix. Mathematically the operations are
\[ B=LA, \;\;\; B=L\tr A, \;\;\; B=L\inv A, \;\;\; B=L\trinv A
\]
\[ B=AL, \;\;\; B=AL\tr, \;\;\; B=AL\inv, \;\;\; B=AL\trinv
\]
Gandalf macro routines to implement these operations are
\begin{verbatim}
      Gan_Matrix34 m34A, m34B;
      Gan_SquMatrix33 sm33L;
      Gan_SquMatrix44 sm44L;

      /* ... set up m34A, lower triangular sm33L and sm44L ... */
      gan_mat34_lmultl33_q   ( &m34A, &sm33L, &m34B ); /* B = L*A,    macro */
      gan_mat34_lmultl33T_q  ( &m34A, &sm33L, &m34B ); /* B = L^T*A,  macro */
      gan_mat34_lmultl33I_q  ( &m34A, &sm33L, &m34B ); /* B = L^-1*A, macro */
      gan_mat34_lmultl33IT_q ( &m34A, &sm33L, &m34B ); /* B = L^-T*A, macro */
      gan_mat34_rmultl44_q   ( &m34A, &sm44L, &m34B ); /* B = A*L,    macro */
      gan_mat34_rmultl44T_q  ( &m34A, &sm44L, &m34B ); /* B = A*L^T,  macro */
      gan_mat34_rmultl44I_q  ( &m34A, &sm44L, &m34B ); /* B = A*L^-1, macro */
      gan_mat34_rmultl44IT_q ( &m34A, &sm44L, &m34B ); /* B = A*L^-T, macro */
\end{verbatim}
There are also function calls to implement the same operations:
\begin{verbatim}
      m34B = gan_mat34_lmultl33_s   ( &m34A, &sm33L ); /* B = L*A,    function call */
      m34B = gan_mat34_lmultl33T_s  ( &m34A, &sm33L ); /* B = L^T*A,  function call */
      m34B = gan_mat34_lmultl33I_s  ( &m34A, &sm33L ); /* B = L^-1*A, function call */
      m34B = gan_mat34_lmultl33IT_s ( &m34A, &sm33L ); /* B = L^-T*A, function call */
      m34B = gan_mat34_rmultl44_s   ( &m34A, &sm44L ); /* B = A*L,    function call */
      m34B = gan_mat34_rmultl44T_s  ( &m34A, &sm44L ); /* B = A*L^T,  function call */
      m34B = gan_mat34_rmultl44I_s  ( &m34A, &sm44L ); /* B = A*L^-1, function call */
      m34B = gan_mat34_rmultl44IT_s ( &m34A, &sm44L ); /* B = A*L^-T, function call */
\end{verbatim}
Finally there is a set of macros for writing the result in-place into
the $3\times 4$ matrix $A$.
\begin{verbatim}
      gan_mat34_lmultl33_i   ( &m34A, &sm33L ); /* A = L*A,    macro */
      gan_mat34_lmultl33T_i  ( &m34A, &sm33L ); /* A = L^T*A,  macro */
      gan_mat34_lmultl33I_i  ( &m34A, &sm33L ); /* A = L^-1*A, macro */
      gan_mat34_lmultl33IT_i ( &m34A, &sm33L ); /* A = L^-T*A, macro */
      gan_mat34_rmultl44_i   ( &m34A, &sm44L ); /* A = A*L,    macro */
      gan_mat34_rmultl44T_i  ( &m34A, &sm44L ); /* A = A*L^T,  macro */
      gan_mat34_rmultl44I_i  ( &m34A, &sm44L ); /* A = A*L^-1, macro */
      gan_mat34_rmultl44IT_i ( &m34A, &sm44L ); /* A = A*L^-T, macro */
\end{verbatim}

The next set of functions deals with multiplying a matrix by itself in
transpose, resulting in a symmetric matrix. These operations have the form
\[ S_{4\times 4} = A_{3\times 4}\tr A_{3\times 4}\;\;\;\;\mbox{OR}\;\;\;\;
   S_{3\times 3} = A_{3\times 4} A_{3\times 4}\tr
\]
The Gandalf macro routines to implement these operations are
\begin{verbatim}
      Gan_Matrix34 m34A;
      Gan_SquMatrix33 sm33S;
      Gan_SquMatrix44 sm44S;

      /* ... set up m34A using e.g. gan_mat34_fill_q() ... */
      gan_mat34_slmultT_q ( &m34A, &sm44S ); /* S = A^T*A */
      gan_mat34_srmultT_q ( &m34A, &sm33S ); /* S = A*A^T */
\end{verbatim}
with equivalent function calls
\begin{verbatim}
      sm44S = gan_mat34_slmultT_s ( &m34A ); /* S = A^T*A */
      sm33S = gan_mat34_srmultT_s ( &m34A ); /* S = A*A^T */
\end{verbatim}
There are similar routines for general $3\times 3$ matrices. For triangular
matrices the functions are
\begin{verbatim}
      Gan_SquMatrix33 sm33L, sm33S;

      /* ... set up sm33L using e.g. gan_ltmat33_fill_q()... */
      gan_ltmat33_slmultT_q ( &sm33L, &sm33S ); /* S = L^T*L, macro */
      sm33S = gan_ltmat33_slmultT_s ( &sm33L ); /* S = L^T*L, function call */
      gan_ltmat33_srmultT_q ( &sm33L, &sm33S ); /* S = L*L^T, macro */
      sm33S = gan_ltmat33_srmultT_s ( &sm33L ); /* S = L*L^T, function call */
\end{verbatim}
In the case of triangular matrices the operation ``multiply by transposed
self'' can also be done in-place, so we also have the macro routines
\begin{verbatim}
      Gan_SquMatrix33 sm33A;

      /* ... set up sm33A as triangular using e.g. gan_ltmat33_fill_q()... */
      gan_ltmat33_slmultT_i ( &sm33A ); /* A = A^T*A, macro */
      gan_ltmat33_srmultT_i ( &sm33A ); /* A = A*A^T, macro */
\end{verbatim}

There are also routines to multiply a matrix by the transpose
of another matrix of the same size, where the result is {\em assumed} to be
a symmetric matrix. So mathematically the operations have the form
\[ S_{4\times 4} = A_{3\times 4}\tr B_{3\times 4}\;\;\;\;\mbox{OR}\;\;\;\;
   S_{3\times 3} = A_{3\times 4} B_{3\times 4}\tr
\]
The Gandalf macro routines to implement these operations are
\begin{verbatim}
      Gan_Matrix34 m34A, m34B;
      Gan_SquMatrix33 sm33S;
      Gan_SquMatrix44 sm44S;

      /* ... set up m34A, m34B using e.g. gan_mat34_fill_q() ... */
      gan_mat34_rmultm34T_sym_q ( &m34A, &m34B, &sm33S ); /* S = A*B^T */
      gan_mat34_lmultm34T_sym_q ( &m34B, &m34A, &sm44S ); /* S = A^T*B */
\end{verbatim}
with equivalent function calls
\begin{verbatim}
      sm33S = gan_mat34_rmultm34T_sym_s ( &m34A, &m34B ); /* S = A*B^T */
      sm44S = gan_mat34_lmultm34T_sym_s ( &m34B, &m34A ); /* S = A^T*B */
\end{verbatim}

A common operation is to multiply a symmetric matrix on left and right by
a matrix and its transpose, producing another symmetric matrix.
Gandalf supports all combinations of these operations.
Those involving $3\times 4$ matrices are
\begin{verbatim}
      Gan_SquMatrix33 sm33Sa;
      Gan_SquMatrix44 sm44Sb;
      Gan_Matrix34 m34A;

      gan_symmat33_lrmultm34T_q ( &sm33Sa, &m34A, &sm44Sb ); /* Sb = A^T*Sa*A, macro */
      sm44Sb = gan_symmat33_lrmultm34T_s ( &sm33Sa, &m34A ); /* Sb = A^T*Sa*A, function call */
      gan_symmat44_lrmultm34_q ( &sm44Sb, &m34A, &sm33Sa ); /* Sa = A*Sb*A^T, macro */
      sm33Sa = gan_symmat44_lrmultm34_s ( &sm44Sb, &m34A ); /* Sa = A*Sb*A^T, function call */
\end{verbatim}

{\bf Error detection:} If implicit inverse is used (e.g. the
{\tt ...multl33I\_[qsi]()} or {\tt ...multl33IT\_[qsi]()} routines),
the matrix must be non-singular, which for triangular matrices
means that none of the diagonal elements should be zero.
If the matrix was produced
by successful Cholesky factorisation of a symmetric matrix
(see Section~\ref{fixed-decomp-sec}) the matrix is guaranteed to be
non-singular. This is the normal method of creating a triangular matrix,
and Gandalf uses {\tt assert()} to check for the singularity of the matrix.

\subsubsection{Fixed size matrix inverse}
All types of square matrices can be inverted in Gandalf.
If the matrix is singular, {\tt NULL}
is normally returned and an error condition set using
{\tt gan\_err\_register()} (see Section~\ref{error-handling-sec}).
The routines to invert a $3\times 3$ matrix are
\begin{verbatim}
      Gan_Matrix33 m33A, m33B;

      /* ... set up m33A using e.g. gan_mat33_fill_q() ... */
      gan_mat33_invert_q ( &m33A, &m33B ); /* B = A^-1, OR */
      m33B = gan_mat33_invert_s ( &m33A ); /* B = A^-1, OR */
      gan_mat33_invert_i ( &m33A ); /* A = A^-1, in-place in A */
\end{verbatim}
Note that the routine {\tt gan\_mat33\_invert\_s()} returns the structure,
rather than a pointer value, so cannot return an error condition.
Do not use it unless you are SURE that your matrix is non-singular!
The basic routines {\tt gan\_mat33\_invert\_[qs]()} are implemented as
functions rather than macros, because they require temporary variables.
There are similar routines for symmetric matrices
\begin{verbatim}
      Gan_SquMatrix33 sm33Sa, sm33Sb;

      /* ... set up sm33Sa using e.g. gan_symmat33_fill_q() ... */
      gan_symmat33_invert_q ( &sm33Sa, &sm33Sb ); /* Sb = Sa^-1, OR */
      sm33Sb = gan_symmat33_invert_s ( &sm33Sa ); /* Sb = Sa^-1, OR */
      gan_symmat33_invert_i ( &sm33Sa ); /* Sa = Sa^-1, in-place in Sa */
\end{verbatim}
and for triangular matrices
\begin{verbatim}
      Gan_SquMatrix33 sm33La, sm33Lb;

      /* ... set up sm33La using e.g. gan_ltmat33_fill_q() ... */
      gan_ltmat33_invert_q ( &sm33La, &sm33Lb ); /* Lb = La^-1, OR */
      sm33Lb = gan_ltmat33_invert_s ( &sm33La ); /* Lb = La^-1, OR */
      gan_ltmat33_invert_i ( &sm33La ); /* La = La^-1, in-place in La */
\end{verbatim}

If you don't want to invoke the error package when inversion is attempted
on a singular matrix, there is a set of routines which allows to instead to
return the error condition as part of the result. For instance the code
fragment
\begin{verbatim}
      Gan_Matrix33 m33A, m33B;
      int error_code;

      /* ... set up m33A using e.g. gan_mat33_fill_q() ... */
      if ( gan_mat33_invert ( &m33A, &m33B, &error_code ) == NULL )
      {
         /* error found, act on it ... */
      }

      /* no error found */
\end{verbatim}
attempts to invert matrix $A$, and if an error is found, returns NULL,
with the error condition returned in the {\tt error\_code} variable.
For singular matrices the error condition is
{\tt GAN\_ERROR\_SINGULAR\_MATRIX}. There are similar routines
{\tt gan\_symmat33\_invert()} and {\tt gan\_ltmat33\_invert()} for
symmetric and triangular matrices respectively.

\subsubsection{Determinant, trace, norms of fixed size matrix}
\label{fixed-det-trace-norms-sec}
To compute the determinant of a square matrix use one of the routines
\begin{verbatim}
      Gan_Matrix m33A;
      Gan_SquMatrix33 sm33S, sm33L;
      double dDetA, dDetS, dDetL;

      /* set up m33A, sm33S as symmetric and sm33L as lower triangular */
      dDetA = gan_mat33_det_q(&m33A); /* macro computing det(A) */
      dDetA = gan_mat33_det_s(&m33A); /* function computing det(A) */
      dDetS = gan_symmat33_det_q(&sm33S); /* macro computing det(S) */
      dDetS = gan_symmat33_det_s(&sm33S); /* function computing det(S) */
      dDetL = gan_ltmat33_det_q(&sm33L); /* macro computing det(L) */
      dDetL = gan_ltmat33_det_s(&sm33L); /* function computing det(L) */
\end{verbatim}
The routines to compute the trace of a square matrix are similar:
\begin{verbatim}
      Gan_Matrix m33A;
      Gan_SquMatrix33 sm33S, sm33L;
      double dTraceA, dTraceS, dTraceL;

      /* set up m33A, sm33S as symmetric and sm33L as lower triangular */
      dTraceA = gan_mat33_trace_q(&m33A); /* macro computing trace(A) */
      dTraceA = gan_mat33_trace_s(&m33A); /* function computing trace(A) */
      dTraceS = gan_symmat33_trace_q(&sm33S); /* macro computing trace(S) */
      dTraceS = gan_symmat33_trace_s(&sm33S); /* function computing trace(S) */
      dTraceL = gan_ltmat33_trace_q(&sm33L); /* macro computing trace(L) */
      dTraceL = gan_ltmat33_trace_s(&sm33L); /* function computing trace(L) */
\end{verbatim}
For all types of fixed size matrix there are also routines to compute the
sum of squares of the matrix elements, as well as the Frobenius norm,
which is the square-root of the sum of squares. For $3\times 4$ matrices
the routines are
\begin{verbatim}
      Gan_Matrix34 m34A;
      double dSumSA, dFNormA;

      /* ... set up m34A using e.g. gan_mat34_fill_q() ... */
      dSumSA = gan_mat34_sumsqr_q(&m34A); /* macro computing sum(A_ij^2) */
      dSumSA = gan_mat34_sumsqr_s(&m34A); /* function computing sum(A_ij^2) */
      dFNormA = gan_mat34_Fnorm_q(&m34A); /* macro computing sqrt(sum(A_ij^2)) */
      dFNormA = gan_mat34_Fnorm_s(&m34A); /* function computing sqrt(sum(A_ij^2)) */
\end{verbatim}
There are equivalent routines for other types of matrix.

\subsubsection{Fixed size matrix decompositions} \label{fixed-decomp-sec}
Gandalf supports several of the standard matrix decompositions. Cholesky
factorisation applies to any positive definite symmetric matrix $S$, producing
the lower triangular matrix $L$ so that
\[ S = L L\tr
\]
It can be computed for $3\times 3$ symmetric matrices using the routines
\begin{verbatim}
      Gan_SquMatrix33 sm33S, sm33L;

      /* ... set up sm33S using e.g. gan_symmat33_fill_q() ... */
      gan_symmat33_cholesky_q ( &sm33S, &sm33L ); /* L = chol(S) */
      sm33L = gan_symmat33_cholesky_s ( &sm33S ); /* L = chol(S) */
\end{verbatim}
There is also a routine for computing the Cholesky factorisation in-place
in the input matrix, converting an input symmetric matrix $A$ into a lower
triangular matrix:
\begin{verbatim}
      Gan_SquMatrix33 sm33A;

      /* ... set up sm33A as symmetric using e.g. gan_symmat33_fill_q() ... */
      gan_symmat33_cholesky_i ( &sm33A ); /* A = chol(A) */
\end{verbatim}
The routines {\tt gan\_symmat33\_cholesky\_[qi]()} return {\tt NULL} and
invoke the Gandalf error handler\\ {\tt gan\_err\_register()} if the matrix
is not positive definite ({\tt gan\_symmat33\_cholesky\_s()} aborts the
program on error, so don't use it unless you're SURE your matrix is OK!).
If you don't want to invoke the error package when factorisation is attempted
on a non-positive-definite matrix, there is a set of routines which allows
to instead to return the error condition as part of the result.
For instance the code fragment
\begin{verbatim}
      Gan_SquMatrix33 sm33S, sm33L;
      int error_code;

      /* ... set up sm33S using e.g. gan_symmat33_fill_q() ... */
      if ( gan_symmat33_choleksy ( &sm33S, &sm33L, &error_code ) == NULL )
      {
         /* error found, act on it ... */
      }

      /* no error found */
\end{verbatim}
attempts to factorise matrix $S$, and if an error is found, returns NULL,
with the error condition returned in the {\tt error\_code} variable.
For non-positive-definite matrices the error condition is
{\tt GAN\_ERROR\_NOT\_POSITIVE\_DEFINITE}.

Other factorisations are available in Gandalf. Singular value decomposition
(SVD) can be used to decompose almost any matrix $A$ into factors as
\[ A = UWV\tr
\]
where $U$ and $V$ are orthogonal matrices and $W$ is diagonal.
Currently Gandalf supports SVD for $3\times 3$ and $4\times 4$ matrices.
To use the functions for $3\times 3$ matrices, include the header file
\begin{verbatim}
      #include <gandalf/linalg/3x3matrix_svd.h>
\end{verbatim}
There are routines for SVD of a matrix or its transpose, as follows
\begin{verbatim}
      Gan_Matrix33 m33A, m33U, m33VT;
      Gan_Vector3 v3W;

      /* ... set up m33A using e.g. gan_mat33_fill_q() ... */
      gan_mat33_svd ( &m33A, &m33U, &v3W, &m33VT ); /* A = U*W*V^T, OR */
      gan_mat33T_svd ( &m33A, &m33U, &v3W, &m33VT ); /* A^T = U*W*V^T */
\end{verbatim}
These routines return a {\tt Gan\_Bool} result, which is {\tt GAN\_TRUE}
on success and {\tt GAN\_FALSE} on failure.

There are also a number of routines for computing the eigenvalues and
eigenvectors of fixed size matrices. For $3\times 3$ matrices only there
is a routine to compute the eigenvectors and complex eigenvalues of
a $3\times 3$ matrix. To use the routine include the header file
\begin{verbatim}
      #include <gandalf/linalg/3x3matrix_eigen.h>
\end{verbatim}
The matrix $A$ has ``left'' and ``right'' eigenvectors associated with the
same eigenvalues $\lambda_i$, so that the equation
\[ A \vvec_i = \lambda_i \vvec_i
\]
defines the right eigenvectors, and
\[ \uvec_i^H A = \lambda_i \uvec_i^H
\]
defines the left eigenvectors, where $\uvec_i^H$ denotes the conjugate
transpose of vector $\uvec_i$.
The computed eigenvectors are normalized to have Euclidean norm   
equal to one and largest component real. The Gandalf routine that implements
this is built on the equivalent CLAPACK routine {\tt dgeev()}:
\begin{verbatim}
      Gan_Matrix33 m33A; /* matrix to be decomposed */
      Gan_Matrix33 m33UL, m33VR; /* left and right eigenvectors */
      Gan_Vector3 v3lr, v3li; /* real and imaginary parts of eigenvalues */

      /* ... set up m33A using e.g. gan_mat33_fill_q() ... */
      gan_mat33_eigen ( &m33A, &v3lr, &v3li, &m33UL, &m33VR );
\end{verbatim}

The eigenvalues of symmetric matrices are guaranteed to be real.
Routines are available for computing the eigenvalues and eigenvectors of
$3\times 3$ and $4\times 4$ symmetric matrices, based on either the
CLAPACK routine {\tt dspev()} or the CCMath routine {\tt eigval()}.
For $3\times 3$ matrices the routine is declared in the header file
\begin{verbatim}
      #include <gandalf/linalg/3x3matrix_eigsym.h>
\end{verbatim}
Here is an example using the routine
\begin{verbatim}
      Gan_SquMatrix33 sm33S; /* symmetric matrix to be decomposed */
      Gan_Matrix33 m33Z; /* (right) eigenvectors of A */
      Gan_Vector3 v3W; /* eigenvalues */

      /* ... set up sm33S using e.g. gan_symmat33_fill_q() ... */
      gan_symmat33_eigen ( &sm33S, &v3W, &m33Z );
\end{verbatim}

\subsubsection{Fixed size matrix file I/O}
Gandalf supports both ASCII and binary format file I/O of matrices.
Both formats use standard {\tt FILE~*} file streams.
ASCII format is obviously more convenient to use, while binary format is
more compact and guarantees no loss of precision when the data is read.
To print a $3\times 4$ matrix in ASCII format, use
\begin{verbatim}
      Gan_Bool gan_mat34_fprint ( FILE *fp, Gan_Matrix34 *p,
                                  const char *prefix, int indent, const char *fmt );
\end{verbatim}
{\tt prefix} is a prefix string to print before the matrix itself, {\tt indent}
is the number of spaces to indent the matrix by, and {\tt fmt} is a format
string to use when printing the matrix, e.g. {\tt "\%f"}. So for example
\begin{verbatim}
      FILE *pfFile;

      pfFile = fopen ( "/tmp/matrices", "w" );
      gan_mat34_fill_q ( &m34A, 1.0,  2.0,  3.0,  4.0,
                                5.0,  6.0,  7.0,  8.0,
                                9.0, 10.0, 11.0, 12.0 );
      gan_mat34_fprint ( pfFile, &m34A, "Example matrix", 3, "%f" );
\end{verbatim}
will print the output
\begin{verbatim}
         Example matrix
          1.000000 2.000000 3.000000 4.000000
          5.000000 6.000000 7.000000 8.000000
          9.000000 10.000000 11.000000 12.000000
\end{verbatim}
to the file {\tt "/tmp/matrices"}.
There is also a version {\tt gan\_mat34\_print()} for printing to
standard output:
\begin{verbatim}
      Gan_Bool gan_mat34_print ( Gan_Matrix34 *p,
                                 const char *prefix, int indent, const char *fmt );
\end{verbatim}
The corresponding input function is
\begin{verbatim}
      Gan_Bool gan_mat34_fscanf ( FILE *fp, Gan_Matrix34 *p,
                                  char *prefix, int prefix_len )
\end{verbatim}
which reads the matrix from the file stream {\tt fp} into the $3\times 4$
matrix pointer {\tt p}. It also reads the prefix string (up to the specified
maximum length {\tt prefix\_len}), which can be compared with the expected
prefix string to check for consistency.

Binary file I/O is handled by the functions {\tt gan\_mat34\_fwrite()}
and {\tt gan\_mat34\_fread()}. To write a $3\times 4$ matrix in binary
format use
\begin{verbatim}
      Gan_Bool gan_mat34_fwrite ( FILE *fp, Gan_Matrix34 *p, gan_ui32 magic_number );
\end{verbatim}
The {\tt magic\_number} takes the same role as the {\tt prefix} string
in {\tt gan\_mat34\_fprint()}, and is written into the file so that it can be
used later to identify the matrix when it is read back using
\begin{verbatim}
      Gan_Bool gan_mat34_fread ( FILE *fp, Gan_Matrix34 *p, gan_ui32 *magic_number );
\end{verbatim}

There are similar functions {\tt gan\_mat33\_fprint()},
{\tt gan\_mat33\_print()}, {\tt gan\_mat33\_fscanf()} for ASCII I/O of
$3\times 3$ matrices, and {\tt gan\_mat33\_fwrite()}, {\tt gan\_mat33\_fread()}
for binary I/O of $3\times 3$ matrices. Functions for symmetric and
triangular matrices follow the same pattern.

{\tt Error detection:} The I/O routines return a boolean value, returning
{\tt GAN\_TRUE} on success, {\tt GAN\_FALSE} on failure, invoking the Gandalf
error handle in the latter case.

\subsubsection{Conversion from general to fixed size matrix}
Functions are provided to convert a general matrix {\tt Gan\_Matrix}
{\tt Gan\_Matrix34}, provided that the general matrix has actually been
created with size three. So for instance
\begin{verbatim}
      Gan_Matrix *pmMatrix;

      pmMatrix = gan_mat_alloc(3,4);
      gan_mat_fill_va ( pmMatrix, 3, 4, 1.0,  2.0,  3.0,  4.0,
                                        5.0,  6.0,  7.0,  8.0,
                                        9.0, 10.0, 11.0, 12.0 );
      gan_mat34_from_mat_q ( pmMatrix, &m34A );
\end{verbatim}
or
\begin{verbatim}
      m34A = gan_mat34_from_mat_s ( pmMatrix );
\end{verbatim}
fill {\tt m34A} with the same values as the general matrix.
Calling these functions with a general matrix {\tt pmMatrix} not having
the same size as the fixed size matrix is an error.

{\tt Error detection:} The conversion routines return the pointer to the
filled fixed size matrix, or {\tt NULL} on failure, invoking the Gandalf
error handle in the latter case.

\subsubsection{Single precision fixed size matrices}
\begin{verbatim}
      #include <gandalf/linalg/3matrixf.h>
\end{verbatim}
There is an identical set of functions for handling single precision
floating point matrices, the names of which are obtained by replacing
{\tt "gan\_mat34\_..."} in the above functions with {\tt "gan\_mat34f\_..."}.
For example, to add two single precision $3\times 4$ matrices use
\begin{verbatim}
      Gan_Matrix34_f m34A, m34B, m34C;

      /* ... fill m34A and m34B ... */
      gan_mat34f_add_q ( &m34A, &m34B, &m34C );
\end{verbatim}
