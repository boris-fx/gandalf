\chapter{The Image Package} \label{image-chapter}
The {\bf image} package covers images of all formats and types, and defines
low-level image manipulation routines.
To be able to use any routine or structure in the image package
use the declaration
\begin{verbatim}
      #include <gandalf/image.h>
\end{verbatim}
but including individual module header files instead will speed up program
compilation.

\section{Image formats and types} \label{formats-types-sec}
Gandalf distinguishes between the external {\tt file} format of an image
and the an internal image format used by Gandalf to represent the image data.
The former is often used to select the latter, but it is important to
separate the two. The available internal Gandalf image formats are defined
by the {\tt Gan\_ImageFormat} enumerated type, found
in {\tt <gandalf/image/pixel.h>}:
\begin{verbatim}
      typedef enum { GAN_GREY_LEVEL_IMAGE,       /* grey-level images */
                     GAN_GREY_LEVEL_ALPHA_IMAGE, /* grey-level images with alpha
                                                    channel */
                     GAN_RGB_COLOUR_IMAGE,       /* RGB colour images */
                     GAN_RGB_COLOUR_ALPHA_IMAGE, /* RGB colour images with alpha
                                                    channel */
                     GAN_VECTOR_FIELD_2D,        /* 2D image of 2D vectors */
                     GAN_VECTOR_FIELD_3D }       /* 2D image of 3D vectors */
        Gan_ImageFormat;
\end{verbatim}
The formats are (hopefully) self-explanatory, and allow Gandalf to represent
most useful kinds of image data. Along with the format there is also an
image {\bf type}, which determines what type of data is stored in each pixel
of the image. The {\tt Gan\_Type} enumerated type is used to distinguish
the image type. It allows for instance boolean, unsigned character, short
integer or floating point types to be defined, and is described in
Section~\ref{simple-types-sec}.
Note that not all types are supported by each format.
For instance boolean images ({\tt GAN\_BOOL} type) are only supported as
grey-level format images (format {\tt GAN\_GREY\_LEVEL\_IMAGE}), and
the vector field formats {\tt GAN\_VECTOR\_FIELD\_2D} and
{\tt GAN\_VECTOR\_FIELD\_3D} are currently supported only for signed types.
Support for extra types can be added when required.

There is also a structure to represent a rectangular sub-window of an
image, useful in many situations:
\begin{verbatim}
      /* Definition of a rectangular sub-part of an image.
       */
      typedef struct Gan_ImageWindow
      {
         /* offset of window relative to top-left corner of the image */
         unsigned c0, r0;

         /* dimensions of window */
         unsigned width, height;
      } Gan_ImageWindow;
\end{verbatim}

Apart from the standard simple C types, boolean and pointer image types are
supported. These are described in Sections~\ref{binary-image-sec}
and~\ref{pointer-image-sec}.

\section{Simple image/pixel routines} \label{simple-image-sec}
To minimise computational overheads, Gandalf provides a common set
of low level image and pixel handling routines specific to each
supported format and type of image. We will illustrate the routines with
reference to two representative examples:
\begin{enumerate}
  \item Grey-level signed short integer images. The format here is
	{\tt GAN\_GREY\_LEVEL\_IMAGE}, and the type is {\tt GAN\_SHORT}.
	To use the routines specific to this format and type use the
	header file
	\begin{verbatim}
      #include <gandalf/image/image_gl_short.h>
	\end{verbatim}
  \item RGB colour unsigned character images. The format here is
	{\tt GAN\_RGB\_COLOUR\_IMAGE}, and the type is {\tt GAN\_UCHAR}.
	To use the routines specific to this format and type use the
	header file
	\begin{verbatim}
      #include <gandalf/image/image_rgb_uchar.h>
	\end{verbatim}
\end{enumerate}
We need to use two examples because the way that pixels are handled for
simple grey-level images is different to RGB colour and other image formats,
since in the latter and related formats (every format except grey-level)
a pixel is represented as a structure rather than a simple C object.

There are also higher level routines that work with all the formats and
types, accessible through the header file.
\begin{verbatim}
      #include <gandalf/image/image_rgb_uchar.h>
\end{verbatim}
Examples of these are also provided in the following sections.

\subsection{Image creation/destruction}
In the same manner as the general size matrix/vector package, Gandalf images
can be represented either using structures or pointers to structures.
The normal Gandalf convention is to use pointers, because images are
relatively large objects, and the extra overhead of having to use
{\tt malloc()} to create the image structure is insignificant relative to
the computation time needed to process the image. We will follow this
convention, but bear in mind that either convention is possible.
To create a grey-level short integer image, use the routine
\begin{verbatim}
      Gan_Image *pImage;

      pImage = gan_image_alloc_gl_s ( 150, 100 );
\end{verbatim}
This creates an image with dimensions 150 (height) by 100 (width).
The same operation using an image structure rather than a pointer would be
\begin{verbatim}
      Gan_Image Image;

      gan_image_form_gl_s ( &Image, 150, 100 );
\end{verbatim}
To free the image, use
\begin{verbatim}
      gan_image_free ( pImage ); /* OR */
      gan_image_free ( &Image );
\end{verbatim}
This function can be used universally to free a Gandalf image created in
any of the ways described here.

Sometimes the image data array is already present in memory, and we want
to create a Gandalf image that points to the data. Let us assume that the
data for a $256\times 256$ pixel grey-level short integer image is available
in an array {\tt asData}. Assume for now that it is a contiguous array,
stored with rows following rows consecutively without any gaps, i.e. an
array of 65536 elements. Then to build a Gandalf image that points into
this data we might use the code
\begin{verbatim}
      short asData[65536];

      /* ... set up array asData with image data ... */
      pImage = gan_image_alloc_data_gl_s ( 256, 256, 256*sizeof(short),
                                           asData, 65536, NULL, 0 ); /* OR */
      gan_image_form_data_gl_s ( &Image, 256, 256, 256*sizeof(short),
                                 asData, 65536, NULL, 0 );
\end{verbatim}
After the height and width arguments is a ``stride'' argument, which indicates
the separation in memory between adjacent rows of the image, as stored in
the {\tt asData} array. Here it is 256 pixels (the image width), but since
stride is measured in bytes, we need to multiply by the pixel size, as here.
The data array {\tt asData} is passed in along with its size in pixels (65536).
The size is passed mainly as a means of error checking: if the requested
Gandalf image as defined by the height, width and stride were to exceed the
size of the data array, it would be an error, the Gandalf error handler
would be invoked, and {\tt NULL} would be returned. Here the data array size
and the image size match exactly. The final two arguments allow the programmer
also to pass in an array of {\em row pointers} which point into the start
of each row of the image. Here we pass {\tt NULL} for the row pointers, which
means that they will be allocated inside the function.

Note that the {\tt Gan\_Image} structure stores the information concerning
which parts of the structure were dynamically allocated: the structure itself,
the image data array and the row pointer array. {\tt gan\_image\_free()}
then knows which bits to free.

A slightly more complex example is when the rows of the image as stored
in the data array are not contiguous in memory. This might happen for
instance in frame-grabber (video) memory, where the hardware might restrict
the stride to a fixed number of bytes, say 1024. We shall also provide
an array of row pointers to the image creation function. Then we would have to
call the above functions as follows:
\begin{verbatim}
      short *psData, *apRowPointer[256];

      /* ... set psData to point to video memory ... */
      pImage = gan_image_alloc_data_gl_s ( 256, 256, 1024,
                                           asData, 65536, apRowPointer, 256 ); /* OR */
      gan_image_form_data_gl_s ( &Image, 256, 256, 256*sizeof(short),
                                 asData, 131072, apRowPointer, 256 );
\end{verbatim}
Here we assume that {\tt short}s are 2 bytes. These function calls will
set the Gandalf image to point directly into the video memory, so that
if desired the image stored may be copied for further processing (see below)
or processed directly.

For RGB unsigned character images, the function calls would be similar:
\begin{verbatim}
      Gan_Image *pImage, Image;
      Gan_RGBPixel_uc argbucData[65536];

      pImage = gan_image_alloc_rgb_uc ( 150, 100 ); /* OR */
      gan_image_form_rgb_uc ( &Image, 150, 100 ); /* OR */

      /* ... set up array aucData with image data ... */
      pImage = gan_image_alloc_data_rgb_uc ( 256, 256, 256*sizeof(unsigned char),
                                             argbucData, 65536, NULL, 0 ); /* OR */
      gan_image_form_data_rgb_uc ( &Image, 256, 256, 256*sizeof(unsigned char),
                                   argbucData, 65536, NULL, 0 );
\end{verbatim}
For RGB and other similar formats, Gandalf assumes that the channels for
each pixel are grouped in memory, so that a pixel can be represented as
a structure, rather than the channels being stored in separate arrays.
For RGB unsigned character images, the pixel structure is
{\tt Gan\_RGBPixel\_uc}, as defined in {\tt <gandalf/image/pixel.h>}:
\begin{verbatim}
      /* Structure defining RGB colour unsigned character pixel
       */
      typedef struct Gan_RGBPixel_uc
      {
         unsigned char R, G, B;
      } Gan_RGBPixel_uc;
\end{verbatim}
A different structure type is defined for each image format
(apart from grey-level) and type.

There are also higher level functions which create a Gandalf images using
arguments to determine the format and type. Use these functions only if
the format/type is determined at run-time. An example emulating the above
examples for grey-level images is
\begin{verbatim}
      Gan_Image *pImage, Image;

      pImage = gan_image_alloc ( GAN_GREY_LEVEL_IMAGE, GAN_SHORT, 150, 100 ); /* OR */
      gan_image_form ( &Image, GAN_GREY_LEVEL_IMAGE, GAN_SHORT, 150, 100 );
\end{verbatim}

{\bf Error detection:} All the above routines return {\tt NULL} and invoke the
Gandalf error handler if they fail. The most likely failure modes are failing
to allocate the data required (i.e. internal {\tt malloc()} or {\tt realloc()}
calls failing), or passing too small an array into the
{\tt ...\_alloc\_data...()} or  {\tt ...\_form\_data...()} routines.

\section{Image file I/O}
\begin{verbatim}
      #include <gandalf/image/io/image_io.h>
\end{verbatim}
Currently Gandalf supports six image file formats: PNG, PBM, PGM, PPM, TIFF
and JPEG. These are described by the\\ {\tt Gan\_ImageFormat} enumerated type:
\begin{verbatim}
      /* image file formats supported by Gandalf */
      typedef enum
      {
         GAN_PNG_FORMAT,  /**< PNG image format */
         GAN_PBM_FORMAT,  /**< Portable bitmap image format */
         GAN_PGM_FORMAT,  /**< Portable greymap image format */
         GAN_PPM_FORMAT,  /**< Portable pixmap image format */
         GAN_TIFF_FORMAT, /**< TIFF image format */
         GAN_JPEG_FORMAT, /**< JPEG image format */
         GAN_UNKNOWN_FORMAT /**< Unknown Image Format */
      } Gan_ImageFileFormat;
\end{verbatim}
PBM, PGM and PPM are very simple formats, for boolean, grey-level and
RGB colour image formats respectively, and the code to implement I/O in those
formats is built into Gandalf, although currently only binary file formats
are supported. PNG, TIFF and JPEG formats are considerably
more complex, and require specific libraries to be installed.
The Gandalf {\tt configure} script detects the presence of the PNG,
TIFF and JPEG libraries, and only compiles
in the I/O code for those formats when the relevant libraries
are detected on the host system.

The {\tt image\_io.h} header file contains declaration of the basic
image I/O functions. To read a image from a PNG image file, for instance,
you can use the code
\begin{verbatim}
      Gan_Image *pImage;

      /* read the image from a file in PNG format */
      pImage = gan_image_read ( "image1.png", GAN_PNG_FORMAT, NULL );
\end{verbatim}
The first argument is the file name, the second the file format (Gandalf
doesn't currently support automatic file format determination via magic
numbers. Who wants to volunteer?). The last argument is either a pointer
to an already created image structure or {\tt NULL}, as here. In the latter
case the image is created inside the {\tt gan\_image\_read()} function
and returned.

To write an RGB unsigned character image to a PNG file you could write
\begin{verbatim}
      Gan_Image *pRGBImage;

      /* ... create and fill RGB unsigned character image ... */

      /* output the image to a file in PNG format */
      gan_image_write ( "image1.png", GAN_PNG_FORMAT, pRGBImage, 0.0 );
\end{verbatim}
We recommend that where possible you should use the PNG format. It is the
most flexible of the formats supported by Gandalf, allowing alpha channels
to be stored with the image, and also supporting binary images. PPM images
are restricted to unsigned character type ({\tt GAN\_UCHAR}), while
PGM format supports unsigned character and binary ({\tt GAN\_BOOL}) type.
However the binary support in PGM files is very inefficient, storing
one byte per pixel, so again PNG is the better format.

\subsection{Setting an image to a new format, type and dimensions}
Once an image has been created with a certain format, type, width and height,
that is not the end of the matter. In a similar manner to the Gandalf
general size vectors and matrices, the format, type and dimensions of a
Gandalf image may be changed an arbitrary number of times. Gandalf will
reallocate the image data array and row pointer array as necessary,
as the internal attributes of the image are changed. However note that
if the image data array or row pointer array is passed in by the
user program, as in {\tt gan\_image\_alloc\_data\_gl\_short()} above,
The provided array(s) cannot be reallocated, so care should be taken
never to attempt to set format/type/dimensions that cause the array bounds
to be exceeded.

Let us take one of the above examples and modify it a bit.
\begin{verbatim}
      Gan_Image *pImage;

      /* create image */
      pImage = gan_image_alloc_gl_s ( 150, 100 );

      /* convert an existing image to new format, type & dimensions */
      gan_image_set_rgb_uc ( pImage, 300, 200 );
\end{verbatim}
This code fragment allocates an image as a $150\times 100$ grey-level short
integer image, and then converts it into a $300\times 200$ RGB unsigned
character image. This feature allows the same image to be used as the
result of several different computations, easing the burden of keeping
track of a large number of images, as well as potentially saving memory.

There are also higher level functions that set an image to a format,
type and dimensions all selected by variables. For instance
\begin{verbatim}
      Gan_Image *pImage;

      /* create image */
      pImage = gan_image_alloc_gl_s ( 150, 100 );

      /* convert an existing image to new format, type & dimensions */
      gan_image_set_format_type_dims ( pImage, GAN_VECTOR_FIELD_2D, GAN_FLOAT,
                                       200, 50 );
\end{verbatim}
sets the image {\tt pImage} to be a 2D vector field (2D image of 2-vectors),
float type and dimensions 200 by 50. There are also routines for setting
the format, type or dimensions, leaving the other attributes fixed.
So for instance
\begin{verbatim}
      gan_image_set_format_type ( pImage, GAN_VECTOR_FIELD_2D, GAN_FLOAT );
\end{verbatim}
sets just the format and type of the image, leaving the dimensions unchanged,
while
\begin{verbatim}
      gan_image_set_type ( pImage, GAN_FLOAT );
\end{verbatim}
sets only the image type, leaving the format and dimensions unchanged.
Finally
\begin{verbatim}
      gan_image_set_dims ( pImage, 200, 50 );
\end{verbatim}
changes only the image dimensions.

{\bf Error detection:} All the above routines return {\tt NULL} and invoke the
Gandalf error handler if they fail. The most likely failure mode is failing
to reallocate the image data, i.e. an internal {\tt realloc()} call failing.

\subsection{Accessing single pixels}
To return the value of a single image pixel use
\begin{verbatim}
      Gan_Image *pImage;
      short sPixel;

      /* ... create and fill grey-level short integer image pImage ... */
      sPixel = gan_image_get_pix_gl_s ( pImage, 33, 40 );
\end{verbatim}
This returns the pixel value at row position 33 and column position 40
(starting from zero). The RGB colour version would be
\begin{verbatim}
      Gan_Image *pImage;
      Gan_RGBPixel_uc rgbucPixel;

      /* ... create and fill RGB colour unsigned character image pImage ... */
      rgbucPixel = gan_image_get_pix_rgb_uc ( pImage, 33, 40 );
\end{verbatim}

An alternative is to return a pointer to a pixel. This operation is available
for every type and format of image except binary images, which are stored
packed with 32 or 64 pixels to a memory word. To return a pointer to the
above pixels you would use
\begin{verbatim}
      Gan_Image *pImage;
      short *psPixel;

      /* ... create and fill grey-level short integer image pImage ... */
      psPixel = gan_image_get_pixptr_gl_s ( pImage, 33, 40 );
\end{verbatim}
for a grey-level image, or
\begin{verbatim}
      Gan_Image *pImage;
      Gan_RGBPixel_uc *prgbucPixel;

      /* ... create and fill RGB colour unsigned character image pImage ... */
      prgbucPixel = gan_image_get_pixptr_rgb_uc ( pImage, 33, 40 );
\end{verbatim}
for an RGB colour image. This type of image access is useful when you want
to read or set a lot of consecutive pixels on a row of an image, since you
can use the returned pointer as a starting point. For instance the code
fragment
\begin{verbatim}
      Gan_Image *pImage;
      Gan_RGBPixel_uc *prgbucPixel, rgbucZeroPixel = {0,0,0};
      int iCount;

      /* ... create and fill RGB colour unsigned character image pImage ... */
      prgbucPixel = gan_image_get_pixptr_rgb_uc ( pImage, 33, 40 );
      for ( iCount = 4; iCount >= 0; iCount-- )
         *prgbucPixel++ = rgbucZeroPixel;
\end{verbatim}
sets the five RGB pixels at positions (33,40-44) to zero.

To set a pixel in a grey-level short integer image to a particular value, use
the routine
\begin{verbatim}
      Gan_Image *pImage;

      /* ... create grey-level short integer image pImage ... */
      gan_image_set_pix_gl_s ( pImage, 33, 40, 123 );
\end{verbatim}
This sets the pixel value at position 33, 40 to value 123. For an RGB unsigned
character image you would use the code
\begin{verbatim}
      Gan_Image *pImage;
      Gan_RGBPixel_uc rgbucPixel = {12, 13, 14};

      /* ... create RGB colour unsigned character image pImage ... */
      gan_image_set_pix_rgb_uc ( pImage, 33, 40, &rgbucPixel );
\end{verbatim}
This builds a pixel with RGB values 12 (red), 13 (green), 14 (blue) and
sets the pixel at position 33, 40 to that RGB value.

With {\tt NDEBUG} set these routines evaluate to macros which implement
direct memory access, so there is no efficiency advantage to be gained
from using other methods of accessing individual image pixels.

The higher level routines for accessing single pixels use the {\tt Gan\_Pixel}
structure, which can be used to store data for a single pixel of any format
and type. The {\tt Gan\_Pixel} structure stores the format and type of the
pixel internally, and is defined in {\tt <gandalf/image/pixel.h>}:
\begin{verbatim}
      /* structure definition for image pixel of any format or type */
      typedef struct Gan_Pixel
      {
         /// format of image: grey-level, RGB colour etc.
         Gan_ImageFormat format;

         /// type of pixel values: unsigned char, float etc.
         Gan_Type type;

         /// nested union defining pixel types
         union
         {
            /// grey level
            union
            {
               unsigned char  uc;
               short          s;
               unsigned short us;
               int            i;
               unsigned int   ui;
               double         d;
               float          f;
               Gan_Bool       b;
               void          *p;

      #ifdef GAN_UINT8
               gan_ui8 ui8;
      #endif
      #ifdef GAN_UINT16
               gan_ui16 ui16;
      #endif
      #ifdef GAN_UINT32
               gan_ui32 ui32;
      #endif
            } gl;

            /// grey level with alpha channel
            union
            {
               Gan_GLAPixel_uc uc;
               Gan_GLAPixel_s  s;
               Gan_GLAPixel_us us;
               Gan_GLAPixel_i  i;
               Gan_GLAPixel_ui ui;
               Gan_GLAPixel_d  d;
               Gan_GLAPixel_f  f;

      #ifdef GAN_UINT8
               Gan_GLAPixel_ui8 ui8;
      #endif
      #ifdef GAN_UINT16
               Gan_GLAPixel_ui16 ui16;
      #endif
      #ifdef GAN_UINT32
               Gan_GLAPixel_ui32 ui32;
      #endif
            } gla;

            /// RGB colour
            union
            {
               Gan_RGBPixel_uc uc;
               Gan_RGBPixel_s  s;
               Gan_RGBPixel_us us;
               Gan_RGBPixel_i  i;
               Gan_RGBPixel_ui ui;
               Gan_RGBPixel_d  d;
               Gan_RGBPixel_f  f;

      #ifdef GAN_UINT8
               Gan_RGBPixel_ui8 ui8;
      #endif
      #ifdef GAN_UINT16
               Gan_RGBPixel_ui16 ui16;
      #endif
      #ifdef GAN_UINT32
               Gan_RGBPixel_ui32 ui32;
      #endif
            } rgb;

            /// RGB colour with alpha channel
            union
            {
               Gan_RGBAPixel_uc uc;
               Gan_RGBAPixel_s  s;
               Gan_RGBAPixel_us us;
               Gan_RGBAPixel_i  i;
               Gan_RGBAPixel_ui ui;
               Gan_RGBAPixel_d  d;
               Gan_RGBAPixel_f  f;

      #ifdef GAN_UINT8
               Gan_RGBAPixel_ui8 ui8;
      #endif
      #ifdef GAN_UINT16
               Gan_RGBAPixel_ui16 ui16;
      #endif
      #ifdef GAN_UINT32
               Gan_RGBAPixel_ui32 ui32;
      #endif
            } rgba;

            /// 2D vector field
            union
            {
               Gan_Vector2_f f;
               Gan_Vector2   d;
               Gan_Vector2_s s;
               Gan_Vector2_i i;
            } vfield2D;

            /// 3D vector field
            union
            {
               Gan_Vector3_f f;
               Gan_Vector3   d;
               Gan_Vector3_s s;
               Gan_Vector3_i i;
            } vfield3D;
         } data;
      } Gan_Pixel;
\end{verbatim}
The {\tt Gan\_Pixel} structure should be accessed directly. There are no
Gandalf access routines for it. The doubly nested union contains a structure
for each Gandalf image format and type. These structures are also defined
in the {\tt pixel.h} header file. We have seen the definition of the
{\tt Gan\_RGBPixel\_uc} structure above, and the other structures are
defined similarly. For instance the pixel to represent a single-precision
floating point RGB pixel with alpha channel is
\begin{verbatim}
      /**
       * \brief Structure defining RGB single precision floating point pixel with alpha channel.
       */
      typedef struct Gan_RGBAPixel_f
      {
         float R, /**< Red channel */
               G, /**< Green channel */
               B, /**< Blue channel */
               A; /**< Alpha channel */
      } Gan_RGBAPixel_f;
\end{verbatim}
Note that the vector field pixels use Gandalf fixed size vectors to hold
the image data.

To set/get a pixel in an image using the higher level routines
{\tt gan\_image\_set\_pix()} and {\tt gan\_image\_get\_pix()},
look at the following code fragment.
\begin{verbatim}
      Gan_Image *pImage;
      int iRow, iCol;
      Gan_Pixel Pixel;

      /* create grey-level signed short image */
      pImage = gan_image_alloc_gl_s ( 200, 100 );

      /* set up pixel format and type */
      Pixel.format = GAN_GREY_LEVEL_IMAGE;
      Pixel.type = GAN_SHORT;

      /* fill image with ramp data */
      for ( iRow = (int)pImage->height-1; iRow >= 0; iRow-- )
         for ( iCol = (int)pImage->width-1; iCol >= 0; iCol-- )
         {
            /* set pixel data */
            Pixel.data.gl.s = iRow+iCol;

            /* fill pixel in image. The format and type of the pixel should
               be the same as that of the image */
            gan_image_set_pix ( pImage, iRow, iCol, &Pixel );
         }

      /* print pixel value, should be 27+35 = 62 */
      Pixel = gan_image_get_pix ( pImage, 27, 35 );
      printf ( "pixel value = %d\n", Pixel.data.gl.s );
\end{verbatim}
Here we created an image, filled it with ``ramp'' data that linearly
increases the grey-level value with the row and column coordinates of the
image, and extract a single pixel. For an RGB image we could add the
following code:
\begin{verbatim}
      /* convert the image to RGB format and unsigned character type */
      gan_image_set_rgb_uc ( pImage, 100, 50 );

      /* set up pixel format and type */
      Pixel.format = GAN_RGB_COLOUR_IMAGE;
      Pixel.type = GAN_UCHAR;

      /* fill image with ramp data */
      for ( iRow = (int)pImage->height-1; iRow >= 0; iRow-- )
         for ( iCol = (int)pImage->width-1; iCol >= 0; iCol-- )
         {
            /* set pixel data */
            Pixel.data.rgb.uc.R = iRow+iCol;
            Pixel.data.rgb.uc.G = iRow;
            Pixel.data.rgb.uc.B = iCol;

            /* fill pixel in image. The format and type of the pixel should
               be the same as that of the image */
            gan_image_set_pix ( pImage, iRow, iCol, &Pixel );
         }

      /* print pixel value, should be R=37+11=48, G=37, B=11 */
      Pixel = gan_image_get_pix ( pImage, 37, 11 );
      printf ( "pixel value R=%d G=%d B=%d\n",
               Pixel.data.rgb.uc.R, Pixel.data.rgb.uc.G, Pixel.data.rgb.uc.B );
\end{verbatim}

If you have a {\tt Gan\_Pixel} structure in a different format/type to the
image, use {\tt gan\_image\_convert\_pixel\_[qsi]()} to convert
it to the format \& type of the image before calling
{\tt gan\_image\_set\_pix()}. See Section~\ref{convert-pixel-sec} for
details.

{\bf Error detection:} The {\tt ..\_get\_pix...()} routines cannot return
an error condition. Instead they invoke\\ {\tt gan\_assert()}
(see Section~\ref{debugging-tools-sec}) to check for errors, which aborts the
program if an error is found. The\\ {\tt ...\_set\_pix...()} routines return
a boolean value, returning {\tt GAN\_TRUE} on success, invoking the
Gandalf error handler and returning {\tt GAN\_FALSE} on failure.
The most likely failure modes are accessing a pixel outside the image
(both {\tt ...\_get\_pix...()} and {\tt ...\_set\_pix...()}) and
mismatch between image and pixel format/type ({\tt ...\_set\_pix...()} only).
These errors are program bugs rather than data-dependent errors, so
using {\tt gan\_assert()} to handle errors is fairly safe.

\subsection{Filling an image with a constant value}
To fill a grey-level image with a constant value use this routine:
\begin{verbatim}
      Gan_Image *pImage;

      /* create grey-level signed short image */
      pImage = gan_image_alloc_gl_s ( 200, 100 );

      /* fill with constant */
      gan_image_fill_const_gl_s ( pImage, 23 );
\end{verbatim}
which fills each pixel with the value 23. For other formats of image
you will need to build a structure of the relevant type, for instance
\begin{verbatim}
      Gan_Image *pImage;
      Gan_RGBPixel_uc rgbucPixel;

      /* create RGB unsigned character image */
      pImage = gan_image_alloc_rgb_uc ( 200, 100 );

      /* set up pixel */
      rgbucPixel.R = 34;
      rgbucPixel.G =  2;
      rgbucPixel.B = 65;

      /* fill with constant RGB value */
      gan_image_fill_const_rgb_uc ( pImage, &rgbucPixel );
\end{verbatim}

Higher level routines are available using the {\tt Gan\_Pixel} structure:
\begin{verbatim}
      Gan_Image *pImage;
      Gan_Pixel Pixel;

      /* create RGBA single precision floating point image */
      pImage = gan_image_alloc_rgba_f ( 200, 100 );

      /* set up pixel */
      Pixel.format = GAN_RGB_COLOUR_ALPHA_IMAGE;
      Pixel.type = GAN_FLOAT;
      Pixel.data.rgba.f.R = 0.1F;
      Pixel.data.rgba.f.G = 0.2F;
      Pixel.data.rgba.f.B = 0.3F;
      Pixel.data.rgba.f.A = 0.4F;

      /* fill with constant RGBA value. The format & type of the pixel and
         image should match */
      gan_image_fill_const ( pImage, &Pixel );
\end{verbatim}

If the {\tt Gan\_Pixel} structure has a different format/type to the
image, use {\tt gan\_image\_convert\_pixel\_[qsi]()} to convert
it to the format \& type of the image before calling
{\tt gan\_image\_fill\_const()}. See Section~\ref{convert-pixel-sec} for
details.

There is a special function {\tt gan\_image\_fill\_zero()} to fill an image
with zero, whatever format and type it has:
\begin{verbatim}
      Gan_Image *pImage;

      /* create RGBA single precision floating point image */
      pImage = gan_image_alloc_rgba_f ( 200, 100 );

      /* set all image pixels to zero */
      gan_image_fill_zero ( pImage );
\end{verbatim}

For boolean images (Section~\ref{binary-image-sec}), ``zero'' is interpreted
as false ({\tt GAN\_FALSE}), and for pointer images
(Section~\ref{pointer-image-sec}) ``zero'' means {\tt NULL}.
To fill a single pixel with zero, use
\begin{verbatim}
      /* set a single pixel at position row=10, column=21 to zero */
      gan_image_set_pix_zero ( pImage, 10, 21 );
\end{verbatim}

There are also routines to fill a rectangular sub-region of an image,
either with a constant value or zero:
\begin{verbatim}
      Gan_Image *pImage;
      Gan_Pixel Pixel;

      /* create grey-level signed short image */
      pImage = gan_image_alloc_gl_s ( 200, 100 );

      /* set pixels in 30x40 (heightxwidth) pixel region starting at position
         100,30 (row,column) to constant value 125 */
      Pixel.format = GAN_GREY_LEVEL_IMAGE;
      Pixel.type = GAN_SHORT;
      Pixel.data.gl.s = 125;
      gan_image_fill_const_window ( pImage, 100, 30, 30, 40, &Pixel );

      /* reset image to RGB unsigned character */
      gan_image_set_rgb_uc ( pImage, 100, 50 );

      /* set pixels in 20x15 (heightxwidth) pixel region starting at position
         10,35 (row,column) to zero */
      gan_image_fill_zero_window ( pImage, 10, 35, 20, 15 );
\end{verbatim}

{\bf Error detection:} The image filling routines return a boolean value,
so a return value of {\tt GAN\_FALSE} indicates failure, with the Gandalf
error handling module being invoked.

\subsection{Converting a pixel to a given format/type}
\label{convert-pixel-sec}
Gandalf routines taking {\tt Gan\_Pixel} structure pointers as arguments,
such as {\tt gan\_image\_fill\_const()}, require that the format and type
of the pixel and image arguments match. This can be done by using the
routines in this section. To convert a pixel to a specific format and type
use the routine
\begin{verbatim}
      Gan_Pixel Pixel1, Pixel2; /* declare pixels 1 & 2 */

      /* let's initialise pixel 1 to a grey-level unsigned character value */
      Pixel1.format = GAN_GREY_LEVEL_IMAGE;
      Pixel1.type = GAN_UCHAR;
      Pixel1.data.gl.uc = 255;

      /* now convert pixel to RGB format and floating point type */
      gan_image_convert_pixel_q ( &Pixel1, GAN_RGB_COLOUR_IMAGE, GAN_FLOAT,
                                  &Pixel2 );

      /* print new pixel value, which should be R=G=B=1 */
      printf ( "pixel RGB value %f %f %f\n", Pixel2.data.rgb.f.R,
               Pixel2.data.rgb.f.G, Pixel2.data.rgb.f.B );
\end{verbatim}
Another version of this function returns the result as a new pixel:
\begin{verbatim}
      /* convert pixel to RGB format and floating point type */
      Pixel2 = gan_image_convert_pixel_s ( &Pixel1, GAN_RGB_COLOUR_IMAGE, GAN_FLOAT );
\end{verbatim}
There is also a routine that converts the format and type in-place in the
input pixel:
\begin{verbatim}
      /* convert pixel to RGB format and floating point type in-place */
      gan_image_convert_pixel_i ( &Pixel1, GAN_RGB_COLOUR_IMAGE, GAN_FLOAT );
\end{verbatim}

\section{Binary images} \label{binary-image-sec}
\begin{verbatim}
      #include <gandalf/image/image_bit.h>
\end{verbatim}
Gandalf binary images support compact storage of an array of boolean values.
Binary images have format {\tt GAN\_GREY\_LEVEL\_IMAGE} and type
{\tt GAN\_BOOL}. The complete set of functions described above is
available for binary images, as well as other special functions.
Here is an illustration of using the standard routines.
\begin{verbatim}
      Gan_Image *pImage;
      Gan_Pixel Pixel;

      /* allocate 300x200 binary image, and initialise all values to
         zero (false) */
      pImage = gan_image_alloc_b ( 300, 200 );
      gan_image_fill_zero(pImage);

      /* fill rectangular region of image with ones (true) */
      Pixel.format = GAN_GREY_LEVEL_IMAGE;
      Pixel.type = GAN_BOOL;
      Pixel.data.gl.b = GAN_TRUE;
      gan_image_fill_const_window ( pImage, 120, 100, 40, 30, &Pixel );

      /* reset size of image and fill with zero again */
      gan_image_set_b ( pImage, 400, 600 );
      gan_image_fill_zero(pImage);

      /* set some other pixels to one (true) */
      gan_image_set_pix_b ( pImage, 250,   4, GAN_TRUE );
      gan_image_set_pix_b ( pImage,  50, 140, GAN_TRUE );
      gan_image_set_pix_b ( pImage, 150,  40, GAN_TRUE );

      /* free image */
      gan_image_free ( pImage );
\end{verbatim}

Several other routines are provided for binary images. Firstly there is
a routine to return the ``active'' region of an image, defined as the
bounding box around the pixels set to one:
\begin{verbatim}
      Gan_ImageWindow SubWindow;

      /* ... set pImage as a binary image with ones and zeros ... */

      /* determine image window surrounding "active" pixels, i.e. those
         set to one */
      gan_image_get_active_subwindow_b ( pImage, GAN_WORD_ALIGNMENT,
                                         &SubWindow );
\end{verbatim}
The {\tt Gan\_ImageWindow} result structure was described in
Section~\ref{formats-types-sec}. The second argument defines how
precisely to determine the horizontal limits of the bounding box.
The coarsest method is to find the limits to the nearest word, as in the
above example. More precise but slower alignment is possible using
either {\tt GAN\_BYTE\_ALIGNMENT} or {\tt GAN\_BIT\_ALIGNMENT}.

To compute the number of active bits in a binary image use
\begin{verbatim}
      int iCount;

      iCount = gan_image_get_pixel_count_b ( pImage, GAN_TRUE, NULL );
\end{verbatim}
The second argument is {\tt GAN\_TRUE} to count the ones or {\tt GAN\_FALSE}
to count the zeroes. The last argument is an optional pointer to a
sub-window of the image in which to apply the count.

There are functions to return a boolean value indicating whether a local
group of pixels are all set to one. These routines are
\begin{verbatim}
      /* check whether the group of four pixels at positions (100,100),
         (100,101), (101,100), (101,101) are all set to one */
      if ( gan_image_pix_get_pix_4group ( pImage, 100, 100 ) )
         printf ( "group of four found\n" );

      /* check whether the group of four pixels at positions (99,100),
         (100,99), (100,100), (100,101) and (101,100) are all set to one */
      if ( gan_image_pix_get_pix_5group ( pImage, 100, 100 ) )
         printf ( "group of five found\n" );

      /* check whether the group of three pixels at positions (100,99),
         (100,100), (100,101) are all set to one */
      if ( gan_image_pix_get_pix_3group_horiz ( pImage, 100, 100 ) )
         printf ( "group of three horizontally found\n" );

      /* check whether the group of three pixels at positions (99,100),
         (100,100), (101,100) are all set to one */
      if ( gan_image_pix_get_pix_3group_vert ( pImage, 100, 100 ) )
         printf ( "group of three vertically found\n" );
\end{verbatim}

There is a set of functions to apply a boolean operation to every pixel
in one image or a pair of images. Firstly there is are routines to invert
a boolean image:
\begin{verbatim}
      Gan_Image *pImage1, *pImage2, *pImage3;

      /* ... create and fill image 1 as a boolean image, create image 2 ... */

      gan_image_bit_invert_q ( pImage1, pImage2 ); /* invert image 1 into image 2, OR */
      pImage3 = gan_image_bit_invert_s ( pImage1 ); /* invert image 1 as a new image, OR */
      gan_image_bit_invert_i ( pImage1 ); /* replace image 1 with its inverse */
\end{verbatim}
Then there are routines to apply the operations AND, OR, exclusive-OR (EOR) and
not-AND (NAND) to a pair of binary images, which must have the same dimensions.
Illustrating the AND operation first, we have the options
\begin{verbatim}
      Gan_Image *pImage1, *pImage2, *pImage3, *pImage4;

      /* ... create and fill images 1 & 2 as boolean images, create image 3 ... */

      gan_image_bit_and_q ( pImage1, pImage2, pImage3 ); /* AND(1,2) into image 3, OR */
      pImage4 = gan_image_bit_and_s ( pImage1, pImage2 ); /* AND(1,2) as a new image, OR */
      gan_image_bit_and_i ( pImage1, pImage2 ); /* replace image 1 with AND(1,2) */
\end{verbatim}
The other operations follow similar lines. Firstly the OR operation:
\begin{verbatim}
      Gan_Image *pImage1, *pImage2, *pImage3, *pImage4;

      /* ... create and fill images 1 & 2 as boolean images, create image 3 ... */

      gan_image_bit_or_q ( pImage1, pImage2, pImage3 ); /* OR(1,2) into image 3, OR */
      pImage4 = gan_image_bit_or_s ( pImage1, pImage2 ); /* OR(1,2) as a new image, OR */
      gan_image_bit_or_i ( pImage1, pImage2 ); /* replace image 1 with OR(1,2) */
\end{verbatim}
Now the exclusive-OR operation:
\begin{verbatim}
      Gan_Image *pImage1, *pImage2, *pImage3, *pImage4;

      /* ... create and fill images 1 & 2 as boolean images, create image 3 ... */

      gan_image_bit_eor_q ( pImage1, pImage2, pImage3 ); /* EOR(1,2) into image 3, OR */
      pImage4 = gan_image_bit_eor_s ( pImage1, pImage2 ); /* EOR(1,2) as a new image, OR */
      gan_image_bit_eor_i ( pImage1, pImage2 ); /* replace image 1 with EOR(1,2) */
\end{verbatim}
Finally the not-AND operation:
\begin{verbatim}
      Gan_Image *pImage1, *pImage2, *pImage3, *pImage4;

      /* ... create and fill images 1 & 2 as boolean images, create image 3 ... */

      gan_image_bit_nand_q ( pImage1, pImage2, pImage3 ); /* NAND(1,2) into image 3, OR */
      pImage4 = gan_image_bit_nand_s ( pImage1, pImage2 ); /* NAND(1,2) as a new image, OR */
      gan_image_bit_nand_i ( pImage1, pImage2 ); /* replace image 1 with NAND(1,2) */
\end{verbatim}

A few more miscellaneous routines are available for binary images.
To fill part of a row with either zero or one use the routine
\begin{verbatim}
      /* fill section of row 13 of image with ones (true) starting at
         horizontal position 20 and filling 30 pixels */
      gan_image_bit_fill_row ( pImage, 13, 20, 30, GAN_TRUE );
\end{verbatim}
To invert part of a row of a binary image use
\begin{verbatim}
      /* invert section of row 13 starting at horizontal position 20 and
         filling 30 pixels */
      gan_image_bit_invert_row ( pImage, 13, 20, 30 );
\end{verbatim}
Finally if you want to clear a binary image to zero except inside a
specified rectangular region of the image, try this:
\begin{verbatim}
      /* clear binary image to zero except in 50(h)x30(w) pixel area starting
         at position 20,60 (y,x) */
      gan_image_mask_window_b ( pImage, 20, 60, 50, 30 );
\end{verbatim}

{\bf Error detection:} The standard binary image routines detect errors
as described in Section~\ref{simple-image-sec}. The boolean operation
routines ({\tt gan\_image\_bit\_invert\_q()} etc.) return a pointer to
the result image, and return {\tt NULL} on an error. All the other binary
image routines, with one exception, return a boolean value;
thus {\tt GAN\_FALSE} is returned on error. The exception is
{\tt gan\_image\_get\_pixel\_count\_b()}, which returns an integer
value, which in case of error is returned as -1.
The Gandalf error handler is invoked in all these cases.

\section{Pointer images} \label{pointer-image-sec}
\begin{verbatim}
      #include <gandalf/image/image_pointer.h>
\end{verbatim}
Gandalf pointer images allow storage and manipulation of a 2D array of
generic pointers, stored as {\tt void *} values. Pointer images have
format {\tt GAN\_GREY\_LEVEL\_IMAGE} and type {\tt GAN\_POINTER}.
All the standard functions given above. Note that when a pointer image
is freed, the pointer pixels are left ``hanging'', so they should
if necessary be freed first before freeing the pointer image.
This code fragment illustrates the use of pointer image functions.
\begin{verbatim}
      Gan_Image *pImage;
      Gan_Vector4 *apv4Vector[5], *pv4Vector;
      int iCount, iRow, iCol;

      /* allocate 300x200 pointer image, and initialise all pointer "pixels"
         to NULL */
      pImage = gan_image_alloc_p ( 300, 200 );
      gan_image_fill_zero(pImage);

      /* allocate some pointers to 4-vectors */
      for ( iCount = 5-1; iCount >= 0; iCount-- )
         apv4Vector[iCount] = gan_malloc_object(Gan_Vector4);

      /* set some pointer "pixels" */
      gan_image_set_pix_p ( pImage, 271,  39, apv4Vector[0] );
      gan_image_set_pix_p ( pImage,  30, 120, apv4Vector[1] );
      gan_image_set_pix_p ( pImage,  78,  49, apv4Vector[2] );
      gan_image_set_pix_p ( pImage, 147, 120, apv4Vector[3] );
      gan_image_set_pix_p ( pImage, 232, 130, apv4Vector[4] );

      /* now free allocated vectors by searching for non-NULL values in the
         image */
      for ( iRow = (int)pImage->height-1; iRow >= 0; iRow-- )
         for ( iCol = (int)pImage->width-1; iCol >= 0; iCol-- )
            if ( (pv4Vector = gan_image_get_pix_p ( pImage, iRow, iCol )) != NULL )
               free(pv4Vector);

      /* free image */
      gan_image_free ( pImage );
\end{verbatim}

\section{Copying/converting the whole or part of an image}
\begin{verbatim}
      #include <gandalf/image/image_defs.h>
\end{verbatim}
To copy a whole image of any format or type, use one of the following
routines:
\begin{verbatim}
      Gan_Image *pImage1, *pImage2, *pImage3; /* declare images 1, 2 & 3 */

      /* ... create images 1 & 2, fill image 1 ... */
      gan_image_copy_q ( pImage1, pImage2 ); /* copy image 1 to image 2, OR */
      pImage3 = gan_image_copy_s ( pImage1 ); /* copy image 1 as new image */
\end{verbatim}
Image 2 here may have been created with any format, type or dimensions.
Gandalf will reset the attributes of image 2 to those of image 1 before
copying the image data. These routines make copies of the image data,
so image 1 may be destroyed after it is copied.

To copy parts of an image, you will need to include the header file
\begin{verbatim}
      #include <gandalf/image/image_extract.h>
\end{verbatim}
The routines to extract sub-parts of an image are
{\tt gan\_image\_extract\_q()} and {\tt gan\_image\_extract\_s()}.
They have the following extra features over a simple routine to
copy image sub-regions:
\begin{enumerate}
  \item You can convert the image sub-region to any desired format and type,
        avoiding the need to perform the two steps of extracting and
        converting sequentially, and thus saving computation and memory.
  \item There is an option to make the resulting sub-image point into the
        source image, rather than copy the pixel data from it.
        This saves computation time, and the sub-image produced can be
        manipulated in the same way as other Gandalf images.
        Obviously use of this feature precludes use of feature 1.
\end{enumerate}
Here are a couple of examples using the sub-region extraction routines.
Firstly a code fragment showing showing the simplest form, where the
region is copied from the source image and the format/type remain the same.
\begin{verbatim}
      Gan_Image *pImage1, *pImage2; /* declare images 1 & 2 */
      Gan_RGBPixel_uc rgbucPixel;

      /* create RGB unsigned character image 1 and fill with constant */
      pImage1 = gan_image_alloc_rgb_uc ( 200, 100 );
      rgbucPixel.R = 128; rgbucPixel.R = 80; rgbucPixel.R = 200;
      gan_image_fill_const_rgb_uc ( pImage1, &rgbucPixel );

      /* create image 2 in an arbitrary way */
      pImage2 = gan_image_alloc_gl_uc(0,0);

      /* extract sub-region in image 1 into image 2, with height 60, width 50,
         starting as position 30,40 (y,x), leaving the format/type the same.
         The pixel data is copied */
      gan_image_extract_q ( pImage1, 30, 40, 60, 50,
                            pImage1->format, pImage1->type, GAN_TRUE,
                            pImage2 );
\end{verbatim}
Now an example continuing from the above, and showing how to make the result
image point into the source image.
\begin{verbatim}
      /* extract sub-region in image 1 into image 2, with height 60, width 50,
         starting as position 30,40 (y,x), leaving the format/type the same.
         Here the pixel data is not copied; instead the result image points
         into the source image */
      gan_image_extract_q ( pImage1, 30, 40, 60, 50,
                            pImage1->format, pImage1->type, GAN_FALSE,
                            pImage2 );
\end{verbatim}
Finally an example showing how to convert the sub-region to a different
format and type.
\begin{verbatim}
      {
         Gan_Image *pImage3; /* declare image 3 */

         /* extract sub-region in image 1 into image 2, with height 60, width 50,
            starting as position 30,40 (y,x), converting the format to grey-level
            and the type to unsigned short. Here the format and type are modified
*           as the pixels are extracted from the source image */
         pImage3 = gan_image_extract_s ( pImage1, 30, 40, 60, 50,
                                         GAN_GREY_LEVEL_IMAGE, GAN_USHORT,
                                         GAN_TRUE );
      }
\end{verbatim}

There are also routines to convert the whole of an image to a different
format or type (or both). These are simpler macro versions of
{\tt gan\_image\_extract\_[qs]()}, and can be illustrated as follows:
\begin{verbatim}
      Gan_Image *pImage1, *pImage2, *pImage3; /* declare images 1, 2 & 3 */

      /* ... create RGB unsigned character image 1 and fill with constant,
             and create image 2 in an arbitrary way ... */

      /* convert image 1 to grey-level format and unsigned short type */
      gan_image_convert_q ( pImage1, GAN_GREY_LEVEL_IMAGE, GAN_USHORT,
                            pImage2 ); /* convert image 1 to image 2, OR */
      pImage3 = gan_image_convert_s ( pImage1, GAN_GREY_LEVEL_IMAGE, GAN_USHORT );
\end{verbatim}

{\bf Error detection:} These routines return the result image pointer,
and return {\tt NULL} on error.

\subsection{Accessing channels of an image}
\begin{verbatim}
      #include <gandalf/image/image_channel.h>
\end{verbatim}
Gandalf stores images with the channels combined for each pixel.
If you wish to extract a channel of an image as a separate image, Gandalf
provides the following function:
\begin{verbatim}
      Gan_Image *pRGBImage; /* declare RGB image */
      Gan_Image *pRedChannel; /* declare image storing red channel */

      /* ... create and fill RGB image, create red channel image ... */

      /* extract red channel from image */
      gan_image_extract_channel_q ( pRGBImage, GAN_RED_CHANNEL,
                                    0, 0, pRGBImage->height, pRGBImage->width,
                                    pRedChannel );
\end{verbatim}
The second argument specifies which channel is to be extracted.
The different options are described by the following enumerated type.
\begin{verbatim}
      /**
       * \brief Image channel types for extracting individual channels.
       */
      typedef enum
      {
         /// for grey-level/alpha images
         GAN_INTENSITY_CHANNEL,

         ///for RGB and RGB/alpha images
         GAN_RED_CHANNEL, GAN_GREEN_CHANNEL, GAN_BLUE_CHANNEL,

         /// for grey-level/alpha and RGB/alpha images
         GAN_ALPHA_CHANNEL,

         /// for 2D and 3D vector field images
         GAN_X_CHANNEL,

         /// likewise
         GAN_Y_CHANNEL,

         /// for 3D vector field images
         GAN_Z_CHANNEL,

         /// all channels
         GAN_ALL_CHANNELS
      } Gan_ImageChannelType;
\end{verbatim}
The offset (3,4) and dimension (5,6) arguments allow a sub-region to
be extracted rather than the whole image, and work in the same way
as with {\tt gan\_image\_extract\_q()}. There is also a version which
extracts the channel as a new image:
\begin{verbatim}
      pRedChannel = gan_image_extract_channel_s ( pRGBImage, GAN_RED_CHANNEL,
                                                  0, 0,
                                                  pRGBImage->height, pRGBImage->width );
\end{verbatim}

There are also functions for filling a channel of an RGB image with
a constant value. For instance
\begin{verbatim}
      Gan_Pixel Pixel;

      /* ... fill pRGBImage as an RGB unsigned character image ...*/

      /* fill green channel of pRGBImage with constant value */
      Pixel.format = GAN_GREY_LEVEL_IMAGE;
      Pixel.type = GAN_UCHAR;
      Pixel.data.gl.uc = 128;
      gan_image_fill_channel_const ( pRGBImage, GAN_GREEN_CHANNEL, &Pixel );
\end{verbatim}
sets the all the green pixel components to the value 128. Note that
the format of the pixel is set to grey-level, so defining a single channel
pixel. To set the channel to zero there is the macro
\begin{verbatim}
      gan_image_fill_channel_zero ( pRGBImage, GAN_GREEN_CHANNEL );
\end{verbatim}
instead.

{\bf Error detection:} The {\tt gan\_image\_extract\_channel\_[qs]()}
return a pointer to the result image, returning {\tt NULL} and invoking
the Gandalf error handler on error. {\tt gan\_image\_fill\_channel\_const()}
and\\ {\tt gan\_image\_fill\_channel\_zero()} return a boolean value, so
{\tt GAN\_FALSE} is returned on error.

\section{Displaying images}
\begin{verbatim}
      #include <gandalf/image/image_display.h>
\end{verbatim}
Gandalf uses OpenGL to display images. It assumes that as well as the
standard OpenGL libraries, the GL user toolkit (GLUT) is also installed.
Because GLUT is event-driven, program control needs to be passed to the
GLUT event handler by calling {\tt glutMainLoop()} after creating the
initial windows you want. Remember that creating a window using the
{\tt gan\_display\_new\_window()} function (see below) will not make it
appear immediately. The window creation event needs to be processed by GLUT.
This needs to be borne in mind when reading the description of the functions
below. The simplest example using the functions is in
{\tt gandalf/image/bitmap\_test.c}.

Once an OpenGL window has been set up and a Gandalf image {\tt pImage}
created, calls to
\begin{verbatim}
      glRasterPos2i ( 0, 0 );
      gan_image_display ( pImage );
\end{verbatim}
will display the image using the OpenGL function {\tt glDrawPixels}.
This involves a fair amount of OpenGL calls to set the display windows up.
To simply create an OpenGL window and display a Gandalf image in it,
Gandalf provides functions to make this easy for you. You can use the code
\begin{verbatim}
      Gan_Image *pImage;
      int iWindowID;

      /* create OpenGL window to display image/graphics with coordinates
         in the range (0-200) vertically and (0-300) horizontally, using a
         zoom factor of 2 so that the size on the screen will be 400x600.
         The window is placed at offset 100,100 from the corner of the screen.
       */
      gan_display_new_window ( 200, 300, 2.0, "Graphics", 100, 100,
                               &iWindowID );

      /* ... create and fill image pImage ... */

      /* display image with top-left pixel at position (0,0) */
      gan_image_display ( pImage );
\end{verbatim}
The image is drawn so that if its dimensions match those passed as the first
two arguments to\\ {\tt gan\_display\_new\_window()}, the displayed image will
completely fill the window. If you want the image displayed at a position
offset from the top-left corner of the window you will need
an appropriate call to {\tt glRasterPos2i()}, such as
\begin{verbatim}
      /* set position in OpenGL window as top-left position in image drawn
       * subsequently by gan_image_display() */
      glRasterPos2i ( 30, 40 );
\end{verbatim}
The window is available for the standard
OpenGL graphics functions. The name of the window ("Graphics" in the above
example) is shown in the bar at the top of the graphics window.
If the graphics window changes, the window
identifier {\tt iWindowID} can be used to switch back to the created
window using
\begin{verbatim}
      glutSetWindow ( iWindowID );
\end{verbatim}
To create several identically sized graphics windows, use this routine:
\begin{verbatim}
      int iWindowID, *aiWindowID;

      /* create an OpenGL window containing 2 rows and 3 columns of
         sub-windows, each containing a 300x200 graphics window, each of
         which contains a (0-900) by (0-600) coordinate frame shrunk to
         300x200 using a zoom factor of 1/3 */
      gan_display_new_window_array ( 2, 3, 900, 600, 1.0/3.0, "Graphics",
                                     100, 100, &iWindowID, &aiWindowID );
\end{verbatim}
The sub-windows are created using the function {\tt glutCreateSubWindow()}.
In this case there is both a window identifier for the main display window
and an array of window identifiers for the sub-windows, stored in the
array in raster-scan order.

Gandalf also provides a routine that creates a window and displays the
image all in one, determining the graphics window size from a zoom factor
passed in:
\begin{verbatim}
      /* create display window and display image zoomed to double its size */
      gan_image_display_new_window ( pImage, 2.0, "Graphics", 100, 100, &iWindowID );
\end{verbatim}
This is useful for debugging purposes as the easiest way to display an image.

The {\tt gan\_display\_new\_window()} function stores the windows
created in a list, so that the images displayed in the windows can be
automatically refreshed. When you have finished with the graphics windows
created by {\tt gan\_display\_new\_window()}, remove them using
the function
\begin{verbatim}
      gan_image_display_free_windows();
\end{verbatim}
Note that this free function only applies to windows created by
{\tt gan\_display\_new\_window()}. Graphics windows created using
the other functions in this section are refreshed using standard
OpenGL routines ({\tt glutDisplayFunc()} etc.).

{\bf Error detection:} All the routines except
{\tt gan\_image\_display\_free\_windows()} return a boolean result,
which is {\tt GAN\_FALSE} if an error occurs, invoking the Gandalf
error handler.

\section{Image pyramids}
\begin{verbatim}
      #include <gandalf/image/image_pyramid.h>
\end{verbatim}
A quite common construction in image processing is a ``pyramid'' of
images, a multi-resolution representation of an image. We think in fact
of an {\em inverted} pyramid, with the top level of the pyramid representing
the image at the original resolution, and lower levels representing the
image at lower resolutions (the inversion of the pyramid is to avoid
changing the sense of ``high'' and ``low'' between describing the
``resolution'' and the ``level''). The pyramid is constructed as an
array of structures, one for each resolution level. The structure is
defined as
\begin{verbatim}
      /* structure to hold image and mask at a single pyramid level */
      typedef struct Gan_ImagePyramid
      {
         Gan_Image *img; /* image represented at a single resolution level */
         Gan_Image *mask; /* mask at this level defining which pixels are set */
      } Gan_ImagePyramid;
\end{verbatim}
Intrinsic to the pyramid is the notion of a ``mask'' of pixels, a binary
image defining which pixels are available in the image. The convention is
that available pixels are marked in the mask as ones. Gandalf currently
supports pyramids produced in the simplest way, by averaging four adjacent
pixels to convert the image from a higher resolution image to a lower
resolution image. An example code fragment to create an image pyramid is
\begin{verbatim}
      Gan_Image *pImage, *pMask; /* declare image and mask */
      Gan_ImagePyramid *aPyramid;

      /* ... create and fill pImage and pMask ... */

      /* build image with four resolution levels */
      gan_image_pyramid_build ( pImage, pMask, 4, &aPyramid );
\end{verbatim}
The image and mask pointer at the original (highest) resolution levels can
be accessed as {\tt aPyramid[0].img} and {\tt aPyramid[0].mask} respectively.
There are four levels here so the lowest resolution image and mask are
{\tt aPyramid[3].img} and {\tt aPyramid[3].mask}. The mask can be passed
into {\tt gan\_image\_pyramid\_build()} as {\tt NULL}, indicating that
all the pixels in the image are available. In this case the masks at all
resolution levels will be set to {\tt NULL}.

To free the pyramid, use
\begin{verbatim}
      gan_image_pyramid_free ( aPyramid, 4, GAN_FALSE );
\end{verbatim}
The second argument here is the number of resolution levels of the pyramid.
The last argument determines whether the image/mask at the top level of
the pyramid are to freed. Here {\tt GAN\_FALSE} is passed, so the original
image {\tt pImage} and mask {\tt pMask} will not be freed and are available
for further processing.

Note that when transferring pixels to a lower resolution, a pixel is
computed and a mask bit at the lower resolution only if all four corresponding
pixels in the higher resolution image are set.

{\bf Error detection:} {\tt gan\_image\_pyramid\_build()} returns a boolean
value, which is {\tt GAN\_FALSE} on error, the Gandalf error handler being
invoked.

\section{Inverting an image}
\begin{verbatim}
      #include <gandalf/image_invert.h>
\end{verbatim}
Image inversion in Gandalf use {\tt gan\_image\_invert\_[qsi]}.
The simplest call is
\begin{verbatim}
      Gan_Image *pImage, *pInvImage; /* declare image and inverted image */

      /* ... create and fill pImage ... */

      /* invert image the simple way */
      pInvImage = gan_image_invert_s ( pImage );
\end{verbatim}
or else invert using a pre-allocated result image:
\begin{verbatim}
      /* ... create and fill pImage, allocate pInvImage ... */

      /* invert image the simple way */
      gan_image_invert_q ( pImage, pInvImage );
\end{verbatim}
and finally the in-place version:
\begin{verbatim}
      /* ... create and fill pImage ... */

      /* invert image in-place */
      gan_image_invert_i ( pImage );
\end{verbatim}

\section{Image sequence I/O}
\begin{verbatim}
      #include <gandalf/image/io/movie.h>
\end{verbatim}
Gandalf has a module for reading and writing image sequences.
These are accessed one image at a time. The {\tt Gan\_MovieStruct}
structure defines an image sequence.
A movie structure is created using the {\tt gan\_movie\_new()} function.
An example call is
\begin{verbatim}
      Gan_MovieStruct *pMovie;

      pMovie = gan_movie_new ( "/tmp", "movie.", 3, ".png", 1, 20,
                               GAN_PNG_FORMAT );
\end{verbatim}
The arguments to the function define the image sequence attributes,
in the following order:
\begin{enumerate}
  \item The directory in which to find the images;
  \item the base name of the image file names;
  \item the number of digits in the number part of the image file name;
  \item the suffix of each image, usually related to the image file format;
  \item the number of the first image in the sequence;
  \item the image file format;
\end{enumerate}
The above example defines a sequence of PNG format image files
\begin{verbatim}
      /tmp/movie.001.png
      /tmp/movie.002.png
              .
              .
              .
      /tmp/movie.020.png
\end{verbatim}

Other parameters of a movie structure have defaults which can be set using
functions before the movie images are accessed. These functions are
\begin{verbatim}
      gan_movie_set_step ( pMovie, 2 );
\end{verbatim}
to set the step in numbers between images. The default is one, and the above
call would set the frame numbers to 1, 3, 5 etc.
\begin{verbatim}
      gan_movie_set_crop_window ( pMovie, 5, 10, 8, 12 );
\end{verbatim}
sets the values of any crop parameters, i.e. the widths of areas at the edge
of each image which should be ignored by image processing operations.
The widths are give for the left, right, top and bottom edges respectively.

The movie structure is used both for reading and writing images in a
sequence. The number of digits indicates the amount of zero-padding of
the file number. A value of zero indicates that no padding is done.
To read a single image from the sequence, use the function
{\tt gan\_movie\_image\_read()}, for example
\begin{verbatim}
      Gan_Image *pImage;

      pImage = gan_movie_image_read ( pMovie, 8, NULL );
\end{verbatim}
The second argument indicates which image in the sequence is to be read,
from 0 to 19 in this case. The value 8 indicates the file
{\tt /tmp/movie.009.png}. This reads the image file into a new image.
If {\tt pImage} is already created, you can use
\begin{verbatim}
      gan_movie_image_read ( pMovie, 8, pImage );
\end{verbatim}

To write an image in a sequence use
\begin{verbatim}
      gan_movie_image_write ( pMovie, 10, pImage );
\end{verbatim}
This will write the file {\tt /tmp/movie.011.png}.

Sometimes it is desirable to build the full name of a movie image file,
for instance when generating error messages to say that a given file
cannot be read or written. To write an image file name into a string,
use the function
\begin{verbatim}
      char acString[300];

      gan_movie_image_name ( pMovie, 0, acString, 300 );
\end{verbatim}
This writes the name of the first image of the sequence into the provided
string, up to the 300 character total size of the {\tt acString} array.
For the movie created in the above example this will fill the string
{\tt acString} with the value {\tt "/tmp/movie.001.png"}. The second
argument is the number of the image in the sequence, so passing 5 would
give the string {\tt "/tmp/movie.006.png"}.

Finally, to free a movie structure use
\begin{verbatim}
      gan_movie_free ( pMovie );
\end{verbatim}

{\tt Error detection:} {\tt gan\_movie\_new()} returns a pointer to the
allocated movie structure, and {\tt NULL} is returned in case of error.
{\tt gan\_movie\_image\_read()} and {\tt gan\_movie\_image\_name()} also return
{\tt NULL} on error.\\ {\tt gan\_movie\_image\_write()} returns a boolean
value, so {\tt GAN\_FALSE} is returned on error. In all cases the Gandalf
error handler is invoked.


