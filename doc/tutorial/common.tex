\chapter{The Common Package} \label{common-chapter}
The {\bf common} package defines general purpose types and routines used
extensively elsewhere in Gandalf, and also available to other application
code using Gandalf. To use any function or structure in the common package
use the declaration
\begin{verbatim}
      #include <gandalf/common.h>
\end{verbatim}
but including individual module header files instead will speed up program
compilation.
We describe each module in the common package below.

\section{Miscellaneous definitions} \label{common-misc-sec}
\begin{verbatim}
      #include <gandalf/common/misc_defs.h>
\end{verbatim}
The module {\tt misc\_defs.[ch]} defines basic types used in Gandalf.

\subsection{Simple types} \label{simple-types-sec}
A boolean type {\tt GAN\_BOOL} is defined:
\begin{verbatim}
      typedef enum { GAN_FALSE=0, GAN_TRUE=1 } Gan_Bool;
\end{verbatim}
The boolean type is the standard type returned by a Gandalf function,
where a return type of {\tt GAN\_TRUE} indicates success, {\tt GAN\_FALSE}
failure. Another use for the {\tt Gan\_Bool} type is with bit arrays and
binary images. See Sections~\ref{bit-array-sec} and~\ref{binary-image-sec}.

The {\tt Gan\_Type} enumerated type is used extensively to indicated
different kinds of simple objects:
\begin{verbatim}
      /// labels for simple types used throughout Gandalf
      typedef enum
      {
         GAN_CHAR,       /**< signed character */
         GAN_UCHAR,      /**< unsigned character */
         GAN_SHORT,      /**< signed short integer */
         GAN_USHORT,     /**< unsigned short integer */
         GAN_INT,        /**< signed integer */
         GAN_UINT,       /**< unsigned integer */
         GAN_LONG,       /**< signed long integer */
         GAN_ULONG,      /**< unsigned long integer */
      #if (SIZEOF_LONG_LONG != 0)
         GAN_LONGLONG,   /**< signed extra-long integer */
      #endif
         GAN_FLOAT,      /**< single precision floating point */
         GAN_DOUBLE,     /**< double precision floating point */
         GAN_LONGDOUBLE, /**< long double precision floating point */
         GAN_STRING,     /**< string (array of characters) */
         GAN_BOOL,       /**< boolean */
         GAN_POINTER     /**< generic pointer */
      } Gan_Type;
\end{verbatim}
Note that the {\tt GAN\_LONGLONG} value is only defined if the {\tt configure}
program finds the {\tt long~long} C type, and is able to determine its size.
The array {\tt gan\_type\_sizes[]} holds the sizes of each {\tt Gan\_Type}
value:
\begin{verbatim}
      /// array of sizeof()'s of each Gandalf type, one for each value in a Gan_Type
      extern const size_t gan_type_sizes[];
\end{verbatim}
{\tt gan\_type\_sizes} and the {\tt gan\_debug} boolean flag
(see below) are the only global variables in Gandalf.\\{\tt gan\_type\_sizes[]}
is a constant array, so it is thread-safe.

Gandalf also provides single and double precision floating point versions
of the integer limit values found in {\tt <limits.h>}. For instance
{\tt GAN\_INT\_MAXF} and {\tt GAN\_INT\_MAXD} are the {\tt float} and
{\tt double} versions of {\tt INT\_MAX}.

\subsection{Types with specific bit sizes}
Gandalf builds unsigned integer types with specific bit sizes, so that
the types {\tt GAN\_UINT8}, {\tt GAN\_UINT16}, {\tt GAN\_UINT32} and
{\tt GAN\_UINT64} are {\tt \#define}'d to the relevant {\tt Gan\_Type}
value, using the information on the sizes of architecture-dependent
C object sizes provided by {\tt configure}. You can also define your
own variables with specific sizes using the {\tt typedef}s {\tt gan\_ui8}
etc. For instance a declaration
\begin{verbatim}
      gan_ui16 val;
\end{verbatim}
declares a 16-bit variable. {\tt gan\_ui32} and {\tt gan\_ui64} are 
also defined, the latter only on 64-bit architectures.

\subsection{Debugging tools} \label{debugging-tools-sec}
The macro {\tt gan\_assert()} provides a mechanism similar to {\tt assert()}
except that it allows a user-defined message to be printed when the test
fails. When {\tt NDEBUG} is {\tt \#define}'d {\tt gan\_assert()} will have
no effect. The global variable {\tt gan\_debug} determines whether to
print certain debugging messages. It is only available when {\tt NDEBUG}
is not defined.

\section{Linked lists} \label{linked-list-sec}
\begin{verbatim}
      #include <gandalf/common/linked_list.h>
\end{verbatim}
Gandalf linked lists are stored in {\tt Gan\_List} structures.
The underlying structure is a doubly-linked list, so Gandalf lists can be
traversed both forwards and backwards.
A new empty list can be created using
\begin{verbatim}
      Gan_List List;

      gan_list_form(&List);
\end{verbatim}
Note that to detect errors the return value of {\tt gan\_list\_form()}
should be compared with {\tt NULL}, invoking the Gandalf error package
(see Sections~\ref{error-sec} and~\ref{error-handling-sec})
and returning an error condition
if {\tt NULL} is returned. Here and elsewhere we omit these tests for
the sake of clarity, but many examples of this testing can be found in
the Gandalf source.

To insert a new list node at the start of the list, use
\begin{verbatim}
      gan_list_insert_first ( &List, ptr );
\end{verbatim}
where {\tt ptr} is the data item (pointer) to be stored. By repetitively
called this function, a list can be built up, with the last item added as
the first node in the list.
A Gandalf list stores pointers in an ordered way, while still transparently
allowing pseudo-random access to the list nodes. As well as the stored
data, the list maintains a state variable indicating a position within the
list, from 0 to $N-1$ for a list of $N$ nodes. The normal way to traverse
a list is to use the following sequence:
\begin{verbatim}
      int iCount;
      Gan_Matrix *pMatrix;

      gan_list_goto_head ( &List );
      for ( iCount = gan_list_get_size(&List)-1; iCount >= 0; iCount-- )
      {
         pMatrix = gan_list_get_next ( &List, Gan_Matrix );
         ... [ do something with pMatrix ] ...
      }
\end{verbatim}
{\tt gan\_list\_goto\_head()} sets the position state variable to -1,
i.e. the position just before the start of the list.
{\tt gan\_list\_get\_size()} returns the number of nodes in a list.
So the above loop runs through each node of the list, calling
{\tt gan\_list\_get\_next()} to provide each data item in turn, in this
case matrix structure pointers. {\tt gan\_list\_get\_next()} increments
the position state variable by one each time it is called, so on the
first call in the above loop, the position is increment from -1 to 0,
and the node at position 0 returned.

To free the list use
\begin{verbatim}
      gan_list_free ( &List, NULL );
\end{verbatim}
which frees the list nodes but not the data they point to. If you wanted to
free the data as well, for the above list you would use
\begin{verbatim}
      gan_list_free ( &List, (Gan_FreeFunc) gan_mat_free );
\end{verbatim}
which calls {\tt gan\_mat\_free()} on each matrix in the list.

Note that pointers to lists may be used instead of directly using the
structures. To create a list using pointers use
\begin{verbatim}
      Gan_List *pList;

      pList = gan_list_new();
\end{verbatim}
and at the end call
\begin{verbatim}
      gan_list_free ( pList, NULL );
\end{verbatim}
In this and other examples Gandalf keeps track of which parts of a structure
were dynamically allocated, and only frees those which were.

Other ways to build and access linked lists are provided in the reference
documentation.

\section{Bit arrays} \label{bit-array-sec}
\begin{verbatim}
      #include <gandalf/common/bit_array.h>
\end{verbatim}
Gandalf bit arrays are a compact representation of binary flags.
They are used both directly and as the foundation of binary images in Gandalf
(see Section~\ref{binary-image-sec}).
They allow compact storage of an array of boolean values.
The architecture of the computer determines how the boolean values are stored,
so for instance on 32 bit machines, the boolean values are packed as 32 values
in a single word.
To create a bit array use the following code:
\begin{verbatim}
      Gan_BitArray BitArray;

      gan_bit_array_form ( &BitArray, 200 );
\end{verbatim}
for an array of 200 bits.

A bit array may be initialised to zero by calling
\begin{verbatim}
      gan_bit_array_fill ( &BitArray, GAN_FALSE );
\end{verbatim}
or to one (all bits set) by passing {\tt GAN\_TRUE} instead of
{\tt GAN\_FALSE}. To set a bit to one, use
\begin{verbatim}
      gan_bit_array_set_bit ( &BitArray, pos );
\end{verbatim}
where {\tt pos} is the bit you want to set, from zero to $N-1$ for a bit
array of $N$ bits. Similarly use\\ {\tt gan\_bit\_array\_clear\_bit()} to
clear a bit to zero. To return the value of a particular bit use
\begin{verbatim}
      Gan_Bool bBit;

      bBit = gan_bit_array_get_bit ( &BitArray, pos );
\end{verbatim}

Several boolean operations are supported on bit arrays. Given two bit arrays,
the operation
\begin{verbatim}
      gan_bit_array_and_i ( &BitArray1, &BitArray2 );
\end{verbatim}
performs the bitwise AND operation on each bit of the bit arrays, overwriting
{\tt BirArray1} with the result. Bitwise OR, exclusive-OR (EOR) and not-AND
(NAND) are also supported, as well as inversion (NOT).

To free a bit array after you have finished using it, call
\begin{verbatim}
      gan_bit_array_free ( &BitArray );
\end{verbatim}

\section{Memory allocation} \label{allocation-sec}
\begin{verbatim}
      #include <gandalf/common/allocate.h>
\end{verbatim}
Gandalf provides some macros to simplify access to the standard {\tt malloc()}
and {\tt realloc()} functions. So for instance to allocate an array of
a hundred integers you can use
\begin{verbatim}
      int *aiArray;

      aiArray = gan_malloc_array ( int, 100 );
\end{verbatim}
instead of the usual
\begin{verbatim}
      int *aiArray;

      aiArray = (int *) malloc ( 100*sizeof(int) );
\end{verbatim}

\section{Array fill/copy} \label{array-sec}
\begin{verbatim}
      #include <gandalf/common/array.h>
\end{verbatim}
There are two sets of functions in this module, one set dealing with
filling an array of numbers or pointers with a constant value, the other
dealing with copying an array into another array. To fill an array of
{\tt float}s with the value five, for instance, use
\begin{verbatim}
      float afArray[100];

      gan_fill_array_f ( afArray, 100, 1, 5.0F );
\end{verbatim}
The third {\tt stride} argument will be one for filling a simple contiguous
array. A different value indicates the number of elements of the array to
skip when filling each value. A value of three, for instance, indicates that
only every third element of the array is to be filled.

To copy one array into another, each having arbitrary stride, use
\begin{verbatim}
      float afArray1[100], float afArray2[100];

      /* fill array afArray1 with values */
      ...

      gan_copy_array_f ( afArray1, 1, 100, afArray2, 1 );
\end{verbatim}
This copies array {\tt afArray1} into {\tt afArray2}.

\section{Complex numbers} \label{complex-sec}
\begin{verbatim}
      #include <gandalf/common/complex.h>
\end{verbatim}
Complex numbers are defined here, used 

\section{Numerical routines} \label{numerics-sec}
\begin{verbatim}
      #include <gandalf/common/numerics.h>
\end{verbatim}
This contains a few useful numerical routines not often included in
C libraries, such as square (e.g. {\tt gan\_sqr\_d()} or the macro
{\tt gan\_sqr()}),
cube-root {\tt gan\_cbrt()} and various flavours of random number
generators. There are functions {\tt gan\_solve\_quadratic ()} and
{\tt gan\_solve\_cubic()} for finding the real and complex roots
of quadratic and cubic equations,
and also a function {\tt gan\_normal\_sample()} for
generating samples from a normal distribution.


\section{Comparison routines} \label{compare-sec}
\begin{verbatim}
      #include <gandalf/common/compare.h>
\end{verbatim}
This is a set of functions to compute the maximum or minimum of up to
six numbers, either as macros (e.g. {\tt gan\_min3()} for the minumum of
three numbers) or as functions such as {\tt gan\_max4\_d()} to return the
maximum of four double precision floating point numbers.

\section{Error handling} \label{error-handling-sec}
\begin{verbatim}
      #include <gandalf/common/gan_err.h>
\end{verbatim}
The purpose of the error module is to provide a
mechanism by which generic reusable code (typically a software library) can
report errors to a variety of applications without the need to
modify the library code for each new application context. That is,
the error reporting mechanism of the library is highly decoupled
from that of the application. Communication of error information
from library to application is performed using a small and well
defined interface.

The role of the library is to communicate full and
unprocessed error information to the application. The role of the
application is to access the error information and act on it, whether
reporting the error back to the user, ignore it or perform some other action.
This demarcation of roles allows the application to use its
own error reporting mechanism, without any need to embed application
specific code in the library. The library achieves generality
because it plays no role in reporting the error information, which
usually requires system and application specific facilities.
Specifically, the library writes (registers) error information into
a LIFO stack (error trace) which is built up as the error unwinds
through the nested calls. When the library function called by the
application finally returns, with an error code, the application
uses an error reporter to access the errors details and processes
that information in any way it chooses (e.g. displays an error
dialogue box, logs the error in a database).

The library function at which a new error occurs must first flush
the error trace before registering the error.

When using the error handling code the following definitions are useful.
\begin{description}
  \item{{\bf Error record:}} a struct holding error code, file name, line
	number, and text message for one error.
  \item{{\bf Error trace:}} a LIFO stack of error records, which allows
	temporary storage of error information until defered retrieval by
	application.
  \item{{\bf Top record:}} The most recent error stored in trace.
  \item{{\bf Error detection:}} Code that detects occurrence of an error.
  \item{{\bf Error handling:}} Action undertaken as a result of detecting an
	error. In library this typically involves registering the error and
	returning from current function with an error status. In application
        this typically involves invoking the reporter function.
  \item{{\bf Registering:}} The process of placing an error into the trace
  \item{{\bf Flushing:}} The clearing of the error trace
  \item{{\bf Reporter:}} A function provided by the application to access error
        stored in trace and then communicating that information
        to the user or to a log. The reporter function must then
        flush the error trace.
\end{description}

To illustrate the use of the error handling package, consider an example
application which calls library (Gandalf or other) function A, which itself
calls library function B, which has an error that is detected.
Function B ``flushes'' the ``error trace'' (because it is
the last function called that uses the facilities of the error package),
and then ``registers'' the error details into the error trace and unwinds
to function A with a return value that indicates an error has occurred.
Function A tests the return value and detects the error, and then registers
an error into the trace and unwinds to the application with a return value
which indicates that an error has occurred. The application tests the return
value and detects that an error has occurred, so calls a facility in the
error package to report the error. The error report function in
turn calls an application-supplied error ``reporter'' function with
a pointer to the error trace as an argument. The address of the
error trace is stored as a module scoped variable in the error package
The reporter accesses the information contained in
the error trace using accessor functions, and communicates the
error details to the user or to a log in some application specific way
(or it may ignore the error or perform some other action).

Consequences and liabilities:
\begin{enumerate}
  \item The application is able to:
    \begin{itemize}
       \item control when errors are reported to the user interface 
             (the library should not itself report errors to the user)
       \item provide its own error reporting mechanism 
             (to suit its own user interface).
       \item extract sufficient information from the library to
             enable sufficient error reporting to be performed.
    \end{itemize}
 \item The library can be used with many applications, without modification.
 \item Interactive resolution of errors occurring in library is problematic.
      Essentially the library is a black box to the application.
\end{enumerate}

Usage notes for application writer: No
code is needed to initialise the error trace. But a error reporting
function is optionally installed in the error module using
{\tt gan\_err\_set\_reporter()}. The reporter is an application function of
type {\tt Gan\_ErrorReporterFunc}, which is defined in {\tt gan\_err.h}. The
reporter must get the error count using {\tt gan\_err\_get\_error\_count()}
and then sequentially access the errors stored in the trace using
{\tt gan\_err\_get\_error(n)}, where {\tt n} is the {\tt n}-th error,
and {\tt n=1} is the most
recent error. If no error reporter is installed, then the error
module provides a default reporter {\tt gan\_err\_default\_reporter()}, whose
action is to print the error details into {\tt stderr}. The function call
{\tt gan\_err\_set\_reporter(GAN\_ERR\_DFL)} causes the default error reporter
to be used, and the call {\tt gan\_err\_set\_reporter(GAN\_ERR\_IGN)} inhibits
the error reporter from being called. {\tt gan\_err\_set\_reporter()} returns
the address of the error reporter that was replaced so that it can
be reinstalled later.

When the application tests the return value of a library function
and detects that an error has occurred, it should call {\tt gan\_err\_report()}
which invokes the error reporter.

The application writer can choose not to buffer the error details in
a trace, but instead have the library function report errors
immediately, by automatically calling {\tt gan\_err\_report()} inside
{\tt gan\_err\_register()}. No error trace is built up. If the application
calls {\tt gan\_err\_report()}, no errors are reported because the trace will
be empty. Usage of the trace is controlled by {\tt gan\_err\_set\_trace()} with
arguments {\tt GAN\_ERR\_TRACE\_ON} or {\tt GAN\_ERR\_TRACE\_OFF}.

Usage notes for library writer: When a
error is detected at the deepest function call that uses the
facilities of the error module, then {\tt gan\_err\_flush\_trace()} should be
called, followed by {\tt gan\_err\_register()}. As the subsequent library
function unwinds, they should call {\tt gan\_err\_register()} (but not
{\tt gan\_err\_flush\_trace()}), and return with an error code. This
continues until the call stack unwinds into the applicaton.

Multi-thread safe: A programmer attempting to use this module in a
multithreaded system must heed all precautions attendent with using
fully share memory address spaces. To make this module multithread
safe, global locks must used to prevent concurrent access to the
error trace.

\subsection{More on the error trace}
\begin{verbatim}
      #include <gandalf/common/gan_err_trace.h>
\end{verbatim}
This module implements the error trace used in {\tt gan\_err.[ch]}.
The header file would not normally be included explicitly in library or
application code. An error trace is a last-in first-out (LIFO) stack for
temporarily holding details of multiple error events
until an application is ready to read the stack.

The stack is usually flushed by the function in a
sequence of nested calls that initially detects an
error. As the call stack unwinds the successive
functions also register errors, but they should not
flush the error trace.

The stack is implemented as a linked list of error
records. If in the process of allocating heap memory
for a new error record a memory error occurs, then
this is refered to as a deep error.

The stack always maintains two preallocated and unused
error records for storing the details of the deep
error and the error that was in the process of being
registered when the deep error occured.

Even if the top of the stack holds a deep error record
and the two preallocated records are used, new errors
can still be registered into the trace. These attempts
may lead to repeated deep errors, in which case the
top deep error serves as an indicator of the deep
continuing error state. However, if the registration
process is successful (because in the intervening
time, some external agent has {\tt free}'d heap memory) the
old deep error record is left on the stack and the new
errors are registered on top of it.

To ensure that the stack has at least two preallocated
records at process startup time, the bottom and second
to bottom records of the stack use statically
allocated memory. These can never be dynamically {\tt free}'d.

To do this, an external module must define two static
error records. In {\tt gan\_err.c}, this is implemented as:
\begin{verbatim}
      /* The error trace */

      /* Statically allocate last and 2nd to last records for error trace */
      static Gan_ErrorTrace record_last = { NULL, GAN_ET_YES, GAN_ET_NO,
                                            GAN_ET_YES, NULL,
                                            GAN_EC_DFT_SPARE, NULL, 0, NULL };
      static Gan_ErrorTrace record_2nd_last = { &record_last, GAN_ET_YES, GAN_ET_NO,
                                                GAN_ET_YES, NULL,
                                                GAN_EC_DFT_SPARE, NULL, 0, NULL};

      /* Address of error trace (i.e. top of LIFO stack) */
      static Gan_ErrorTrace * gan_err_trace_top = &record_2nd_last;
\end{verbatim}
The symbol {\tt gan\_et\_get\_record\_first()} refers to the current top
of stack and is passed into the functions defined in this module as argument 1.

NB. A statically allocated string containing the deep error text message must
also exist, but this is defined in {\tt gan\_err\_trace.c}.

\section{Error tests and codes} \label{error-sec}
\begin{verbatim}
      #include <gandalf/common/misc_error.h>
\end{verbatim}
The Gandalf error package itself is described in
Section~\ref{error-handling-sec}.  The {\tt misc\_error.[ch]}
defines some Gandalf-specific error codes.
%The test macros should be used only to test for error that indicate
%program bugs, since they are deactivated when {\tt NDEBUG} is set.
