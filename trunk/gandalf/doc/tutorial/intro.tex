\chapter{Introduction} \label{introduction-chapter}
Gandalf is a C library designed to support the development of computer
vision applications. For installation instructions see the INSTALL file
provided with the distribution. Gandalf is divided into the following
packages:--
\begin{itemize}
  \item The {\bf Common} package contains general purpose tools used by other
	packages. It includes routines for memory allocation, linked lists,
	various numerical functions, array manipulations and error handling.
  \item The {\bf Linear algebra} package includes vector/matrix manipulation
	routines both for small objects (sizes 2,3,4) and general size vectors
	\& matrices. The latter optionally employs LAPACK for certain
	operations, where a fast implementation of LAPACK is available.
  \item The {\bf Image} package contains low-level image creation and
	manipulation functions, supporting grey-level, RGB colour images with
	or without alpha channels, with various different pixel depths.
	2D and 3D vector field images are also supported.
  \item The {\bf Vision} package contains some useful vision utility modules,
	including edge/line/corner feature detection, and geometrical
	fitting routines.
\end{itemize}
In the following chapters we introduce each package in turn, describing
the scope and applications of the package, followed by tutorial examples
showing how to build a user application using Gandalf. This is followed
by chapter~\ref{testing-chapter} describing the testing framework used by
Gandalf. Firstly we describe the conventions and style of Gandalf.

\section{Conventions and style} \label{conventions-sec}
\subsection{Function/name prefixes and case}
All Gandalf functions begin with the {\tt gan\_...} prefix, to minimise the
possibility of name conflicts. Function names are written in lower case, with
a few exceptions where upper case is used for single letters
in a function name. Defined constants
and enum values are written in upper case, for instance {\tt GAN\_TRUE} and
{\tt GAN\_FALSE} for the boolean type {\tt Gan\_Bool}. Structure, union
and enum type names are capitalised, for instance {\tt Gan\_List},
{\tt Gan\_Vector} and {\tt Gan\_Image}, the names of the Gandalf linked list,
vector and image structures.

\subsection{``Quick'', ``slow'' and ``in-place'' Gandalf routines}
A convention used extensively in the linear algebra and image packages is
to provide ``quick'' and ``slow'' versions of the same operation,
indicating the difference using the suffices {\tt ...\_q} and {\tt ...\_s}
respectively. The meanings of the suffices vary, but are one of:
\begin{enumerate}
  \item The ``slow'' version dynamically allocates the memory to hold the
	result, whereas the ``quick'' version uses a pre-allocated result
	passed in to the function, avoiding repetitively allocating and
	freeing memory when the function is called several times.
	This is the convention used in the image package and for general size
	matrix and vector functions in the linear algebra package.
  \item The ``slow'' version returns a structure as its result, and the
	``quick'' version expects a pointer to the result structure to be
	passed in. This is the convention used for fixed-size matrices and
	vectors in the linear algebra package.
\end{enumerate}
There is also a conventional {\tt ...\_i} suffix for operations that overwrite
one of the input arguments with the result.

\subsection{Variable argument list functions}
Variable argument lists have been avoided where possible, because of the
lack of argument type and number checking. All functions with a variable
argument list have a {\tt ...\_va} suffix.

\subsection{Dynamic self-modification}
Another important design feature of Gandalf is the ability of many Gandalf
objects to dynamically modify themselves. Thus Gandalf matrices, vectors and
images can be redefined at any time to a new type and/or size.
This feature has many benefits, among which are:
\begin{enumerate}
  \item Results of computations may be overwritten in-place on the input
	arguments, in many cases without any memory or computational overhead.
	For instance, the Cholesky factorisation of a positive-definite
	symmetric matrix is a lower triangular matrix factor. In Gandalf
	the operation can be performed in-place on the symmetric matrix
	input argument, producing a lower triangular matrix.
	See the function {\tt gan\_squmat\_cholesky\_i()}.

  \item The same structure can be used to store the results of several
	computations, whether or not the sizes and types of the output
	results vary. This saves many calls to {\tt malloc()} and {\tt free()}.
\end{enumerate}

\subsection{Coordinate pair ordering}
A large number of Gandalf routines require horizontal and vertical sizes
or coordinates to be passed as arguments, whether representing the dimensions
of a matrix or coordinates of a pixel in an image. The ordering of the
arguments in such cases is problematic. For specifying a matrix element
the most natural ordering is to have the row coordinate first, corresponding
to the coordinate order $i,j$ for matrix element $A_{ij}$.
On the contrary, for image coordinates there is no obvious convention.
In Gandalf a universal convention is applied that in all such cases the
{\em vertical} or row coordinate is the first argument.

\subsection{Pixel values}
Gandalf uses the convention that image pixels represented in floating point
are always in the range [0,1]. In integer formats the range is the whole
range of the type, whether character, short integer, integer or long integer.
This convention is applied for all conversions between pixels, as well as
the image processing and display routines. When programming with Gandalf
you should where possible stick to this convention.

\subsection{Image orientation}
The normal convention is that the first (zero) row of an image is at the top,
and Gandalf uses this convention. It is only relevant in certain image
processing and vision algorithms where the image orientation affects
the results. OpenGL assumes the opposite convention, which Gandalf gets
around by displaying images using a negative vertical scaling of image
coordinates. Currently the only image processing algorithm which assumes
an image orientation is the Canny edge detector.

\subsection{Error checking}
Gandalf is written with a large amount of automatic error checking, for
instance testing for accesses to illegal matrix or image data.
When {\tt NDEBUG} is defined, most of these tests are switched off,
using a similar mechanism to {\tt assert()}. It is up to the programmer
to decide which tests are for bugs in the code, and can therefore be
turned off when code is compiled for release by defining {\tt NDEBUG}, and
which are data-dependent errors and should therefore remain.

