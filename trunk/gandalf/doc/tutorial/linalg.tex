\chapter{The Linear Algebra Package} \label{linalg-chapter}
The {\bf linear algebra} package covers matrix and vector manipulations,
matrix decompositions and other operations.
To be able to use any function or structure in the linear algebra package
use the declaration
\begin{verbatim}
      #include <gandalf/linalg.h>
\end{verbatim}
but including individual module header files instead will speed up program
compilation.
There are two parts to the linear algebra package, one dealing with small
fixed size vectors and matrices, between size two and four, and another
for general size objects. This separation allows the most efficient
implementation of linear algebra operations when the size of the objects
is known and small. Being designed to support image- and goemetry-based
applications, the size range from two to four allows 2D image and
3D camera/world objects to be manipulated, in homogeneous coordinates
where required; thus four is the natural size limit for Gandalf.

A major design feature of the linear algebra package is the application
of {\em implicit operations}. By this is meant, for example, that adding
a matrix to the transpose of another matrix is a {\em one} step operation.
Rather than transposing the matrix and then adding it, there is a specific
Gandalf routine to apply the operation ``add matrix to transpose of another
matrix'', implemented by indexing the elements of the second matrix in
transposed order. This principle increases greatly the number of routines
that Gandalf implements, but also greatly increases the efficiency of
the package. It can also help to reduce errors, in the case of implicit
matrix inverse. Let us say that we want to compute a matrix/vector product
where the matrix is to be inverted:
\[ \yvec = A\inv \xvec
\]
If $A$ happens to be a diagonal matrix, it makes sense to apply the inverse
operation implicitly, inside the product operation. This is because if, for
example, we are dealing with vectors \& matrix of size 2, we have
\[ \xvec=\beginm{c} x_1 \\ x_2 \endm,\;\;\yvec=\beginm{c} y_1 \\ y_2 \endm,\;\;
   A=\beginm{cc} A_{11} & 0 \\ 0 & A_{22} \endm,
\]
and the operations required are
\[ y_1 \leftarrow \frac{x_1}{A_{11}}, \;\;\; y_2 \leftarrow \frac{x_2}{A_{22}}.
\]
Applying the inverse firstly to $A$ and then computing the product would
involve the following operations
\[ A_{11} \leftarrow \frac{1}{A_{11}}, \;\;
   A_{22} \leftarrow \frac{1}{A_{22}}, \;\;
   y_1 \leftarrow A_{11} x_1, \;\;\; y_2 \leftarrow A_{22} x_2.
\]
This has two drawbacks: the two stages of inverting followed by multiplication
reduces the accuracy of the result compared to a single division operation,
and the $A$ matrix is overwritten with the inverse of $A$, which is not
normally what is wanted (explicit matrix inverse is to be avoided wherever
possible). As we shall see, Gandalf implements a comprehensive set of
implicit transpose and inverse operations, which apply when the matrix
involved is diagonal (as above) or triangular. For these types of matrix
the inverse operation can be conjoined with multiplication, such that
effectively only one operation is performed.
Implicit inverse does not apply to symmetric or general square matrices,
because there is no way of conjoining the inverse with multiplication in
the same way.

\section{Fixed size matrices and vectors} \label{fixed-size-sec}
\subsection{Fixed size vectors} \label{fixed-size-vec-sec}
Vectors of sizes two, three and four are represented by specific structures
in Gandalf. The structure and function definitions are nearly identical,
so we shall only describe the workings of 3-vectors. To use 3-vectors
include the header file
\begin{verbatim}
      #include <gandalf/linalg/3vector.h>
\end{verbatim}
for double precision, or
\begin{verbatim}
      #include <gandalf/linalg/3vectorf.h>
\end{verbatim}
for single precision 3-vectors.
A double precision 3-vector is defined as
\begin{verbatim}
      typedef struct Gan_Vector3
      {
         double x, y, z;
      } Gan_Vector3, Gan_Vector3_d
\end{verbatim}
and a single precision 3-vector as
\begin{verbatim}
      typedef struct Gan_Vector3_f
      {
         float x, y, z;
      } Gan_Vector3_f;
\end{verbatim}

Most of the routines below return a pointer to the result vector/matrix.
This may be used as an argument to another routine, although care must
be taken with macros as regards multiple evaluation. The routines are very
safe, because everything using fixed size vectors \& matrices can be written
to involve only automatic variables with no dynamic allocation, and the only
failure modes are arithmetic overflow (Gandalf does not check for this).
The few exceptions are noted in the text.

\subsubsection{Creating and accessing fixed size vectors}
Single fixed size vectors are such simple objects, it makes sense to
normally use declare structure variables directly, rather than use pointers
to structures created by {\tt malloc()}. So to create a double precision
3-vector, use the declaration
\begin{verbatim}
      Gan_Vector3 v3x;
\end{verbatim}
From now on, we shall describe the functions for double precision vectors
only. Single precision functions are very similar and will be explained below.
Setting the coordinates of a 3-vector can be achieved by one of
\begin{enumerate}
  \item Initialising the 3-vector when it is created, as in
\begin{verbatim}
      Gan_Vector3 v3x = {1.0, 2.0, 3.0};
\end{verbatim}

  \item Accessing the structure elements directly:
\begin{verbatim}
      v3x.x = 1.0; v3x.y = 2.0; v3x.z = 3.0;
\end{verbatim}
  \item Using the macro call
\begin{verbatim}
      gan_vec3_fill_q ( &v3x, 1.0, 2.0, 3.0 );
\end{verbatim}
Note that the Gnu C compiler prints a warning when the above call is compiled,
and also for most other similar calls in the linear algebra package.
This warning can be avoided by inserting an initial {\tt (void)} cast:
\begin{verbatim}
      (void)gan_vec3_fill_q ( &v3x, 1.0, 2.0, 3.0 );
\end{verbatim}
We omit this cast in the following to keep the exposition simple.
  \item The equivalent function call
\begin{verbatim}
      v3x = gan_vec3_fill_s ( 1.0, 2.0, 3.0 );
\end{verbatim}
\end{enumerate}
Setting a 3-vector to zero can be accomplished using one of the calls
\begin{verbatim}
      gan_vec3_zero_q ( &v3x ); /* macro, OR */
      v3x = gan_vec3_zero_s(); /* function call */
\end{verbatim}

Copying 3-vectors can be accomplished either by direct assignment
\begin{verbatim}
      Gan_Vector3 v3y;

      v3y = v3x;
\end{verbatim}
or by use of the one of the routines
\begin{verbatim}
      gan_vec3_copy_q ( &v3x, &v3y ); /* macro, OR */
      v3y = gan_vec3_copy_s ( &v3x ); /* function call */
\end{verbatim}

\subsubsection{Fixed size vector addition}
To add two 3-vectors use either the macro
\begin{verbatim}
      Gan_Vector3 v3z;

      gan_vec3_add_q ( &v3x, &v3y, &v3z ); /* macro */
\end{verbatim}
or the function
\begin{verbatim}
      v3z = gan_vec3_add_s ( &v3x, &v3y ); /* function call */
\end{verbatim}
See the discussion of ``quick'' and ``slow'' versions of the same operation,
identified by the {\tt ...\_q} and {\tt ...\_s} suffices,
in Section~\ref{conventions-sec}. In this case, the ``slow'' version
{\tt gan\_vec3\_add\_s()} has the overhead of a function call relative
to the ``quick'' version {\tt gan\_vec3\_add\_q()}, so the latter should
be used unless the input vectors are not simple variables (i.e. they might
be elements of arrays), in which case the repeated evaluation required
by the macro version might be slower.

There are also in-place versions of the add operation, which overwrite one
of the input vectors with the result. The macro operations
\begin{verbatim}
      gan_vec3_add_i1 ( &v3x, &v3y ); /* result in-place in v3x */
\end{verbatim}
and
\begin{verbatim}
      gan_vec3_add_i2 ( &v3x, &v3y ); /* result in-place in v3y */
\end{verbatim}
produce the same result but overwrite respectively the first {\tt v3x}
and the second {\tt v3y} argument with the result. There is also a
more explicit macro
\begin{verbatim}
      gan_vec3_increment ( &v3x, &v3y ); /* result in-place in v3x */
\end{verbatim}
which increments {\tt v3x} by {\tt v3y}, i.e. identical to
{\tt gan\_vec3\_add\_i1()}. Note that if one of the input arguments is
a non-trivial expression, and the result is being overwritten on the other,
use the function {\tt gan\_vec3\_add\_s()}, as in
\begin{verbatim}
      Gan_Vector3 av3x[100];

      /* ... fill av3x array ... */
      v3x = gan_vec3_add_s ( &v3x, &av3x[33] );
\end{verbatim}

\subsubsection{Fixed size vector subtraction}
To subtract 3-vectors use the equivalent macros and functions
\begin{verbatim}
      gan_vec3_sub_q ( &v3x, &v3y, &v3z ); /* macro */
      v3z = gan_vec3_sub_s ( &v3x, &v3y ); /* function call */
      gan_vec3_sub_i1 ( &v3x, &v3y ); /* result in-place in v3x */
      gan_vec3_sub_i2 ( &v3x, &v3y ); /* result in-place in v3y */
      gan_vec3_decrement ( &v3x, &v3y ); /* result in-place in v3x */
\end{verbatim}

\subsubsection{Rescaling a fixed size vector}
There are similar functions to multiply a 3-vector by a scalar
\begin{verbatim}
      double ds;

      gan_vec3_scale_q ( &v3x, ds, &v3z ); /* macro */
      v3z = gan_vec3_scale_s ( &v3x, ds ); /* function call */
      gan_vec3_scale_i ( &v3x, ds ); /* macro, result in-place in v3x */
\end{verbatim}
to divide a 3-vector by a (non-zero) scalar
\begin{verbatim}
      gan_vec3_divide_q ( &v3x, ds, &v3z ); /* macro */
      v3z = gan_vec3_divide_s ( &v3x, ds ); /* function call */
      gan_vec3_divide_i ( &v3x, ds ); /* macro, result in-place in v3x */
\end{verbatim}
to negate a 3-vector
\begin{verbatim}
      gan_vec3_negate_q ( &v3x, &v3z ); /* macro */
      v3z = gan_vec3_negate_s ( &v3x ); /* function call */
      gan_vec3_negate_i ( &v3x ); /* macro, result in-place in v3x */
\end{verbatim}
and to scale a 3-vector to unit length (2-norm)
\begin{verbatim}
      gan_vec3_unit_q ( &v3x, &v3z ); /* macro */
      v3z = gan_vec3_unit_s ( &v3x ); /* function call */
      gan_vec3_unit_i ( &v3x ); /* macro, result in-place in v3x */
\end{verbatim}
The last guarantees than the total squared element of the vector
returned by {\tt gan\_vec3\_sqrlen()} will be one; see below.

{\tt Error detection:} If zero is passed as the scalar value to the
{\tt ...\_divide\_[qi]()} routines, {\tt NULL} will be returned, while the
{\tt ...\_divide\_s()} routines will abort the program. You should add tests
for division by zero {\em before} calling any of the {\tt ...\_divide\_[qsi]()}
routines. Similarly, the {\tt ...\_unit\_[qsi]()} routines will fail if
the input vector contains only zeros. If this is a possibility then the
program should check beforehand.

\subsubsection{Fixed size vector products}
Vector dot product (scalar product) is compute using the alternatives
\begin{verbatim}
      ds = gan_vec3_dot_q ( &v3x, &v3y ); /* macro, or */
      ds = gan_vec3_dot_s ( &v3x, &v3y ); /* function call */
\end{verbatim}
Similarly, to compute the squared length of a 3-vector, use
\begin{verbatim}
      ds = gan_vec3_sqrlen_q ( &v3x ); /* macro, or */
      ds = gan_vec3_sqrlen_s ( &v3x ); /* function call */
\end{verbatim}

For 3-vectors only we also have the cross product (vector product)
\begin{verbatim}
      gan_vec3_cross_q ( &v3x, &v3y, &v3z ); /* macro */
      v3z = gan_vec3_cross_s ( &v3x, &v3y ); /* function call */
\end{verbatim}

There are also outer products formed by two vectors, producing a matrix.
These functions are described in Section~\ref{fixed-vector-outer-sec}.

\subsubsection{Fixed size vector file I/O}
Gandalf supports both ASCII and binary format file I/O of vectors and
matrices. Both formats use standard {\tt FILE~*} file streams.
ASCII format is obviously more convenient to use, while binary format is
more compact and guarantees no loss of precision when the data is read.
To print a 3-vector in ASCII format, use
\begin{verbatim}
      Gan_Bool gan_vec3_fprint ( FILE *fp, Gan_Vector3 *p,
                                 const char *prefix, int indent, const char *fmt );
\end{verbatim}
{\tt prefix} is a prefix string to print before the vector itself, {\tt indent}
is the number of spaces to indent the vector by, and {\tt fmt} is a format
string to use when printing the vector, e.g. {\tt "\%f"}. So for example
\begin{verbatim}
      FILE *pfFile;

      pfFile = fopen ( "/tmp/vectors", "w" );
      gan_vec3_fill_q ( &v3x, 1.0, 2.0, 3.0 );
      gan_vec3_fprint ( pfFile, &v3x, "Example vector", 3, "%f" );
\end{verbatim}
will print the output
\begin{verbatim}
         Example vector: 1.000000 2.000000 3.000000
\end{verbatim}
to the file {\tt "/tmp/vectors"}.
There is also a version {\tt gan\_vec3\_print()} for printing to
standard output:
\begin{verbatim}
      Gan_Bool gan_vec3_print ( Gan_Vector3 *p,
                                const char *prefix, int indent, const char *fmt );
\end{verbatim}
The corresponding input function is
\begin{verbatim}
      Gan_Bool gan_vec3_fscanf ( FILE *fp, Gan_Vector3 *p,
                                 char *prefix, int prefix_len )
\end{verbatim}
which reads the vector from the file stream {\tt fp} into the 3-vector
pointer {\tt p}. It also reads the prefix string (up to the specified maximum
length {\tt prefix\_len}), which can be compared with the expected prefix
string to check for consistency.

Binary file I/O is handled by the functions {\tt gan\_vec3\_fwrite()}
and {\tt gan\_vec3\_fread()}. To write a 3-vector in binary format use
\begin{verbatim}
      Gan_Bool gan_vec3_fwrite ( FILE *fp, Gan_Vector3 *p, gan_ui32 magic_number );
\end{verbatim}
The {\tt magic\_number} takes the same role as the {\tt prefix} string
in {\tt gan\_vec3\_fprint()}, and is written into the file so that it can be
used later to identify the vector when it is read back using
\begin{verbatim}
      Gan_Bool gan_vec3_fread ( FILE *fp, Gan_Vector3 *p, gan_ui32 *magic_number );
\end{verbatim}

{\tt Error detection:} The I/O routines return a boolean value, returning
{\tt GAN\_TRUE} on success, {\tt GAN\_FALSE} on failure, invoking the Gandalf
error handle in the latter case.

\subsubsection{Conversion from general to fixed size vector}
Functions are provided to convert a general vector {\tt Gan\_Vector}
{\tt Gan\_Vector3}, provided that the general vector has actually been
created with size three. So for instance
\begin{verbatim}
      Gan_Vector *pvx;

      pvx = gan_vec_alloc(3);
      gan_vec_fill_va ( pvx, 3, 2.0, 3.0, 4.0 );
      gan_vec3_from_vec_q ( pvx, &v3x );
\end{verbatim}
or
\begin{verbatim}
      v3x = gan_vec3_from_vec_s ( pvx );
\end{verbatim}
fill {\tt v3x} with the same values as the general vector.
Calling these functions with a general vector {\tt pvx} not having
the same size as the fixed size vector is an error.

{\tt Error detection:} The conversion routines return the pointer to the
filled fixed size vector, or {\tt NULL} on failure, invoking the Gandalf
error handle in the latter case.

\subsubsection{Single precision fixed size vectors}
\begin{verbatim}
      #include <gandalf/linalg/3vectorf.h>
\end{verbatim}
There is an identical set of functions for handling single precision
floating point vectors, the names of which are obtained by replacing
{\tt "gan\_vec3\_..."} in the above functions with {\tt "gan\_vec3f\_..."}.
For example, to add two single precision 3-vectors use
\begin{verbatim}
      Gan_Vector3_f v3x, v3y, v3z;

      /* ... fill v3x and v3y ... */
      gan_vec3f_add_q ( &v3x, &v3y, &v3z );
\end{verbatim}

\subsubsection{Other types of fixed size vector}
As well as floating point coordinates, integer vectors are often useful.
Although Gandalf does not currently provide sets of functions handling
vectors with integer coordinates, structures are defined as follows
\begin{verbatim}
      /* structure definition for unsigned character 3-vector */
      typedef struct Gan_Vector3_uc
      {
         unsigned char x, y, z;
      } Gan_Vector3_uc;

      /* structure definition for short integer 3-vector */
      typedef struct Gan_Vector3_s
      {
         short x, y, z;
      } Gan_Vector3_s;

      /* structure definition for unsigned short integer 3-vector */
      typedef struct Gan_Vector3_us
      {
         unsigned short x, y, z;
      } Gan_Vector3_us;

      /* structure definition for integer 3-vector */
      typedef struct Gan_Vector3_i
      {
         int x, y, z;
      } Gan_Vector3_i;

      /* structure definition for unsigned integer 3-vector */
      typedef struct Gan_Vector3_ui
      {
         unsigned int x, y, z;
      } Gan_Vector3_ui;
\end{verbatim}

\subsubsection{Other sizes of fixed size vector}
Gandalf supports fixed size vectors with sizes two, three and four.
The functions described above for size three vectors are repeated
for sizes two and four, both single and double precision, in the header
files
\begin{verbatim}
      #include <gandalf/linalg/2vector.h>   /* double precision */
      #include <gandalf/linalg/2vectorf.h>  /* single precision */
      #include <gandalf/linalg/4vector.h>   /* double precision */
      #include <gandalf/linalg/4vectorf.h>  /* single precision */
\end{verbatim}

\subsubsection{Setting/extracting parts of fixed size vectors}
Apart from the cross product routines {\tt gan\_vec3\_cross\_[qs]()}
defined only for 3-vectors, there are a few other miscellaneous routines which
apply to a subset of the fixed size vectors. These routines enable setting
or extracting parts of a fixed size vector using another fixed size with
a different size. The most comprehensive set of such routines is for
vectors of size four. So for instance to extract the first three elements
of a 4-vector and write them into a 3-vector, use
\begin{verbatim}
      Gan_Vector3 v3x;
      Gan_Vector4 v4x;

      gan_vec4_fill_q ( &v4x, 1.0, 2.0, 3.0, 4.0 );
      gan_vec4_get_v3t_q ( &v4x, &v3x ); /* macro */
\end{verbatim}
or alternatively
\begin{verbatim}
      v3x = gan_vec4_get_v3t_s ( &v4x ); /* function */
\end{verbatim}
both of which set {\tt v3x} to {\tt \{1.0, 2.0, 3.0\}}.
To build a 4-vector from a 3-vector and a scalar use
\begin{verbatim}
      gan_vec3_fill_q ( &v3x, 1.0, 2.0, 3.0 );
      gan_vec4_set_parts_q ( &v4x, &v3x, 4.0 ); /* macro */
\end{verbatim}
or alternatively
\begin{verbatim}
      v4x = gan_vec4_set_parts_s ( &v3x, 4.0 ); /* function */
\end{verbatim}
both of which set {\tt v4x} to {\tt \{1.0, 2.0, 3.0, 4.0\}}.
To build a 4-vector from two 2-vectors use
\begin{verbatim}
      Gan_Vector3 v2xt, v2xb;
      Gan_Vector4 v4x;

      gan_vec2_fill_q ( &v2xt,  1.0, 2.0 );
      gan_vec2_fill_q ( &v2xb, 3.0, 4.0 );
      gan_vec4_set_blocks_q ( &v4x, &v2xt, &v2xb ); /* macro */
\end{verbatim}
(note that the ``t'' and ``b'' in {\tt v2xt} and {\tt v2xb} stand for the
``top'' and ``bottom'' parts of vector $\xvec$), or alternatively
\begin{verbatim}
      v4x = gan_vec4_set_blocks_s ( &v2xt, &v2xb ); /* function */
\end{verbatim}
both of which again set {\tt v4x} to {\tt \{1.0, 2.0, 3.0, 4.0\}}.

For 3-vectors the equivalent set of functions involves splitting the
3-vector into the {\tt x,y} coordinates as a 2-vector and {\tt z} as the
scalar. Then we have
\begin{verbatim}
      Gan_Vector2 v2xt;
      Gan_Vector3 v3x;

      gan_vec3_fill_q ( &v3x, 1.0, 2.0, 3.0 );
      gan_vec3_get_v2t_q ( &v3x, &v2xt ); /* macro, or */
      v2xt = gan_vec3_get_v2t_s ( &v3x ); /* function */
\end{verbatim}
the last two lines of which both set {\tt v2xt} to {\tt \{1.0, 2.0\}}.
To build a 3-vector from a 2-vector and a scalar use
\begin{verbatim}
      gan_vec2_fill_q ( &v2xt, 1.0, 2.0 );
      gan_vec3_set_parts_q ( &v3x, &v2xt, 3.0 ); /* macro, or */
      v3x = gan_vec3_set_parts_s ( &v2xt, 3.0 ); /* function */
\end{verbatim}
both of which set {\tt v3x} to {\tt \{1.0, 2.0, 3.0\}}.

\subsection{Fixed size matrices} \label{fixed-size-mat-sec}
Matrices of sizes $2\times 2$, $2\times 3$, $2\times 4$, $3\times 3$,
$3\times 4$ and $4\times 4$ are represented by specific structures
in Gandalf. There is a large set of functions which is repeated across
every size of matrix. There is also a set of functions specific to square
matrices. In both cases we will choose a single size of matrix and describe
the functions available for that size. The sizes we will use are $3\times 4$
matrices for the functions available to every size of matrix, and $3\times 3$
matrices for the functions specific to square matrices.

\subsubsection{Definitions of fixed size matrice}\label{fixed34-sec}
To use $3\times 4$ matrices or $3\times 3$ matrices
include the header files
\begin{verbatim}
      #include <gandalf/linalg/3x4matrix.h> /* OR */
      #include <gandalf/linalg/3x3matrix.h>
\end{verbatim}
respectively. This is for double precision matrix elements.
The files to include for single precision elements are
\begin{verbatim}
      #include <gandalf/linalg/3x4matrixf.h> /* OR */
      #include <gandalf/linalg/3x3matrixf.h>
\end{verbatim}
A double precision $3\times 4$ matrix is defined as
\begin{verbatim}
      typedef struct Gan_Matrix34
      {
         double xx, xy, xz, xw,
                yx, yy, yz, yw,
                zx, zy, zz, zw;
      } Gan_Matrix34;
\end{verbatim}
A $3\times 3$ matrix is similarly defined as
\begin{verbatim}
      typedef struct Gan_Matrix33
      {
         double xx, xy, xz,
                yx, yy, yz,
                zx, zy, zz;
      } Gan_Matrix33;
\end{verbatim}
For square matrices there is also a specific structure to handle symmetric
and triangular matrix structures, as follows:
\begin{verbatim}
      #ifndef NDEBUG
      /* square matrix type, for setting and checking in debug mode */
      typedef enum { GAN_SYMMETRIC_MATRIX33, GAN_LOWER_TRI_MATRIX33 }
       Gan_SquMatrix33Type;
      #endif /* #ifndef NDEBUG */

      /* structure definition for square 3x3 matrix */
      typedef struct Gan_SquMatrix33
      {
      #ifndef NDEBUG
         /* square matrix type, for setting and checking in debug mode */
         Gan_SquMatrix33Type type;
      #endif /* #ifndef NDEBUG */

         /* matrix data */
         double xx,
                yx, yy,
                zx, zy, zz;
      } Gan_SquMatrix33;
\end{verbatim}
Note that the matrix type field {\tt Gan\_SquMatrix33} is only used in
debug compilation mode ({\tt NDEBUG} not defined). The {\tt type} field is
actually invisible to the programmers' interface to the Gandalf functions,
and is used merely for internal checking. Note also that Gandalf does not
provide explicit support for {\em upper} triangular fixed size matrices
(although it does for general size matrices;
see Section~\ref{general-size-sec}). Any operations involving upper triangular
matrices can be implemented using implicit transpose of a lower triangular
matrix.

Single precision structures are defined similarly, with the names changed
to {\tt Gan\_Matrix34\_f}, {\tt Gan\_SquMatrix33\_f} etc.

\subsubsection{Creating and accessing fixed size matrices}
Single fixed size matrices are such simple objects, it makes sense to
normally use declare structure variables directly, rather than use pointers
to structures created by {\tt malloc()}. So to create a double precision
$3\times 4$ matrix, use the declaration
\begin{verbatim}
      Gan_Matrix34 m34A;
\end{verbatim}
From now on, we shall describe the routines for double precision matrices
only. Single precision functions are very similar and will be explained below.
Setting the coordinates of a $3\times 4$ matrix can be achieved by one of
\begin{enumerate}
  \item Initialising the $3\times 4$ matrix when it is created, as in
\begin{verbatim}
      Gan_Matrix34 m34A = {1.0,  2.0,  3.0,  4.0,
                           5.0,  6.0,  7.0,  8.0,
                           9.0, 10.0, 11.0, 12.0};
\end{verbatim}

  \item Accessing the structure elements directly:
\begin{verbatim}
      m34A.xx = 1.0; m34A.xy = 2.0; m34A.xz = 3.0; /* etc. */
\end{verbatim}
  \item Using the macro call
\begin{verbatim}
      gan_mat34_fill_q ( &m34A, 1.0,  2.0,  3.0,  4.0,
                                5.0,  6.0,  7.0,  8.0,
                                9.0, 10.0, 11.0, 12.0 );
\end{verbatim}
Note that the Gnu C compiler prints a warning when the above call is compiled,
and also for most other similar calls in the linear algebra package.
This warning can be avoided by inserting an initial {\tt (void)} cast:
\begin{verbatim}
      (void)gan_mat34_fill_q ( &m34A, 1.0,  2.0,  3.0,  4.0,
                                      5.0,  6.0,  7.0,  8.0,
                                      9.0, 10.0, 11.0, 12.0 );
\end{verbatim}
We omit this cast in the following to keep the exposition simple.
  \item The equivalent function call
\begin{verbatim}
      m34A = gan_mat34_fill_s ( 1.0,  2.0,  3.0,  4.0,
                                5.0,  6.0,  7.0,  8.0,
                                9.0, 10.0, 11.0, 12.0 );
\end{verbatim}
\end{enumerate}

The methods of initialising a $3\times 3$ matrix follow those listed above
for $3\times 4$ matrices, for instance
\begin{verbatim}
      Gan_Matrix33 m33A;

      gan_mat33_fill_q ( &m33A, 1.0, 2.0, 3.0,
                                4.0, 5.0, 6.0,
                                7.0, 8.0, 9.0 ); /* OR */
      m33A = gan_mat33_fill_s ( 1.0, 2.0, 3.0,
                                4.0, 5.0, 6.0,
                                7.0, 8.0, 9.0 );
\end{verbatim}
For a symmetric or lower triangular
{\tt Gan\_SquMatrix33} matrix, direct initialisation (options 1 and 2 above) is
not advisable, because of the {\tt type} field of the structure whose
presence depends on {\tt NDEBUG}, Instead use the macro calls
\begin{verbatim}
      Gan_SquMatrix33 sm33S, sm33L;

      /* symmetric matrix */
      gan_symmat33_fill_q ( &sm33S, 1.0,
                                    2.0, 3.0,
                                    4.0, 5.0, 6.0 );

      /* lower triangular matrix */
      gan_ltmat33_fill_q ( &sm33L, 1.0,
                                   2.0, 3.0,
                                   4.0, 5.0, 6.0 );
\end{verbatim}
The first of these fills the matrix without specifying the values above the
diagonal, and actually builds the matrix
\[ S = \beginm{ccc} 1 & 2 & 4 \\ 2 & 3 & 5 \\ 4 & 5 & 6\endm
\]
The second builds the lower triangular matrix
\[ L = \beginm{ccc} 1 & 0 & 0 \\ 2 & 3 & 0 \\ 4 & 5 & 6\endm
\]
Setting a fixed-size matrix to zero can be accomplished using one of the calls
\begin{verbatim}
      gan_mat34_zero_q ( &m34A ); /* OR */ m34A = gan_mat34_zero_s();
      gan_mat33_zero_q ( &m33A ); /* OR */ m33A = gan_mat33_zero_s();
      gan_symmat33_zero_q ( &sm33S ); /* OR */ sm33S = gan_symmat33_zero_s();
      gan_ltmat33_zero_q ( &sm33L ); /* OR */ sm33L = gan_ltmat33_zero_s();
\end{verbatim}

Setting a square matrix to identity is achieved using
\begin{verbatim}
      gan_mat33_ident_q ( &m33A ); /* OR */ m33A = gan_mat33_ident_s();
      gan_symmat33_ident_q ( &sm33S ); /* OR */ sm33S = gan_symmat33_ident_s();
      gan_ltmat33_ident_q ( &sm33L ); /* OR */ sm33L = gan_ltmat33_ident_s();
\end{verbatim}

Copying $3\times 4$ matrices can be accomplished either by direct assignment
\begin{verbatim}
      Gan_Matrix34 m34B;

      m34B = m34A;
\end{verbatim}
or by use of one of the routines
\begin{verbatim}
      gan_mat34_copy_q ( &m34A, &m34B ); /* macro, OR */
      m34B = gan_mat34_copy_s ( &m34A ); /* function call */
\end{verbatim}
The methods of copying general, symmetric and lower triangular matrices follow
that of $3\times 4$ matrices.

\subsubsection{Fixed size matrix addition}
To add two $3\times4$ matrices use either the macro
\begin{verbatim}
      Gan_Matrix34 m34C;

      /* ... set up m34A, m34B using e.g. gan_mat34_fill_q() ... */
      gan_mat34_add_q ( &m34A, &m34B, &m34C ); /* macro */
\end{verbatim}
or the function
\begin{verbatim}
      m34C = gan_mat34_add_s ( &m34A, &m34B ); /* function call */
\end{verbatim}
See the discussion of ``quick'' and ``slow'' versions of the same operation,
identified by the {\tt ...\_q} and {\tt ...\_s} suffices,
in Section~\ref{conventions-sec}. In this case, the ``slow'' version
{\tt gan\_mat34\_add\_s()} has the overhead of a function call relative
to the ``quick'' version {\tt gan\_mat34\_add\_q()}, so the latter should
be used unless the input matrices are not simple variables (i.e. they might
be elements of arrays), in which case the repeated evaluation required
by the macro version might be slower.

There are also in-place versions of the add operation, which overwrite one
of the input matrices with the result. The macro operations
\begin{verbatim}
      gan_mat34_add_i1 ( &m34A, &m34B ); /* result in-place in m34A */
\end{verbatim}
and
\begin{verbatim}
      gan_mat34_add_i2 ( &m34A, &m34B ); /* result in-place in m34B */
\end{verbatim}
produce the same result but overwrite respectively the first {\tt m34A}
and the second {\tt m34B} argument with the result. There is also a
more explicit macro
\begin{verbatim}
      gan_mat34_increment ( &m34A, &m34B ); /* result in-place in m34A */
\end{verbatim}
which increments {\tt m34A} by {\tt m34B}, i.e. identical to
{\tt gan\_mat34\_add\_i1()}. Note that if one of the input arguments is
a non-trivial expression, and the result is being overwritten on the other,
use the function {\tt gan\_mat34\_add\_s()}, as in
\begin{verbatim}
      Gan_Matrix34 am34A[100];

      /* ... fill am34A array ... */
      m34A = gan_mat34_add_s ( &m34A, &am34A[33] );
\end{verbatim}

For general, symmetric and lower triangular $3\times 3$ matrices the
addition routines follow those for $3\times 4$ matrices. So for general
$3\times 3$ matrices we have the options
\begin{verbatim}
      Gan_Matrix33 m33A, m33B, m33C;

      /* ... set up m33A, m33B using e.g. gan_mat33_fill_q() ... */
      gan_mat33_add_q ( &m33A, &m33B, &m33C ); /* macro, OR */
      m33C = gan_mat33_add_s ( &m33A, &m33B ); /* function call */
      gan_mat33_add_i1 ( &m33A, &m33B ); /* macro, result in-place in m33A */
      gan_mat33_add_i2 ( &m33A, &m33B ); /* macro, result in-place in m33B */
      gan_mat33_increment ( &m33A, &m33B ); /* equivalent to gan_mat33_add_i1() */
\end{verbatim}
For symmetric $3\times 3$ matrices we have
\begin{verbatim}
      Gan_SquMatrix33 sm33Sa, sm33Sb, sm33Sc;

      /* ... set up sm33Sa, sm33Sb using e.g. gan_symmat33_fill_q() ... */
      gan_symmat33_add_q ( &sm33Sa, &sm33Sb, &sm33Sc ); /* macro, OR */
      sm33Sc = gan_symmat33_add_s ( &sm33Sa, &sm33Sb ); /* function call */
      gan_symmat33_add_i1 ( &sm33Sa, &sm33Sb ); /* macro, result in-place in sm33Sa */
      gan_symmat33_add_i2 ( &sm33Sa, &sm33Sb ); /* macro, result in-place in sm33Sb */
      gan_symmat33_increment ( &sm33Sa, &sm33Sb ); /* equivalent to gan_symmat33_add_i1() */
\end{verbatim}
Finally for lower triangular $3\times 3$ matrices we have
\begin{verbatim}
      Gan_SquMatrix33 sm33La, sm33Lb, sm33Lc;

      /* ... set up sm33La, sm33Lb using e.g. gan_ltmat33_fill_q() ... */
      gan_ltmat33_add_q ( &sm33La, &sm33Lb, &sm33Lc ); /* macro, OR */
      sm33Lc = gan_ltmat33_add_s ( &sm33La, &sm33Lb ); /* function call */
      gan_ltmat33_add_i1 ( &sm33La, &sm33Lb ); /* macro, result in-place in sm33La */
      gan_ltmat33_add_i2 ( &sm33La, &sm33Lb ); /* macro, result in-place in sm33Lb */
      gan_ltmat33_increment ( &sm33La, &sm33Lb ); /* equivalent to gan_ltmat33_add_i1() */
\end{verbatim}

For general square matrices there are routines to implement the operation
\[ S = A+A\tr,
\]
obtaining a symmetric matrix $S$ by adding together a general square matrix $A$
and its transpose. The routines for $3\times 3$ matrices are
\begin{verbatim}
      Gan_Matrix33 m33A;
      Gan_SquMatrix33 sm33S;

      /* set up m33A using e.g. gan_mat33_fill_q() */
      gan_mat33_saddT_q ( &m33A, &sm33S ); /* S = A+A^T, macro */
      sm33S = gan_mat33_saddT_s ( &m33A ); /* S = A+A^T, function call */
\end{verbatim}

\subsubsection{Fixed size matrix subtraction}
To subtract $3\times4$ matrices use the equivalent macros and functions
\begin{verbatim}
      gan_mat34_sub_q ( &m34A, &m34B, &m34C ); /* macro */
      m34C = gan_mat34_sub_s ( &m34A, &m34B ); /* function call */
      gan_mat34_sub_i1 ( &m34A, &m34B ); /* result in-place in m34A */
      gan_mat34_sub_i2 ( &m34A, &m34B ); /* result in-place in m34B */
      gan_mat34_decrement ( &m34A, &m34B ); /* result in-place in m34A */
\end{verbatim}
For general $3\times 3$ matrices we have the options
\begin{verbatim}
      Gan_Matrix33 m33A, m33B, m33C;

      /* ... set up m33A, m33B using e.g. gan_mat33_fill_q() ... */
      gan_mat33_sub_q ( &m33A, &m33B, &m33C ); /* macro, OR */
      m33C = gan_mat33_sub_s ( &m33A, &m33B ); /* function call */
      gan_mat33_sub_i1 ( &m33A, &m33B ); /* macro, result in-place in m33A */
      gan_mat33_sub_i2 ( &m33A, &m33B ); /* macro, result in-place in m33B */
      gan_mat33_decrement ( &m33A, &m33B ); /* equivalent to gan_mat33_sub_i1() */
\end{verbatim}
For symmetric $3\times 3$ matrices we have
\begin{verbatim}
      Gan_SquMatrix33 sm33Sa, sm33Sb, sm33Sc;

      /* ... set up sm33Sa, sm33Sb using e.g. gan_symmat33_fill_q() ... */
      gan_symmat33_sub_q ( &sm33Sa, &sm33Sb, &sm33Sc ); /* macro, OR */
      sm33Sc = gan_symmat33_sub_s ( &sm33Sa, &sm33Sb ); /* function call */
      gan_symmat33_sub_i1 ( &sm33Sa, &sm33Sb ); /* macro, result in-place in sm33Sa */
      gan_symmat33_sub_i2 ( &sm33Sa, &sm33Sb ); /* macro, result in-place in sm33Sb */
      gan_symmat33_decrement ( &sm33Sa, &sm33Sb ); /* equivalent to gan_symmat33_sub_i1() */
\end{verbatim}
Finally for lower triangular $3\times 3$ matrices we have
\begin{verbatim}
      Gan_SquMatrix33 sm33La, sm33Lb, sm33Lc;

      /* ... set up sm33La, sm33Lb using e.g. gan_ltmat33_fill_q() ... */
      gan_ltmat33_sub_q ( &sm33La, &sm33Lb, &sm33Lc ); /* macro, OR */
      sm33Lc = gan_ltmat33_sub_s ( &sm33La, &sm33Lb ); /* function call */
      gan_ltmat33_sub_i1 ( &sm33La, &sm33Lb ); /* macro, result in-place in sm33La */
      gan_ltmat33_sub_i2 ( &sm33La, &sm33Lb ); /* macro, result in-place in sm33Lb */
      gan_ltmat33_decrement ( &sm33La, &sm33Lb ); /* equivalent to gan_ltmat33_sub_i1() */
\end{verbatim}

\subsubsection{Rescaling a fixed size matrix}
There are similar functions to multiply a $3\times 4$ matrix by a scalar
\begin{verbatim}
      double ds;

      gan_mat34_scale_q ( &m34A, ds, &m34C ); /* macro */
      m34C = gan_mat34_scale_s ( &m34A, ds ); /* function call */
      gan_mat34_scale_i ( &m34A, ds ); /* macro, result in-place in m34A */
\end{verbatim}
to divide a $3\times 4$ matrix by a (non-zero) scalar
\begin{verbatim}
      gan_mat34_divide_q ( &m34A, ds, &m34C ); /* macro */
      m34C = gan_mat34_divide_s ( &m34A, ds ); /* function call */
      gan_mat34_divide_i ( &m34A, ds ); /* macro, result in-place in m34A */
\end{verbatim}
to negate a $3\times 4$ matrix
\begin{verbatim}
      gan_mat34_negate_q ( &m34A, &m34C ); /* macro */
      m34C = gan_mat34_negate_s ( &m34A ); /* function call */
      gan_mat34_negate_i ( &m34A ); /* macro, result in-place in m34A */
\end{verbatim}
and to scale a $3\times 4$ matrix to unit unit Frobenius norm
\begin{verbatim}
      gan_mat34_unit_q ( &m34A, &m34C ); /* macro */
      m34C = gan_mat34_unit_s ( &m34A ); /* function call */
      gan_mat34_unit_i ( &m34A ); /* macro, result in-place in m34A */
\end{verbatim}
The Frobenius norm of a matrix is the square-root of the sum of squares
of the matrix elements. The Gandalf functions for computing it are
described in~\ref{fixed-det-trace-norms-sec}.

Equivalent routiness to the above for multiplying/dividing a matrix by a
scalar, negating a matrix and scaling a matrix to unit Frobenius norm are
available for square fixed size matrices. Without listing the routines
exhaustively, some examples are
\begin{verbatim}
      gan_mat33_scale_q ( &m33A, ds, &m33C ); /* macro */
      sm33Sc = gan_symmat33_divide_s ( &sm33Sa, ds ); /* function call */
      gan_ltmat33_negate_i ( &sm33La ); /* macro, result in-place in sm33La */
      m33C = gan_mat33_unit_s ( &m33A ); /* function call */
\end{verbatim}

{\tt Error detection:} If zero is passed as the scalar value to the
{\tt ...\_divide\_[qi]()} routines, {\tt NULL} will be returned, while the
{\tt ...\_divide\_s()} routines will abort the program. You should add tests
for division by zero {\em before} calling any of the {\tt ...\_divide\_[qsi]()}
routines. Similarly, the {\tt ...\_unit\_[qsi]()} routines will fail if
the input matrix contains only zeros. If this is a possibility then the
program should check beforehand.

\subsubsection{Transposing a fixed size matrix}
Explicit matrix transposition is not often required in Gandalf, because of
the extensive support for implicit transpose in other matrix operations.
However where necessary, computing the transpose can be achieved using
\begin{verbatim}
      Gan_Matrix33 m33A, m33B;

      /* set up m33A using e.g. gan_mat33_fill_q() */
      gan_mat33_tpose_q ( &m33A, &m33B ); /* B = A^T */
      m33B = gan_mat33_tpose_s ( &m33A ); /* B = A^T */
      gan_mat33_tpose_i ( &m33A ); /* A = A^T, result in-place in A */
\end{verbatim}

\subsubsection{Fixed size vector outer products} \label{fixed-vector-outer-sec}
The outer product operation on two fixed size vectors $\xvec$ and $\yvec$
is the matrix
\[ A = \xvec \yvec\tr
\]
In Gandalf, any pair of vectors can be combined in this way\footnote{So long as
$\xvec$ and $\yvec$ are ordered so that the size of $\xvec$ is less than or
equal to the size of $\yvec$.}. The operation
on a 3-vector and a 4-vector produces a $3\times 4$ matrix, using either of
the routines
\begin{verbatim}
      Gan_Vector3 v3x;
      Gan_Vector4 v4y;
      Gan_Matrix34 m34A;

      /* ... set up v3x and v4y using e.g. gan_vec[34]_fill_q() ... */
      gan_vec34_outer_q ( &v3x, &v4y, &m34A ); /* macro, or */
      m34A = gan_vec34_outer_s ( &v3x, &v4y ); /* function call */
\end{verbatim}
Similar routines {\tt gan\_vec33\_outer\_[qs]()} enable the computation
of the outer product of two 3-vectors, resulting in a $3\times 3$ matrix.
If $\xvec=\yvec$, the result of the outer product is a symmetric matrix
\[ S = \xvec \xvec\tr
\]
Gandalf has special routines for this case:
\begin{verbatim}
      Gan_Vector3 v3x;
      Gan_SquMatrix33 sm33S;

      /* ... set up v3x using e.g. gan_vec3_fill_q() ... */
      gan_vec33_outer_sym_q ( &v3x, &sm33S ); /* macro, or */
      sm33S = gan_vec33_outer_sym_s ( &v3x ); /* function call */
\end{verbatim}

\subsubsection{Fixed size matrix/vector multiplication}
Gandalf supports matrix/vector multiplication with the matrix optionally
being (implicitly) transposed. If the matrix is triangular, Gandalf also
supports multiplication by the inverse of the matrix, computed implicitly
as described in the introduction to this chapter.
These operations can be written as
\begin{eqnarray}
 \yvec &\!\!=\!\!& A\xvec\;\;\;\;\mbox{OR}\;\;\;\; \yvec = A\tr\xvec\;\;\;\mbox{OR} \nonumber \\
 \yvec &\!\!=\!\!& A\inv\xvec\;\;\;\;\mbox{OR}\;\;\;\; \yvec = A\trinv\xvec\;\;\;\mbox{(triangular $A$ only)} \nonumber
\end{eqnarray}
The Gandalf routines involving a $3\times 4$ matrix for the first of these
(no transpose of $A$) is
\begin{verbatim}
      Gan_Vector4 v4x;
      Gan_Vector3 v3y;
      Gan_Matrix34 m34A;

      /* ... set up m34A and v4x ... */
      gan_mat34_multv4_q ( &m34A, &v4x, &v3y ); /* macro, or */
      v3y = gan_mat34_multv4_s ( &m34A, &v4x ); /* function call */
\end{verbatim}
while if $A$ is to be transposed then the routines are
\begin{verbatim}
      Gan_Vector3 v3x;
      Gan_Vector4 v4y;
      Gan_Matrix34 m34A;

      /* ... set up m34A and v3x ... */
      gan_mat34T_multv3_q ( &m34A, &v3x, &v4y ); /* macro, or */
      v4y = gan_mat34T_multv3_s ( &m34A, &v3x ); /* function call */
\end{verbatim}
There are similar routines {\tt gan\_mat33\_multv3\_[qs]()} and
{\tt gan\_mat33T\_multv3\_[qs]()} for $A$ being a general $3\times 3$ matrix.
For symmetric matrices, there is only one pair of routines:
\begin{verbatim}
      Gan_Vector3 v3x, v3y;
      Gan_SquMatrix33 sm33S;

      /* ... set up sm33S using e.g. gan_symmat33_fill_q(), and v3x ... */
      gan_symmat33_multv3_q ( &sm33S, &v3x, &v3y ); /* macro, or */
      v3y = gan_symmat33_multv3_s ( &sm33S, &v3x ); /* function call */
\end{verbatim}
In the case that matrix $A$ is triangular, Gandalf also supports
multiplication of vectors by the inverse of $A$. In this case the result
may be computed in-place in the input vector, So the complete set of
matrix/vector multiplication routines for triangular matrices is
\begin{verbatim}
      Gan_Vector3 v3x, v3y;
      Gan_SquMatrix33 sm33L;

      /* ... set up sm33L using e.g. gan_ltmat33_fill_q(), and v3x ... */

      /* multiply vector by lower triangular matrix */
      gan_ltmat33_multv3_q ( &sm33L, &v3x, &v3y ); /* macro, or */
      v3y = gan_ltmat33_multv3_s ( &sm33L, &v3x ); /* function call, or */
      gan_ltmat33_multv3_i ( &sm33L, &v3x ); /* macro, in-place in v3x */

      /* multiply vector by upper triangular matrix */
      gan_ltmat33T_multv3_q ( &sm33L, &v3x, &v3y ); /* macro, or */
      v3y = gan_ltmat33T_multv3_s ( &sm33L, &v3x ); /* function call, or */
      gan_ltmat33T_multv3_i ( &sm33L, &v3x ); /* macro, in-place in v3x */

      /* multiply vector by inverse of lower triangular matrix */
      gan_ltmat33I_multv3_q ( &sm33L, &v3x, &v3y ); /* macro, or */
      v3y = gan_ltmat33I_multv3_s ( &sm33L, &v3x ); /* function call, or */
      gan_ltmat33I_multv3_i ( &sm33L, &v3x ); /* macro, in-place in v3x */

      /* multiply vector by inverse of upper triangular matrix */
      gan_ltmat33IT_multv3_q ( &sm33L, &v3x, &v3y ); /* macro, or */
      v3y = gan_ltmat33IT_multv3_s ( &sm33L, &v3x ); /* function call, or */
      gan_ltmat33IT_multv3_i ( &sm33L, &v3x ); /* macro, in-place in v3x */
\end{verbatim}

{\bf Error detection:} If implicit inverse is used (the {\tt ...I\_multv...()}
or {\tt ...IT\_multv...()} routines), the matrix must
be non-singular, which for triangular matrices means that none of the
diagonal elements should be zero. If the matrix was produced
by successful Cholesky factorisation of a symmetric matrix
(see Section~\ref{fixed-decomp-sec}) the matrix is guaranteed to be
non-singular. This is the normal method of creating a triangular matrix,
and Gandalf uses {\tt assert()} to check for the singularity of the matrix.

\subsubsection{Fixed size matrix/matrix multiplication}
Most useful matrix product combinations are supported by Gandalf.
Here we describe all the combinations involving $3\times 4$ matrices.
The first functions to describe are those which involve multiplication
by a $3\times 3$ on the left or $4\times 4$ matrix on the right,
the square matrix optionally being (implicitly) transposed,
the product producing another $3\times 4$ matrix.
The operator combinations are
\[ D_{3\times 4} = B_{3\times 3} A_{3\times 4},\;\;\;
   D_{3\times 4} = B_{3\times 3}\tr A_{3\times 4},\;\;\;
   D_{3\times 4} = A_{3\times 4} C_{4\times 4},\;\;\;
   D_{3\times 4} = A_{3\times 4} C_{4\times 4}\tr
\]
which are implemented in Gandalf using the macros
\begin{verbatim}
      Gan_Matrix34 m34A, m34D;
      Gan_Matrix33 m33B;
      Gan_Matrix44 m44C;

      /* ... set up m34A, m33B and m44C ... */
      gan_mat34_lmultm33_q  ( &m34A, &m33B, &m34D ); /* D = B*A   */
      gan_mat34_lmultm33T_q ( &m34A, &m33B, &m34D ); /* D = B*A^T */
      gan_mat34_rmultm44_q  ( &m34A, &m44C, &m34D ); /* D = A*C   */
      gan_mat34_rmultm44T_q ( &m34A, &m44C, &m34D ); /* D = A*C^T */
\end{verbatim}
Equivalent function calls are available:
\begin{verbatim}
      m34D = gan_mat34_lmultm33_s  ( &m34A, &m33B ); /* D = B*A   */
      m34D = gan_mat34_lmultm33T_s ( &m34A, &m33B ); /* D = B*A^T */
      m34D = gan_mat34_rmultm44_s  ( &m34A, &m44C ); /* D = A*C   */
      m34D = gan_mat34_rmultm44T_s ( &m34A, &m44C ); /* D = A*C^T */
\end{verbatim}
Note that although by and large the functions described here for $3\times 4$
matrices are repeated for square matrices, there is redundancy because
in the case of $3\times 3$ matrices the routines
\begin{verbatim}
      m33D = gan_mat33_lmultm33_s  ( &m33A, &m33B ); /* D = B*A   */
      m33D = gan_mat33_rmultm33_s  ( &m33B, &m33A ); /* D = B*A   */
\end{verbatim}
would be equivalent, so in fact only the routines
{\tt gan\_mat33\_rmultm33\_[qs]()} are defined.

The square matrix may be symmetric or triangular, for which cases there
are specific Gandalf functions. Firstly for multiplying by symmetric matrices
we have the routines
\begin{verbatim}
      Gan_Matrix34 m34A, m34B;
      Gan_SquMatrix33 sm33S;
      Gan_SquMatrix44 sm44S;

      /* ... set up m34A, symmetric sm33S and sm44S ... */
      gan_mat34_lmults33_q ( &m34A, &sm33S, &m34B ); /* B = S*A, macro */
      gan_mat34_rmults44_q ( &m34A, &sm44S, &m34B ); /* B = A*S, macro */
\end{verbatim}
with equivalent function calls
\begin{verbatim}
      m34B = gan_mat34_lmults33_q ( &m34A, &sm33S ); /* B = S*A, function call */
      m34B = gan_mat34_rmults44_q ( &m34A, &sm44S ); /* B = A*S, function call */
\end{verbatim}
When multiplying by a triangular matrix, there are also options of implicit
transpose and inverse, as described in the introduction to this chapter.
Gandalf also supports in-place operations in this case. So there is a whole
family of functions covering multiplication of a matrix by a triangular
matrix. Mathematically the operations are
\[ B=LA, \;\;\; B=L\tr A, \;\;\; B=L\inv A, \;\;\; B=L\trinv A
\]
\[ B=AL, \;\;\; B=AL\tr, \;\;\; B=AL\inv, \;\;\; B=AL\trinv
\]
Gandalf macro routines to implement these operations are
\begin{verbatim}
      Gan_Matrix34 m34A, m34B;
      Gan_SquMatrix33 sm33L;
      Gan_SquMatrix44 sm44L;

      /* ... set up m34A, lower triangular sm33L and sm44L ... */
      gan_mat34_lmultl33_q   ( &m34A, &sm33L, &m34B ); /* B = L*A,    macro */
      gan_mat34_lmultl33T_q  ( &m34A, &sm33L, &m34B ); /* B = L^T*A,  macro */
      gan_mat34_lmultl33I_q  ( &m34A, &sm33L, &m34B ); /* B = L^-1*A, macro */
      gan_mat34_lmultl33IT_q ( &m34A, &sm33L, &m34B ); /* B = L^-T*A, macro */
      gan_mat34_rmultl44_q   ( &m34A, &sm44L, &m34B ); /* B = A*L,    macro */
      gan_mat34_rmultl44T_q  ( &m34A, &sm44L, &m34B ); /* B = A*L^T,  macro */
      gan_mat34_rmultl44I_q  ( &m34A, &sm44L, &m34B ); /* B = A*L^-1, macro */
      gan_mat34_rmultl44IT_q ( &m34A, &sm44L, &m34B ); /* B = A*L^-T, macro */
\end{verbatim}
There are also function calls to implement the same operations:
\begin{verbatim}
      m34B = gan_mat34_lmultl33_s   ( &m34A, &sm33L ); /* B = L*A,    function call */
      m34B = gan_mat34_lmultl33T_s  ( &m34A, &sm33L ); /* B = L^T*A,  function call */
      m34B = gan_mat34_lmultl33I_s  ( &m34A, &sm33L ); /* B = L^-1*A, function call */
      m34B = gan_mat34_lmultl33IT_s ( &m34A, &sm33L ); /* B = L^-T*A, function call */
      m34B = gan_mat34_rmultl44_s   ( &m34A, &sm44L ); /* B = A*L,    function call */
      m34B = gan_mat34_rmultl44T_s  ( &m34A, &sm44L ); /* B = A*L^T,  function call */
      m34B = gan_mat34_rmultl44I_s  ( &m34A, &sm44L ); /* B = A*L^-1, function call */
      m34B = gan_mat34_rmultl44IT_s ( &m34A, &sm44L ); /* B = A*L^-T, function call */
\end{verbatim}
Finally there is a set of macros for writing the result in-place into
the $3\times 4$ matrix $A$.
\begin{verbatim}
      gan_mat34_lmultl33_i   ( &m34A, &sm33L ); /* A = L*A,    macro */
      gan_mat34_lmultl33T_i  ( &m34A, &sm33L ); /* A = L^T*A,  macro */
      gan_mat34_lmultl33I_i  ( &m34A, &sm33L ); /* A = L^-1*A, macro */
      gan_mat34_lmultl33IT_i ( &m34A, &sm33L ); /* A = L^-T*A, macro */
      gan_mat34_rmultl44_i   ( &m34A, &sm44L ); /* A = A*L,    macro */
      gan_mat34_rmultl44T_i  ( &m34A, &sm44L ); /* A = A*L^T,  macro */
      gan_mat34_rmultl44I_i  ( &m34A, &sm44L ); /* A = A*L^-1, macro */
      gan_mat34_rmultl44IT_i ( &m34A, &sm44L ); /* A = A*L^-T, macro */
\end{verbatim}

The next set of functions deals with multiplying a matrix by itself in
transpose, resulting in a symmetric matrix. These operations have the form
\[ S_{4\times 4} = A_{3\times 4}\tr A_{3\times 4}\;\;\;\;\mbox{OR}\;\;\;\;
   S_{3\times 3} = A_{3\times 4} A_{3\times 4}\tr
\]
The Gandalf macro routines to implement these operations are
\begin{verbatim}
      Gan_Matrix34 m34A;
      Gan_SquMatrix33 sm33S;
      Gan_SquMatrix44 sm44S;

      /* ... set up m34A using e.g. gan_mat34_fill_q() ... */
      gan_mat34_slmultT_q ( &m34A, &sm44S ); /* S = A^T*A */
      gan_mat34_srmultT_q ( &m34A, &sm33S ); /* S = A*A^T */
\end{verbatim}
with equivalent function calls
\begin{verbatim}
      sm44S = gan_mat34_slmultT_s ( &m34A ); /* S = A^T*A */
      sm33S = gan_mat34_srmultT_s ( &m34A ); /* S = A*A^T */
\end{verbatim}
There are similar routines for general $3\times 3$ matrices. For triangular
matrices the functions are
\begin{verbatim}
      Gan_SquMatrix33 sm33L, sm33S;

      /* ... set up sm33L using e.g. gan_ltmat33_fill_q()... */
      gan_ltmat33_slmultT_q ( &sm33L, &sm33S ); /* S = L^T*L, macro */
      sm33S = gan_ltmat33_slmultT_s ( &sm33L ); /* S = L^T*L, function call */
      gan_ltmat33_srmultT_q ( &sm33L, &sm33S ); /* S = L*L^T, macro */
      sm33S = gan_ltmat33_srmultT_s ( &sm33L ); /* S = L*L^T, function call */
\end{verbatim}
In the case of triangular matrices the operation ``multiply by transposed
self'' can also be done in-place, so we also have the macro routines
\begin{verbatim}
      Gan_SquMatrix33 sm33A;

      /* ... set up sm33A as triangular using e.g. gan_ltmat33_fill_q()... */
      gan_ltmat33_slmultT_i ( &sm33A ); /* A = A^T*A, macro */
      gan_ltmat33_srmultT_i ( &sm33A ); /* A = A*A^T, macro */
\end{verbatim}

There are also routines to multiply a matrix by the transpose
of another matrix of the same size, where the result is {\em assumed} to be
a symmetric matrix. So mathematically the operations have the form
\[ S_{4\times 4} = A_{3\times 4}\tr B_{3\times 4}\;\;\;\;\mbox{OR}\;\;\;\;
   S_{3\times 3} = A_{3\times 4} B_{3\times 4}\tr
\]
The Gandalf macro routines to implement these operations are
\begin{verbatim}
      Gan_Matrix34 m34A, m34B;
      Gan_SquMatrix33 sm33S;
      Gan_SquMatrix44 sm44S;

      /* ... set up m34A, m34B using e.g. gan_mat34_fill_q() ... */
      gan_mat34_rmultm34T_sym_q ( &m34A, &m34B, &sm33S ); /* S = A*B^T */
      gan_mat34_lmultm34T_sym_q ( &m34B, &m34A, &sm44S ); /* S = A^T*B */
\end{verbatim}
with equivalent function calls
\begin{verbatim}
      sm33S = gan_mat34_rmultm34T_sym_s ( &m34A, &m34B ); /* S = A*B^T */
      sm44S = gan_mat34_lmultm34T_sym_s ( &m34B, &m34A ); /* S = A^T*B */
\end{verbatim}

A common operation is to multiply a symmetric matrix on left and right by
a matrix and its transpose, producing another symmetric matrix.
Gandalf supports all combinations of these operations.
Those involving $3\times 4$ matrices are
\begin{verbatim}
      Gan_SquMatrix33 sm33Sa;
      Gan_SquMatrix44 sm44Sb;
      Gan_Matrix34 m34A;

      gan_symmat33_lrmultm34T_q ( &sm33Sa, &m34A, &sm44Sb ); /* Sb = A^T*Sa*A, macro */
      sm44Sb = gan_symmat33_lrmultm34T_s ( &sm33Sa, &m34A ); /* Sb = A^T*Sa*A, function call */
      gan_symmat44_lrmultm34_q ( &sm44Sb, &m34A, &sm33Sa ); /* Sa = A*Sb*A^T, macro */
      sm33Sa = gan_symmat44_lrmultm34_s ( &sm44Sb, &m34A ); /* Sa = A*Sb*A^T, function call */
\end{verbatim}

{\bf Error detection:} If implicit inverse is used (e.g. the
{\tt ...multl33I\_[qsi]()} or {\tt ...multl33IT\_[qsi]()} routines),
the matrix must be non-singular, which for triangular matrices
means that none of the diagonal elements should be zero.
If the matrix was produced
by successful Cholesky factorisation of a symmetric matrix
(see Section~\ref{fixed-decomp-sec}) the matrix is guaranteed to be
non-singular. This is the normal method of creating a triangular matrix,
and Gandalf uses {\tt assert()} to check for the singularity of the matrix.

\subsubsection{Fixed size matrix inverse}
All types of square matrices can be inverted in Gandalf.
If the matrix is singular, {\tt NULL}
is normally returned and an error condition set using
{\tt gan\_err\_register()} (see Section~\ref{error-handling-sec}).
The routines to invert a $3\times 3$ matrix are
\begin{verbatim}
      Gan_Matrix33 m33A, m33B;

      /* ... set up m33A using e.g. gan_mat33_fill_q() ... */
      gan_mat33_invert_q ( &m33A, &m33B ); /* B = A^-1, OR */
      m33B = gan_mat33_invert_s ( &m33A ); /* B = A^-1, OR */
      gan_mat33_invert_i ( &m33A ); /* A = A^-1, in-place in A */
\end{verbatim}
Note that the routine {\tt gan\_mat33\_invert\_s()} returns the structure,
rather than a pointer value, so cannot return an error condition.
Do not use it unless you are SURE that your matrix is non-singular!
The basic routines {\tt gan\_mat33\_invert\_[qs]()} are implemented as
functions rather than macros, because they require temporary variables.
There are similar routines for symmetric matrices
\begin{verbatim}
      Gan_SquMatrix33 sm33Sa, sm33Sb;

      /* ... set up sm33Sa using e.g. gan_symmat33_fill_q() ... */
      gan_symmat33_invert_q ( &sm33Sa, &sm33Sb ); /* Sb = Sa^-1, OR */
      sm33Sb = gan_symmat33_invert_s ( &sm33Sa ); /* Sb = Sa^-1, OR */
      gan_symmat33_invert_i ( &sm33Sa ); /* Sa = Sa^-1, in-place in Sa */
\end{verbatim}
and for triangular matrices
\begin{verbatim}
      Gan_SquMatrix33 sm33La, sm33Lb;

      /* ... set up sm33La using e.g. gan_ltmat33_fill_q() ... */
      gan_ltmat33_invert_q ( &sm33La, &sm33Lb ); /* Lb = La^-1, OR */
      sm33Lb = gan_ltmat33_invert_s ( &sm33La ); /* Lb = La^-1, OR */
      gan_ltmat33_invert_i ( &sm33La ); /* La = La^-1, in-place in La */
\end{verbatim}

If you don't want to invoke the error package when inversion is attempted
on a singular matrix, there is a set of routines which allows to instead to
return the error condition as part of the result. For instance the code
fragment
\begin{verbatim}
      Gan_Matrix33 m33A, m33B;
      int error_code;

      /* ... set up m33A using e.g. gan_mat33_fill_q() ... */
      if ( gan_mat33_invert ( &m33A, &m33B, &error_code ) == NULL )
      {
         /* error found, act on it ... */
      }

      /* no error found */
\end{verbatim}
attempts to invert matrix $A$, and if an error is found, returns NULL,
with the error condition returned in the {\tt error\_code} variable.
For singular matrices the error condition is
{\tt GAN\_ERROR\_SINGULAR\_MATRIX}. There are similar routines
{\tt gan\_symmat33\_invert()} and {\tt gan\_ltmat33\_invert()} for
symmetric and triangular matrices respectively.

\subsubsection{Determinant, trace, norms of fixed size matrix}
\label{fixed-det-trace-norms-sec}
To compute the determinant of a square matrix use one of the routines
\begin{verbatim}
      Gan_Matrix m33A;
      Gan_SquMatrix33 sm33S, sm33L;
      double dDetA, dDetS, dDetL;

      /* set up m33A, sm33S as symmetric and sm33L as lower triangular */
      dDetA = gan_mat33_det_q(&m33A); /* macro computing det(A) */
      dDetA = gan_mat33_det_s(&m33A); /* function computing det(A) */
      dDetS = gan_symmat33_det_q(&sm33S); /* macro computing det(S) */
      dDetS = gan_symmat33_det_s(&sm33S); /* function computing det(S) */
      dDetL = gan_ltmat33_det_q(&sm33L); /* macro computing det(L) */
      dDetL = gan_ltmat33_det_s(&sm33L); /* function computing det(L) */
\end{verbatim}
The routines to compute the trace of a square matrix are similar:
\begin{verbatim}
      Gan_Matrix m33A;
      Gan_SquMatrix33 sm33S, sm33L;
      double dTraceA, dTraceS, dTraceL;

      /* set up m33A, sm33S as symmetric and sm33L as lower triangular */
      dTraceA = gan_mat33_trace_q(&m33A); /* macro computing trace(A) */
      dTraceA = gan_mat33_trace_s(&m33A); /* function computing trace(A) */
      dTraceS = gan_symmat33_trace_q(&sm33S); /* macro computing trace(S) */
      dTraceS = gan_symmat33_trace_s(&sm33S); /* function computing trace(S) */
      dTraceL = gan_ltmat33_trace_q(&sm33L); /* macro computing trace(L) */
      dTraceL = gan_ltmat33_trace_s(&sm33L); /* function computing trace(L) */
\end{verbatim}
For all types of fixed size matrix there are also routines to compute the
sum of squares of the matrix elements, as well as the Frobenius norm,
which is the square-root of the sum of squares. For $3\times 4$ matrices
the routines are
\begin{verbatim}
      Gan_Matrix34 m34A;
      double dSumSA, dFNormA;

      /* ... set up m34A using e.g. gan_mat34_fill_q() ... */
      dSumSA = gan_mat34_sumsqr_q(&m34A); /* macro computing sum(A_ij^2) */
      dSumSA = gan_mat34_sumsqr_s(&m34A); /* function computing sum(A_ij^2) */
      dFNormA = gan_mat34_Fnorm_q(&m34A); /* macro computing sqrt(sum(A_ij^2)) */
      dFNormA = gan_mat34_Fnorm_s(&m34A); /* function computing sqrt(sum(A_ij^2)) */
\end{verbatim}
There are equivalent routines for other types of matrix.

\subsubsection{Fixed size matrix decompositions} \label{fixed-decomp-sec}
Gandalf supports several of the standard matrix decompositions. Cholesky
factorisation applies to any positive definite symmetric matrix $S$, producing
the lower triangular matrix $L$ so that
\[ S = L L\tr
\]
It can be computed for $3\times 3$ symmetric matrices using the routines
\begin{verbatim}
      Gan_SquMatrix33 sm33S, sm33L;

      /* ... set up sm33S using e.g. gan_symmat33_fill_q() ... */
      gan_symmat33_cholesky_q ( &sm33S, &sm33L ); /* L = chol(S) */
      sm33L = gan_symmat33_cholesky_s ( &sm33S ); /* L = chol(S) */
\end{verbatim}
There is also a routine for computing the Cholesky factorisation in-place
in the input matrix, converting an input symmetric matrix $A$ into a lower
triangular matrix:
\begin{verbatim}
      Gan_SquMatrix33 sm33A;

      /* ... set up sm33A as symmetric using e.g. gan_symmat33_fill_q() ... */
      gan_symmat33_cholesky_i ( &sm33A ); /* A = chol(A) */
\end{verbatim}
The routines {\tt gan\_symmat33\_cholesky\_[qi]()} return {\tt NULL} and
invoke the Gandalf error handler\\ {\tt gan\_err\_register()} if the matrix
is not positive definite ({\tt gan\_symmat33\_cholesky\_s()} aborts the
program on error, so don't use it unless you're SURE your matrix is OK!).
If you don't want to invoke the error package when factorisation is attempted
on a non-positive-definite matrix, there is a set of routines which allows
to instead to return the error condition as part of the result.
For instance the code fragment
\begin{verbatim}
      Gan_SquMatrix33 sm33S, sm33L;
      int error_code;

      /* ... set up sm33S using e.g. gan_symmat33_fill_q() ... */
      if ( gan_symmat33_choleksy ( &sm33S, &sm33L, &error_code ) == NULL )
      {
         /* error found, act on it ... */
      }

      /* no error found */
\end{verbatim}
attempts to factorise matrix $S$, and if an error is found, returns NULL,
with the error condition returned in the {\tt error\_code} variable.
For non-positive-definite matrices the error condition is
{\tt GAN\_ERROR\_NOT\_POSITIVE\_DEFINITE}.

Other factorisations are available in Gandalf. Singular value decomposition
(SVD) can be used to decompose almost any matrix $A$ into factors as
\[ A = UWV\tr
\]
where $U$ and $V$ are orthogonal matrices and $W$ is diagonal.
Currently Gandalf supports SVD for $3\times 3$ and $4\times 4$ matrices.
To use the functions for $3\times 3$ matrices, include the header file
\begin{verbatim}
      #include <gandalf/linalg/3x3matrix_svd.h>
\end{verbatim}
There are routines for SVD of a matrix or its transpose, as follows
\begin{verbatim}
      Gan_Matrix33 m33A, m33U, m33VT;
      Gan_Vector3 v3W;

      /* ... set up m33A using e.g. gan_mat33_fill_q() ... */
      gan_mat33_svd ( &m33A, &m33U, &v3W, &m33VT ); /* A = U*W*V^T, OR */
      gan_mat33T_svd ( &m33A, &m33U, &v3W, &m33VT ); /* A^T = U*W*V^T */
\end{verbatim}
These routines return a {\tt Gan\_Bool} result, which is {\tt GAN\_TRUE}
on success and {\tt GAN\_FALSE} on failure.

There are also a number of routines for computing the eigenvalues and
eigenvectors of fixed size matrices. For $3\times 3$ matrices only there
is a routine to compute the eigenvectors and complex eigenvalues of
a $3\times 3$ matrix. To use the routine include the header file
\begin{verbatim}
      #include <gandalf/linalg/3x3matrix_eigen.h>
\end{verbatim}
The matrix $A$ has ``left'' and ``right'' eigenvectors associated with the
same eigenvalues $\lambda_i$, so that the equation
\[ A \vvec_i = \lambda_i \vvec_i
\]
defines the right eigenvectors, and
\[ \uvec_i^H A = \lambda_i \uvec_i^H
\]
defines the left eigenvectors, where $\uvec_i^H$ denotes the conjugate
transpose of vector $\uvec_i$.
The computed eigenvectors are normalized to have Euclidean norm   
equal to one and largest component real. The Gandalf routine that implements
this is built on the equivalent CLAPACK routine {\tt dgeev()}:
\begin{verbatim}
      Gan_Matrix33 m33A; /* matrix to be decomposed */
      Gan_Matrix33 m33UL, m33VR; /* left and right eigenvectors */
      Gan_Vector3 v3lr, v3li; /* real and imaginary parts of eigenvalues */

      /* ... set up m33A using e.g. gan_mat33_fill_q() ... */
      gan_mat33_eigen ( &m33A, &v3lr, &v3li, &m33UL, &m33VR );
\end{verbatim}

The eigenvalues of symmetric matrices are guaranteed to be real.
Routines are available for computing the eigenvalues and eigenvectors of
$3\times 3$ and $4\times 4$ symmetric matrices, based on either the
CLAPACK routine {\tt dspev()} or the CCMath routine {\tt eigval()}.
For $3\times 3$ matrices the routine is declared in the header file
\begin{verbatim}
      #include <gandalf/linalg/3x3matrix_eigsym.h>
\end{verbatim}
Here is an example using the routine
\begin{verbatim}
      Gan_SquMatrix33 sm33S; /* symmetric matrix to be decomposed */
      Gan_Matrix33 m33Z; /* (right) eigenvectors of A */
      Gan_Vector3 v3W; /* eigenvalues */

      /* ... set up sm33S using e.g. gan_symmat33_fill_q() ... */
      gan_symmat33_eigen ( &sm33S, &v3W, &m33Z );
\end{verbatim}

\subsubsection{Fixed size matrix file I/O}
Gandalf supports both ASCII and binary format file I/O of matrices.
Both formats use standard {\tt FILE~*} file streams.
ASCII format is obviously more convenient to use, while binary format is
more compact and guarantees no loss of precision when the data is read.
To print a $3\times 4$ matrix in ASCII format, use
\begin{verbatim}
      Gan_Bool gan_mat34_fprint ( FILE *fp, Gan_Matrix34 *p,
                                  const char *prefix, int indent, const char *fmt );
\end{verbatim}
{\tt prefix} is a prefix string to print before the matrix itself, {\tt indent}
is the number of spaces to indent the matrix by, and {\tt fmt} is a format
string to use when printing the matrix, e.g. {\tt "\%f"}. So for example
\begin{verbatim}
      FILE *pfFile;

      pfFile = fopen ( "/tmp/matrices", "w" );
      gan_mat34_fill_q ( &m34A, 1.0,  2.0,  3.0,  4.0,
                                5.0,  6.0,  7.0,  8.0,
                                9.0, 10.0, 11.0, 12.0 );
      gan_mat34_fprint ( pfFile, &m34A, "Example matrix", 3, "%f" );
\end{verbatim}
will print the output
\begin{verbatim}
         Example matrix
          1.000000 2.000000 3.000000 4.000000
          5.000000 6.000000 7.000000 8.000000
          9.000000 10.000000 11.000000 12.000000
\end{verbatim}
to the file {\tt "/tmp/matrices"}.
There is also a version {\tt gan\_mat34\_print()} for printing to
standard output:
\begin{verbatim}
      Gan_Bool gan_mat34_print ( Gan_Matrix34 *p,
                                 const char *prefix, int indent, const char *fmt );
\end{verbatim}
The corresponding input function is
\begin{verbatim}
      Gan_Bool gan_mat34_fscanf ( FILE *fp, Gan_Matrix34 *p,
                                  char *prefix, int prefix_len )
\end{verbatim}
which reads the matrix from the file stream {\tt fp} into the $3\times 4$
matrix pointer {\tt p}. It also reads the prefix string (up to the specified
maximum length {\tt prefix\_len}), which can be compared with the expected
prefix string to check for consistency.

Binary file I/O is handled by the functions {\tt gan\_mat34\_fwrite()}
and {\tt gan\_mat34\_fread()}. To write a $3\times 4$ matrix in binary
format use
\begin{verbatim}
      Gan_Bool gan_mat34_fwrite ( FILE *fp, Gan_Matrix34 *p, gan_ui32 magic_number );
\end{verbatim}
The {\tt magic\_number} takes the same role as the {\tt prefix} string
in {\tt gan\_mat34\_fprint()}, and is written into the file so that it can be
used later to identify the matrix when it is read back using
\begin{verbatim}
      Gan_Bool gan_mat34_fread ( FILE *fp, Gan_Matrix34 *p, gan_ui32 *magic_number );
\end{verbatim}

There are similar functions {\tt gan\_mat33\_fprint()},
{\tt gan\_mat33\_print()}, {\tt gan\_mat33\_fscanf()} for ASCII I/O of
$3\times 3$ matrices, and {\tt gan\_mat33\_fwrite()}, {\tt gan\_mat33\_fread()}
for binary I/O of $3\times 3$ matrices. Functions for symmetric and
triangular matrices follow the same pattern.

{\tt Error detection:} The I/O routines return a boolean value, returning
{\tt GAN\_TRUE} on success, {\tt GAN\_FALSE} on failure, invoking the Gandalf
error handle in the latter case.

\subsubsection{Conversion from general to fixed size matrix}
Functions are provided to convert a general matrix {\tt Gan\_Matrix}
{\tt Gan\_Matrix34}, provided that the general matrix has actually been
created with size three. So for instance
\begin{verbatim}
      Gan_Matrix *pmMatrix;

      pmMatrix = gan_mat_alloc(3,4);
      gan_mat_fill_va ( pmMatrix, 3, 4, 1.0,  2.0,  3.0,  4.0,
                                        5.0,  6.0,  7.0,  8.0,
                                        9.0, 10.0, 11.0, 12.0 );
      gan_mat34_from_mat_q ( pmMatrix, &m34A );
\end{verbatim}
or
\begin{verbatim}
      m34A = gan_mat34_from_mat_s ( pmMatrix );
\end{verbatim}
fill {\tt m34A} with the same values as the general matrix.
Calling these functions with a general matrix {\tt pmMatrix} not having
the same size as the fixed size matrix is an error.

{\tt Error detection:} The conversion routines return the pointer to the
filled fixed size matrix, or {\tt NULL} on failure, invoking the Gandalf
error handle in the latter case.

\subsubsection{Single precision fixed size matrices}
\begin{verbatim}
      #include <gandalf/linalg/3matrixf.h>
\end{verbatim}
There is an identical set of functions for handling single precision
floating point matrices, the names of which are obtained by replacing
{\tt "gan\_mat34\_..."} in the above functions with {\tt "gan\_mat34f\_..."}.
For example, to add two single precision $3\times 4$ matrices use
\begin{verbatim}
      Gan_Matrix34_f m34A, m34B, m34C;

      /* ... fill m34A and m34B ... */
      gan_mat34f_add_q ( &m34A, &m34B, &m34C );
\end{verbatim}

\section{General size matrices and vectors} \label{general-size-sec}
When the vector/matrix object to be represented can have variable size,
or number of rows/columns greater than four, Gandalf provides the
structures and functions through the general size matrix/vector package.
With this package similar operations to the fixed size package are
supported. The general size package has addition features designed
to ease the burden of the programmer, while still maintaining efficient
run-time operation.

General size square matrices are handled in a subtly different way
to their fixed-size equivalents. For fixed size symmetric and triangular
matrices, there are specific functions dealing with each type of square
matrix, currently the two types, symmetric and lower triangular.
A fixed size square matrix does not ``know'' what type of matrix it is.
The {\tt type} field in this case is an inessential field of the matrix,
and is indeed {\tt \#ifdef}'d out when Gandalf is compiled with {\tt NDEBUG}
set. The only reason for using the same structure for both symmetric
and lower triangular matrices is to allow the in-place operations that convert
a symmetric matrix to a triangular matrix or vice versa. This arrangement
is optimal for speed, because no type checking needs to be done at
run time.

In contrast, the general size square matrix structure has a {\tt type} field
that is meaningful as the current matrix type. Many Gandalf functions are
written for general square matrices. This simplifies the programming interface,
in that one function can be used to implement an operation (e.g. square
matrix add) for every type of square matrix, at the expense of some loss
of speed, since the general function has to call different subroutines
depending on the type of matrix. The overhead in implementing this
arrangement is reduced to the minimum by including the routines to
implement each operation for a given matrix type in the matrix structure
itself. This object-oriented design feature is hidden from the
programmer through the use of macros, so the package appears to the
programmer as a normal set of functions.

Another difference between fixed and general vectors and matrices is that
whereas in general fixed size vectors \& matrices should be declared as
structures, avoiding the need for dynamic allocation,
general size vectors \& matrices require dynamic allocation of their
internal data. There is still an advantage in declaring structures rather
than structure pointers, in that with pointer variables you require a
call to {\tt malloc()} to create the structure they point to.
Gandalf lets the programmer decide which style to use. In the following
description examples of both styles are presented.

Once again, both double precision and single precision routines are available,
and once again a parallel set of header files, structures and functions is
provided. We shall concentrate on the double precision package, and the
equivalent structure and function name conversions for single precision are
given in Section~\ref{single-gen-sec}.

Most of the routines return a pointer to the matrix/vector result structure.
{\tt NULL} is returned on failure, and the Gandalf error handler is invoked.
This is reiterated in the text below, and exceptions are noted.

\subsection{General size vectors} \label{general-size-vec-sec}
The structure and functions for general size vectors are declared in the
header file
\begin{verbatim}
      #include <gandalf/linalg/vec_gen.h>
\end{verbatim}
The structure for general size vectors is the {\tt Gan\_Vector}.

\subsubsection{Creating and freeing general size vectors}
To create a general size vector use one of the routines
\begin{verbatim}
      Gan_Vector *pvx;

      pvx = gan_vec_alloc(5);
\end{verbatim}
or
\begin{verbatim}
      Gan_Vector vx;

      gan_vec_form ( &vx, 5 );
\end{verbatim}
Both these examples create a vector with five elements. The former allocates
a structure and passes back a pointer to it, whereas the latter builds the
vector using the provided structure {\tt vx}. Whichever routine is used,
the two vectors are equivalent in every way and can be used in all the
Gandalf general size vector routines.

In the above calls Gandalf will invoke {\tt malloc()} to create the data
block to hold the vector elements. Sometimes you will want to provide the
data block yourself, avoiding the {\tt malloc()} call, if you know the size,
or at least the maximum size, of the vector. Then you can use the
following routine.
\begin{verbatim}
      Gan_Vector vx;
      double adXData[10];

      gan_vec_form_data ( &vx, 5, adXData, 10 );
\end{verbatim}
The last argument is the size of the array {\tt adXData} passed in.
This means that although the vector {\tt vx} is created with size five,
the size of the data block, 10, is also stored, and this allows the
size of {\tt vx} to change (see {\tt gan\_vec\_set\_size()} below) up to
size 10.

Once you have finished with a vector use the routine
\begin{verbatim}
      gan_vec_free ( pvx ); /* for a pointer variable, OR */
      gan_vec_free ( &vx ); /* for a structure variable */
\end{verbatim}
The {\tt gan\_vec\_free()} routine applies without modification to all
the methods of creating the vector. The vector structure maintains
knowledge of which parts of it (the structure, the data block) were
dynamically allocated, and only frees the bits that were allocated.

From now on the example code fragments we provide will use the convention
that vectors are declared as structures rather than pointers, but bear in
mind that either style may be used.

{\bf Error detection:} All the above vector creation routines return a
pointer to the created vector.
If an error occurs, the Gandalf error handler is invoked and {\tt NULL}
is returned. The most likely error modes are failing
to allocate the data required (i.e. internal {\tt malloc()} or {\tt realloc()}
calls failing), or passing too small an array into the
{\tt gan\_vec\_form\_data()} routine.

\subsubsection{Adjusting the size of a general size vector}
\label{set-size-vec-sec}
Once a vector has been created, its size may be adjusted dynamically
as needs arise. Gandalf stores the currently allocated maximum size of
a vector in the vector structure, so it can determine when the size of
the result of a computation will exceed the current size, and reallocate
accordingly. This happens automatically when a vector is the result
of a calculation, but sometimes it is necessary to explicitly set the
size of a vector. This is done using the following routine.
\begin{verbatim}
      gan_vec_set_size ( &vx, 3 );
\end{verbatim}
This resets the size of the vector {\tt vx} to 3.
If the size of a vector created
by {\tt gan\_vec\_alloc()} or {\tt gan\_vec\_form()} is increased in size
in this way beyond its originally allocated size, {\tt gan\_vec\_set\_size()}
will automatically reallocate the vector to the new size.
On the other hand, if {\tt gan\_vec\_form\_data()} was used to create the
vector, it cannot be increased in size beyond the size of the array
passed as the last argument into {\tt gan\_vec\_form\_data()}.

{\bf Error detection:} {\tt NULL} is returned and the error handler is
invoked on failure. The most likely failure mode is failing to reallocate
the vector data, i.e. failure of a call to {\tt realloc()}.

\subsubsection{Filling a general size vector with values}
   \label{fill-vec-gen-sec}
To fill a vector with values, create the vector and then use the routine
{\tt gan\_vec\_fill\_va()}. An example is
\begin{verbatim}
      gan_vec_fill_va ( &vx, 6, 1.0, 2.0, 3.0, 4.0, 5.0, 6.0 );
\end{verbatim}
which sets vector {\tt vx} to have size 6, and sets the six elements to
the values one to six.

To fill a vector with a constant value, you can use
\begin{verbatim}
      gan_vec_fill_const_q ( &vx, 4, 3.0 );
\end{verbatim}
which sets the size of {\tt vx} to four elements, all of which are set to
three. This function is also available in a form which allocates and fills a
vector from scratch:
\begin{verbatim}
      Gan_Vector *pvx;

      pvx = gan_vec_fill_const_s ( 4, 3.0 );
\end{verbatim}
There are special macro routines for setting a vector to zero:
\begin{verbatim}
      gan_vec_fill_zero_q ( &vx, 4 ); /* OR */
      pvx = gan_vec_fill_zero_s(4);
\end{verbatim}

{\bf Error detection:} {\tt NULL} is returned and the error handler is
invoked on failure. The most likely failure mode is failing to reallocate
the vector data when the size of the vector is changed,
i.e. failure of a call to {\tt realloc()}.

\subsubsection{Accessing the elements of a general size vector}
To read the value of a specific element of a vector use
\begin{verbatim}
      double dEl;

      dEl = gan_vec_get_el ( pvx, 1 ); /* returns x[1], x = (x[0] x[1] ... )^T */
\end{verbatim}
This sets {\tt dEl} to the second element of vector {\tt pvx}
(zero is the first). To set an element to a specific value use
\begin{verbatim}
      gan_vec_set_el ( pvx, 2, 3.0 ); /* set x[2] to 3, x = (x[0] x[1] ... )^T */
\end{verbatim}
This sets the third element of vector {\tt pvx} to 3.
There are also routines to increment or decrement an element of a vector:
\begin{verbatim}
      gan_vec_inc_el ( pvx, 0, 2.5 ); /* x_0 += 2.5, x = (x[0] x[1] ... )^T */
      gan_vec_dec_el ( pvx, 4, 5.0 ); /* x_4 -= 5.0, x = (x[0] x[1] ... )^T */
\end{verbatim}
which respectively increment the first element by 2.5 and subtract 5 from
the fifth element of {\tt pvx}.

{\bf Error detection:} {\tt gan\_vec\_set\_el()}, {\tt gan\_vec\_inc\_el()}
and {\tt gan\_vec\_dec\_el()} all return boolean values, with {\tt GAN\_FALSE}
returned on failure, in which case the Gandalf error handler is invoked.
The most likely failure mode is accessing an element outside the bounds of
the vector. If {\tt NDEBUG} is set then no error checking is done.
{\tt gan\_vec\_get\_el()} operates similarly, but returns
{\tt DBL\_MAX} on error.

\subsubsection{Copying a general size vector}
To copy a vector $\xvec$ to another vector $\yvec$, both vectors must have
been created, and $\xvec$ should be filled with values. $\yvec$ can be
created with arbitrary initial size (for instance zero), since Gandalf will
if necessary reallocate $\yvec$ to the same size as $\xvec$ if necessary.
So for instance the following code is perfectly valid:
\begin{verbatim}
      Gan_Vector vx, vy; /* declare vectors x & y */

      gan_vec_form ( &vx, 0 ); /* create vector x */
      gan_vec_form ( &vy, 0 ); /* create vector y */
      gan_vec_fill_va ( &vx, 5, 11.0, 9.0, 7.0, 5.0, 3.0 ); /* reallocate & initialise x */
      gan_vec_copy_q ( &vx, &vy ); /* set y = x, reallocating y */
\end{verbatim}
The last two lines reallocate first $\xvec$ and then $\yvec$, because both
were created with zero size. Note that $\yvec$ may have previously been
filled with other values, which are now lost.

There is also a version that creates a copy of a vector from scratch:
\begin{verbatim}
      Gan_Vector *pvy; /* declare vector y */

      pvy = gan_vec_copy_s ( &vx ); /* create y and set y = x */
\end{verbatim}

{\bf Error detection:} The vector copy routines return {\tt NULL} and
invoke the Gandalf error handler upon failure.

\subsubsection{General size vector addition}
To add two vectors $\xvec$ and $\yvec$ together, obtaining the sum
$\zvec = \xvec + \yvec$, use the routine
\begin{verbatim}
      Gan_Vector vx, vy, vz; /* declare vectors x, y and z */

      /* ... create and fill vx & vy, create vz ... */
      gan_vec_add_q ( &vx, &vy, &vz ); /* compute z = x + y */
\end{verbatim}
Again vector $\zvec$ is reallocated if necessary. Vectors $\xvec$ and
$\yvec$ must of course be the same size, or the error handler is invoked
and {\tt NULL} is returned. The sum vector $\zvec$ may be create from scratch
using
\begin{verbatim}
      Gan_Vector *pvz; /* declare vector z as pointer */

      /* ... create and fill vx & vy ... */
      pvz = gan_vec_add_s ( &vx, &vy ); /* compute z = x + y */
\end{verbatim}
Another way of computing vector addition is to replace one of the input
vector $\xvec$ or $\yvec$ with the result, using one of the in-place routines
\begin{verbatim}
      gan_vec_add_i1 ( &vx, &vy ); /* replace x = x + y */
      gan_vec_add_i2 ( &vx, &vy ); /* replace y = x + y */
\end{verbatim}
An alternative to {\tt gan\_vec\_add\_i1()} is the more explicit routine
\begin{verbatim}
      gan_vec_increment ( &vx, &vy ); /* replace x = x + y */
\end{verbatim}

{\bf Error detection:} {\tt NULL} is returned and the Gandalf error handler
invoked if the vector addition fails.
The most likely failure modes are failing to create/set the result vector,
or size incompatibility between the input vectors.

\subsubsection{General size vector subtraction}
The routines for vector subtraction follow the scheme of those for
vector addition, leading to the options
\begin{verbatim}
      Gan_Vector vx, vy, vz; /* declare vectors x, y and z */
      Gan_Vector *pvz; /* declare vector z alternatively as pointer */

      /* ... create and fill vx & vy, create vz ... */
      gan_vec_sub_q ( &vx, &vy, &vz ); /* compute z = x - y */
      pvz = gan_vec_sub_s ( &vx, &vy ); /* compute z = x - y */
      gan_vec_sub_i1 ( &vx, &vy ); /* replace x = x - y */
      gan_vec_sub_i2 ( &vx, &vy ); /* replace y = x - y */
      gan_vec_decrement ( &vx, &vy ); /* replace x = x - y */
\end{verbatim}

{\bf Error detection:} {\tt NULL} is returned and the Gandalf error handler
invoked if the vector subtraction fails.
The most likely failure modes are failing to create/set the result vector,
or size incompatibility between the input vectors.

\subsubsection{Rescaling a general size vector}
Multiplying or dividing a vector by a scalar value follows the scheme of the
above copy, addition and subtraction operations. To multiply a vector $\xvec$
by a scalar $s$, $\yvec=s\xvec$, use for example
\begin{verbatim}
      Gan_Vector vx, vy; /* declare vectors x & y */

      /* ... create & fill vx, create (& optionally fill) vy ... */
      gan_vec_scale_q ( &vx, 5.0, &vy ); /* y = 5*x */
\end{verbatim}
to multiply all the elements in vector $\xvec$ by five, writing the result
into vector $\yvec$.
Alternatively you can create the rescaled vector from scratch as in
\begin{verbatim}
      Gan_Vector *pvy; /* declare vector y */

      /* ... create & fill vx ... */
      pvy = gan_vec_scale_s ( &vx, 5.0 ); /* y = 5*x */
\end{verbatim}
or overwrite $\xvec$ with the result
\begin{verbatim}
      gan_vec_scale_i ( &vx, 5.0 ); /* replace x = 5*x */
\end{verbatim}

There are similar routines for dividing a general size vector by a scalar
value:
\begin{verbatim}
      gan_vec_divide_q ( &vx, 5.0, &vy ); /* y = x/5 */
      pvy = gan_vec_divide_s ( &vx, 5.0 ); /* y = x/5 */
      gan_vec_divide_i ( &vx, 5.0 ); /* replace x = x/5 */
\end{verbatim}
Passing zero as the scalar value in this case invokes the error handler,
with a division by zero error
(error code {\tt GAN\_ERROR\_DIVISION\_BY\_ZERO}),
and {\tt NULL} is returned.

There are specific routines to negate a vector, i.e. multiply it by -1,
as follows:
\begin{verbatim}
      gan_vec_negate_q ( &vx, &vy ); /* y = -x */
      pvy = gan_vec_negate_s ( &vx ); /* y = -x */
      gan_vec_negate_i ( &vx ); /* replace x = -x */
\end{verbatim}

{\tt Error detection:} The Gandalf error handler is invoked and {\tt NULL}
is returned if an error occurs. The most likely failure modes are
(i) failing to create the result vector; (ii) division by zero.

\subsection{General size matrices} \label{general-size-mat-sec}
The structure and functions for general size matrices are declared in the
header file
\begin{verbatim}
      #include <gandalf/linalg/mat_gen.h>
\end{verbatim}
The structure for general size matrices is the {\tt Gan\_Matrix}.
For special types of square matrix the structure is the {\tt Gan\_SquMatrix},
and is declared in the header file
\begin{verbatim}
      #include <gandalf/linalg/mat_square.h>
\end{verbatim}
The square matrix types are listed in {\tt linalg\_defs.h} (file not to be
included explicitly in application programs):
\begin{verbatim}
      /* types of square matrix */
      typedef enum { GAN_SYMMETRIC_MATRIX,     /* symmetric */
                     GAN_DIAGONAL_MATRIX,      /* diagonal */
                     GAN_SCALED_IDENT_MATRIX,  /* identity times scalar */
                     GAN_LOWER_TRI_MATRIX,     /* lower triangular */
                     GAN_UPPER_TRI_MATRIX,     /* upper triangular */
                     GAN_ZERO_SQUARE_MATRIX }  /* square matrix filled with zeros */
        Gan_SquMatrixType;
\end{verbatim}
Use of the special matrix types producs savings both in memory and computation
time, and should be exploited wherever appropriate.
To use any functions specific to the above square matrix types
(as opposed to general square matrix functions common to many types), you will
need to include the header file for the specific type you need, for instance
\begin{verbatim}
      #include <gandalf/linalg/mat_symmetric.h>
\end{verbatim}
for symmetric matrices, and
\begin{verbatim}
      #include <gandalf/linalg/mat_triangular.h>
\end{verbatim}
for triangular matrices (covers lower and upper triangular matrices).
The full list of header files for the specific square matrix types is
\begin{verbatim}
      #include <gandalf/linalg/mat_symmetric.h>  /* for symmetric matrices */
      #include <gandalf/linalg/mat_triangular.h> /* lower/upper triangular matrices */
      #include <gandalf/linalg/mat_diagonal.h>   /* diagonal matrices */
      #include <gandalf/linalg/mat_scaledI.h>    /* scaled identity matrices */
\end{verbatim}
The square matrix routines are organised so that once the matrix has been
set at a specific type and size, and filled with values, the functions that
operate on it are general functions. The computational overhead is that
whenever an operation is performed (such at matrix/matrix multiplication)
there is an indirection to take the program into the correct routine for
the current square matrix type. This design allows the programmer to easily
design algorithms that will work correctly and efficiently
for any square matrix type.

\subsubsection{Creating and freeing general size matrices}
To create a general size rectangular matrix use one of the routines
\begin{verbatim}
      Gan_Matrix *pmA;

      pmA = gan_mat_alloc ( 3, 5 );
\end{verbatim}
or
\begin{verbatim}
      Gan_Matrix mA;

      gan_mat_form ( &mA, 3, 5 );
\end{verbatim}
Both these examples create a matrix with three rows and five columns.
The former allocates
a structure and passes back a pointer to it, whereas the latter builds the
matrix using the provided structure {\tt mA}. Whichever routine is used,
the two matrices are equivalent in every way and can be used in all the
Gandalf general size matrix routines.

In the above calls Gandalf will invoke {\tt malloc()} to create the data
block to hold the matrix elements. Sometimes you will want to provide the
data block yourself, avoiding the {\tt malloc()} call, if you know the size,
or at least the maximum size, of the matrix. Then you can use the
following routine.
\begin{verbatim}
      Gan_Matrix mA;
      double adAData[30];

      gan_mat_form_data ( &mA, 3, 5, adXData, 10 );
\end{verbatim}
The last argument is the size of the array {\tt adXData} passed in.
This means that although the matrix {\tt mA} is created with size five,
the size of the data block, 10, is also stored, and this allows the
size of {\tt mA} to change (see {\tt gan\_mat\_set\_size()} below) up to
size 10.

Once you have finished with a matrix, use the routine
\begin{verbatim}
      gan_mat_free ( pmA ); /* for a pointer variable, OR */
      gan_mat_free ( &mA ); /* for a structure variable */
\end{verbatim}
The {\tt gan\_mat\_free()} routine applies without modification to all
the methods of creating the matrix. The matrix structure maintains
knowledge of which parts of it (the structure, the data block) were
dynamically allocated, and only frees the bits that were allocated.
To free several rectangular matrices at once use the variable argument
list routine
\begin{verbatim}
      gan_mat_free_va ( pmA, pmB, pmC, NULL ); /* free matrices A, B & C */
\end{verbatim}
which must be terminated by {\tt NULL} to indicate the end of the list.

From now on, we use the convention that a ``square'' matrix refers to a
square matrix with one of the special types listed above. A square general
matrix falls into the category of rectangular matrices, for the purpose of
the Gandalf linear algebra package.
To create a square matrix with one of the special types use a function
from one of the families
\begin{verbatim}
      Gan_SquMatrix *psmS, *psmL, *psmU, *psmD, *psmsI;

      psmS = gan_symmat_alloc ( 3 ); /* create a 3x3 symmetric matrix */
      psmL = gan_ltmat_alloc ( 3 ); /* create a 3x3 lower triangular matrix */
      psmU = gan_utmat_alloc ( 3 ); /* create a 3x3 upper triangular matrix */
      psmD = gan_diagmat_alloc ( 3 ); /* create a 3x3 diagonal matrix */
      psmD = gan_scalImat_alloc ( 3 ); /* create a 3x3 scaled Identity matrix */
\end{verbatim}
or
\begin{verbatim}
      Gan_SquMatrix smS, smL, smU, smD, smsI;

      gan_symmat_form ( &smS, 3 ); /* create a 3x3 symmetric matrix */
      gan_ltmat_form ( &smL, 3 ); /* create a 3x3 lower triangular matrix */
      gan_utmat_form ( &smU, 3 ); /* create a 3x3 upper triangular matrix */
      gan_diagmat_form ( &smD, 3 ); /* create a 3x3 diagonal matrix */
      gan_scalImat_form ( &smsI, 3 ); /* create a 3x3 scaled Identity matrix */
\end{verbatim}
There are also {\tt ...\_form\_data()} functions available for the case that
an array to hold the matrix data is already available. The size of the data
array depends on the type of matrix; for instance a $4\times 4$ symmetric
and triangular matrix has ten independent elements, while a $4\times 4$
diagonal matrix has only four, and a scaled identity matrix only one.
The general formula for the number of independent elements in a triangular
or symmetric matrix is
\begin{equation}
 \mbox{\# independent elements} = \frac{n(n+1)}{2} \label{symtri-els}
\end{equation}
for an $n\times n$ matrix, compared with $n$ for a diagonal matrix and $1$ for
a scaled identity matrix. So these are appropriate function calls.
\begin{verbatim}
      Gan_SquMatrix smS, smL, smU, smD, smsI;
      double adSdata[10], adLdata[10], adUdata[10], adDdata[4], dsIdata;

      gan_symmat_form_data ( &smS, 4, adSdata, 10 ); /* create 4x4 symmetric matrix */
      gan_ltmat_form_data ( &smL, 4, adLdata, 10 ); /* create 4x4 lower triangular matrix */
      gan_utmat_form_data ( &smU, 4, adUdata, 10 ); /* create 4x4 upper triangular matrix */
      gan_diagmat_form_data ( &smD, 4, adDdata, 4 ); /* create 4x4 diagonal matrix */
      gan_scalImat_form_data ( &smsI, 4, &dsIdata, 1 ); /* create 4x4 scaled Ident. mat. */
\end{verbatim}
The final way of creating a square matrix should be used only when the
matrix type is a variable:
\begin{verbatim}
      Gan_SquMatrixType type;
      Gan_SquMatrix *psmA, smA;
      double adAdata[10];

      /* ... set matrix type e.g. GAN_SYMMETRIC_MATRIX ... */
      psmA = gan_squmat_alloc ( type, 4 ); /* create 4x4 square matrix, pointer version, OR */
      gan_squmat_form ( &smA, type, 4 ); /* create 4x4 square matrix, structure version, OR */
      gan_squmat_form_data ( &smA, type, 4, adAdata, 10 ); /* create 4x4 square matrix */
\end{verbatim}
These routines call the lower level routine specific to the provided
{\tt type}. Whichever type of matrix is created, use the function
\begin{verbatim}
      gan_squmat_free ( psmA );
\end{verbatim}
to free the memory associated with it.
The {\tt gan\_squmat\_free()} routine applies to all the square matrix types
and all methods ({\tt ...\_alloc()}, {\tt ...\_form()} and
{\tt ...\_form\_data()}) of creating the matrix. The matrix structure maintains
knowledge of which parts of it (the structure, the data block) were
dynamically allocated, and only frees the bits that were allocated.
To free several square matrices at once use the variable argument
list routine
\begin{verbatim}
      gan_squmat_free_va ( &smsI, &smD, &smU, &smL, &smS, NULL ); /* free square matrices */
\end{verbatim}
which must be terminated by {\tt NULL} to indicate the end of the list.

From now on the example code fragments we provide will use the convention
that matrices are declared as structures rather than pointers, but bear in
mind that either style may be used.

{\bf Error detection:} All the above matrix creation routines
return a pointer to the created matrix. If an error occurs,
the Gandalf error handler is invoked and {\tt NULL}
is returned. The most likely error modes are failing
to allocate the data required (i.e. internal {\tt malloc()} or {\tt realloc()}
calls failing), or passing too small an array into the
{\tt ...\_form\_data()} routines.

\subsubsection{Adjusting the size of a general size matrix}
Once a matrix has been created, its size may be adjusted dynamically
as needs arise. Gandalf stores the currently allocated maximum size of
a matrix in the matrix structure, so it can determine when the size of
the result of a computation will exceed the current size, and reallocate
accordingly. This happens automatically when a matrix is the result
of a calculation, but sometimes it is necessary to explicitly set the
size of a matrix. This is done using the following routine.
\begin{verbatim}
      Gan_Marix mA;

      /* create matrix A using e.g. gan_mat_form() */
      gan_mat_set_size ( &mA, 3 );
\end{verbatim}
This resets the size of the matrix {\tt mA} to 3.
If the size of a matrix created
by {\tt gan\_mat\_alloc()} or {\tt gan\_mat\_form()} is increased in size
in this way beyond its originally allocated size, {\tt gan\_mat\_set\_size()}
will automatically reallocate the matrix to the new size.
On the other hand, if {\tt gan\_mat\_form\_data()} was used to create the
matrix, it cannot be increased in size beyond the size of the array
passed as the last argument into {\tt gan\_mat\_form\_data()}.

For square matrices there are similar routines for specific matrix types,
for instance
\begin{verbatim}
      Gan_SquMarix smA;

      /* create matrix A using e.g. gan_squmat_form() */
      gan_symmat_set_size ( &mA, 3 ); /* set A to be a symmetric matrix with size 3 */
\end{verbatim}
An important feature here is that Gandalf allows both the size and type of
the matrix to be changed. For instance, the following code is valid:
\begin{verbatim}
      Gan_SquMarix smA;

      gan_diagmat_form ( &smA, 2 ); /* create matrix A as diagonal with size 2 */
      gan_symmat_set_size ( &mA, 3 ); /* set A to be a symmetric matrix with size 3 */
\end{verbatim}
Gandalf will reallocate the matrix internally if necessary. The main proviso
here is that if the matrix was created using a {\tt ...\_form\_data()} routine,
setting it to a type and size which requires more independent elements than
the size of the data array passed in is an error, so for instance
\begin{verbatim}
      Gan_SquMarix smA;
      double adAdata[10];

      /* create matrix A as diagonal with size 2 */
      gan_diagmat_form_data ( &smA, 2, adAdata, 10 );

      /* set A to be a symmetric matrix with size 4 */
      gan_symmat_set_size ( &mA, 4 );
\end{verbatim}
is OK, because a $4\times 4$ symmetric matrix has ten independent elements,
but an additional line
\begin{verbatim}
      gan_ltmat_set_size ( &smA, 5 ); /* set A to be a lower triangular matrix with size 5 */
\end{verbatim}
will fail (return {\tt NULL}), because a $5\times 5$ triangular matrix
has fifteen independent elements (see Equation~\ref{symtri-els}), and the
array {\tt adAdata} passed into the matrix originally has only ten elements.

The complete list of routines for setting a square matrix to a specific size
(here 5) and type is
\begin{verbatim}
      gan_symmat_set_size ( &smA, 5 ); /* set A to be a 5x5 symmetric matrix */
      gan_ltmat_set_size ( &smA, 5 ); /* set A to be a 5x5 lower triangular matrix */
      gan_utmat_set_size ( &smA, 5 ); /* set A to be a 5x5 upper triangular matrix */
      gan_diagmat_set_size ( &smA, 5 ); /* set A to be a 5x5 diagonal matrix */
      gan_scalImat_set_size ( &smA, 5 ); /* set A to be a 5x5 scaled identity matrix */
\end{verbatim}
and there is also a function for setting a matrix with a variable type:
\begin{verbatim}
      /* set A to be a symmetric matrix with size 5 */
      gan_squmat_set_type_size ( &smA, GAN_SYMMETRIC_MATRIX, 5 );
\end{verbatim}
and routines to set only the type, or only the size, of the matrix:
\begin{verbatim}
      /* set A to be a symmetric matrix, size unchanged */
      gan_squmat_set_type ( &smA, GAN_SYMMETRIC_MATRIX );

      /* set A to be size 4, type unchanged */
      gan_squmat_set_size ( &smA, 4 );
\end{verbatim}

{\bf Error detection:} {\tt NULL} is returned and the error handler is
invoked on failure. The most likely failure mode is failing to reallocate
the matrix data, i.e. failure of a call to {\tt realloc()}.

\subsubsection{Filling a general size matrix with values}
   \label{fill-mat-gen-sec}

To fill a matrix with values, create the matrix and then use the routine
{\tt gan\_mat\_fill\_va()}. An example is
\begin{verbatim}
      Gan_Matrix mA;

      /* ... create mA using e.g. gan_mat_form() ... */
      gan_mat_fill_va ( &mA, 2, 3, 1.0, 2.0, 3.0,
                                   4.0, 5.0, 6.0 );
\end{verbatim}
which sets matrix {\tt mA} to have dimensions 2 rows by 3 columns,
and sets the value to
\[ A = \beginm{ccc} 1 & 2 & 3 \\ 4 & 5 & 6 \endm
\]
To fill a matrix with a constant value, you can use
\begin{verbatim}
      gan_mat_fill_const_q ( &mA, 4, 2, 3.0 );
\end{verbatim}
which sets the dimensions of {\tt mA} to four rows by two columns,
and sets all the elements to three. This gives rise to the matrix
\[ A = \beginm{cccc} 3 & 3 & 3 & 3 \\ 3 & 3 & 3 & 3 \endm
\]
This function is also available in a form which allocates and fills a
matrix from scratch:
\begin{verbatim}
      Gan_Matrix *pmA;

      pmA = gan_mat_fill_const_s ( 4, 2, 3.0 );
\end{verbatim}
There are special macro routines for setting a matrix to zero:
\begin{verbatim}
      gan_mat_fill_zero_q ( &mA, 4, 2 ); /* OR */
      pmA = gan_mat_fill_zero_s ( 4, 2 );
\end{verbatim}

For square matrices there are specific routines for each square matrix
type. The order in which the elements are passed in the variable argument
list corresponds to the matrix type. For symmetric matrice, only the
{\em lower} triangle is passed (including the diagonal). So for instance
to create a symmetric matrix
\[ S = \beginm{ccc} 1 & 2 & 4 \\ 2 & 3 & 5 \\ 4 & 5 & 6 \endm
\]
use the code     
\begin{verbatim}
      Gan_SquMatrix smS;

      /* ... create smS using e.g. gan_symmat_form() ... */
      gan_symmat_fill_va ( &smS, 3, 1.0,
                                    2.0, 3.0,
                                    4.0, 5.0, 6.0 );
\end{verbatim}
For lower and upper triangular matrices pass the elements in the relevant order
for the corresponding triangle. So for instance
\begin{verbatim}
      Gan_SquMatrix smL;

      /* ... create smL using e.g. gan_ltmat_form() ... */
      gan_ltmat_fill_va ( &smL, 3, 1.0,
                                   2.0, 3.0,
                                   4.0, 5.0, 6.0 );
\end{verbatim}
creates the lower triangular  matrix
\[ L = \beginm{ccc} 1 & 0 & 0 \\ 2 & 3 & 0 \\ 4 & 5 & 6 \endm
\]
while
\begin{verbatim}
      Gan_SquMatrix smU;

      /* ... create smU using e.g. gan_utmat_form() ... */
      gan_utmat_fill_va ( &smU, 3, 1.0, 2.0, 4.0,
                                        3.0, 5.0,
                                             6.0 );
\end{verbatim}
creates the upper triangular  matrix
\[ U = \beginm{ccc} 1 & 2 & 4 \\ 0 & 3 & 5 \\ 0 & 0 & 6 \endm
\]
The routines for diagonal and scaled identity matrices follow a similar
pattern, so you can use
\begin{verbatim}
      Gan_SquMatrix smD;

      /* ... create smD using e.g. gan_diagmat_form() ... */
      gan_diagmat_fill_va ( &smD, 4, 1.0, 2.0, 3.0, 4.0 );
\end{verbatim}
to create the diagonal matrix
\[ D = \beginm{cccc} 1 & 0 & 0 & 0 \\ 0 & 2 & 0 & 0 \\ 0 & 0 & 3 & 0 \\ 0 & 0 & 0 & 4 \endm
\]
or
\begin{verbatim}
      Gan_SquMatrix smsI;

      /* ... create smsI using e.g. gan_scalImat_form() ... */
      gan_scalImat_fill_va ( &msI, 4, 2.0 );
\end{verbatim}
to create the scaled identity matrix
\[ D = \beginm{cccc} 2 & 0 & 0 & 0 \\ 0 & 2 & 0 & 0 \\ 0 & 0 & 2 & 0 \\ 0 & 0 & 0 & 2 \endm
\]

There are also routines to fill square matrices with a constant value, with
a special routine for filling with zero:
\begin{verbatim}
      Gan_SquMatrix smA;

      /* ... create smA using e.g. gan_squmat_form() ... */
      gan_symmat_fill_const_q ( &smA, 4, 3.0 ); /* set A as symmetric(4x4), each element 3 */
      gan_ltmat_fill_const_q ( &smA, 4, 3.0 ); /* set A as l. triang.(4x4), each element 3 */
      gan_utmat_fill_const_q ( &smA, 4, 3.0 ); /* set A as u. triang.(4x4), each element 3 */
      gan_diagmat_fill_const_q ( &smA, 4, 3.0 ); /* set A as diagonal(4x4), each element 3 */
      gan_scalImat_fill_const_q ( &smA, 4, 3.0 ); /* set A as scaled I(4x4), each element 3 */
      gan_symmat_fill_zero_q ( &smA, 4 ); /* set A as symmetric(4x4), each element zero */
      gan_ltmat_fill_zero_q ( &smA, 4 ); /* set A as l. triang.(4x4), each element zero */
      gan_utmat_fill_zero_q ( &smA, 4 ); /* set A as u. triang.(4x4), each element zero */
      gan_diagmat_fill_zero_q ( &smA, 4 ); /* set A as diagonal(4x4), each element zero */
      gan_scalImat_fill_zero_q ( &smA, 4 ); /* set A as scaled I(4x4), each element zero */
\end{verbatim}
set the type and size of an existing square matrix, and sets all its elements
to the same value,
\begin{verbatim}
      Gan_SquMatrix *psmS, *psmL, *psmU, *psmD, *psmsI;

      psmS = gan_symmat_fill_const_s ( 4, 3.0 ); /* create 4x4 symmetric mat., each el. 3 */
      psmL = gan_ltmat_fill_const_s ( 4, 3.0 ); /* create 4x4 u. tri. mat., each el. 3 */
      psmU = gan_utmat_fill_const_s ( 4, 3.0 ); /* create 4x4 l. tri. mat., each el. 3 */
      psmD = gan_diagmat_fill_const_s ( 4, 3.0 ); /* create 4x4 diagonal mat., each el. 3 */
      psmsI = gan_scalImat_fill_const_s ( 4, 3.0 ); /* create 4x4 scaled I mat., each el. 3 */
      psmS = gan_symmat_fill_zero_s ( 4 ); /* create 4x4 symmetric mat., each el. zero */
      psmL = gan_ltmat_fill_zero_s ( 4 ); /* create 4x4 u. tri. mat., each el. zero */
      psmU = gan_utmat_fill_zero_s ( 4 ); /* create 4x4 l. tri. mat., each el. zero */
      psmD = gan_diagmat_fill_zero_s ( 4 ); /* create 4x4 diagonal mat., each el. zero */
      psmsI = gan_scalImat_fill_zero_s ( 4 ); /* create 4x4 scaled I mat., each el. zero */
\end{verbatim}
create new matrices with the given type and size, and set all the elements to
the same value,

There are also equivalent routines that work with a variable square
matrix type:
\begin{verbatim}
      Gan_SquMatrix smA, *psmA;
      Gan_SquMatrixType type;

      /* ... create smA using e.g. gan_squmat_form() and set type to
             desired square matrix type, e.g. GAN_SYMMETRIC_MATRIX ...*/

      /* set up an existing matrix, fill it with a constant value */
      gan_squmat_fill_const_q ( &smA, type, 4, 3.0 ); /* constant element value 3 */
      gan_squmat_fill_zero_q ( &smA, type, 4 ); /* fill with zero */

      /* create a matrix from scratch */
      psmA = gan_squmat_fill_const_s ( type, 4, 3.0 ); /* constant element value 3 */
      psmA = gan_squmat_fill_const_s ( type, 4, 3.0 ); /* fill with zero */
\end{verbatim}

Note that the dynamic reconfiguration feature of square matrices means again
that the {\tt ...\_fill\_va()},\\ {\tt ..\_fill\_const\_q()} and
{\tt ...\_fill\_zero\_q()} square matrix
routines do not require the matrix to be set up initially with either
the same type or size.

{\bf Error detection:} {\tt NULL} is returned and the error handler is
invoked on failure. The most likely failure mode is failing to reallocate
the matrix data when the size of the matrix is changed,
i.e. failure of a call to {\tt realloc()}.

\subsubsection{Accessing the elements of a general size matrix}
To read the value of a specific element of a matrix $A$ use
\begin{verbatim}
      Gan_Matrix mA; /* matrix A */
      double dEl;

      /* ... create and fill matrix A ... */
      dEl = gan_mat_get_el ( &mA, 1, 2 ); /* returns A[1][2], A = (A[0][0] A[0][1] ... )
                                                                  (A[1][0] A[1][1] ... )
                                                                  (   :       :    ... ) */
\end{verbatim}
This sets {\tt dEl} to the third element of the second row of matrix $A$.
To set an element to a specific value use
\begin{verbatim}
      gan_mat_set_el ( &mA, 0, 3, 3.0 ); /* sets A[0][3] to 3.0 */
\end{verbatim}
This sets the fourth element of the first row of $A$ to 3.
There are also routines to increment or decrement an element of a matrix:
\begin{verbatim}
      gan_mat_inc_el ( &mA, 0, 1, 2.5 ); /* A[0][1] += 2.5 */
      gan_mat_dec_el ( &mA, 3, 2, 5.0 ); /* A[3][2] -= 5.0 */
\end{verbatim}
which respectively increment the second element of the first row of $A$
by 2.5 and subtract 5 from the third element of the fourth row of $A$.

For special square matrices there are equivalent routines which can be
illustrated by the following code fragment.
\begin{verbatim}
      Gan_SquMatrix smA; /* matrix A */
      double dEl;

      /* ... create and fill matrix A ... */
      dEl = gan_squmat_get_el ( &smA, 1, 2 ); /* returns A[1][2] */
      gan_squmat_set_el ( &smA, 0, 3, 3.0 ); /* sets A[0][3] to 3.0 */
      gan_squmat_inc_el ( &smA, 0, 1, 2.5 ); /* A[0][1] += 2.5 */
      gan_squmat_dec_el ( &smA, 3, 2, 5.0 ); /* A[3][2] -= 5.0 */
\end{verbatim}

{\bf Error detection:} {\tt gan\_\{squ\}mat\_set\_el()},
{\tt gan\_\{squ\}mat\_inc\_el()} and {\tt gan\_\{squ\}mat\_dec\_el()}
all return boolean values, with {\tt GAN\_FALSE}
returned on failure, in which case the Gandalf error handler is invoked.
The most likely failure modes is accessing an element outside the bounds of
the matrix, or seting an illegal element of a square matrix
(e.g. an off-diagonal element of a diagonal matrix).
If {\tt NDEBUG} is set then no error checking is done.
{\tt gan\_\{squ\}mat\_get\_el()} operate similarly, but return
{\tt DBL\_MAX} on error.

\subsubsection{Copying a general size matrix}
To copy a matrix $A$ to another matrix $B$, both matrices must have
been created, and $A$ should be filled with values. $B$ can be
created with arbitrary initial dimensions (for instance zero),
since Gandalf will
if necessary reallocate $B$ to the same size as $A$.
So for instance the following code is perfectly valid:
\begin{verbatim}
      Gan_Matrix mA, mB; /* declare matrices A & B */

      gan_mat_form ( &mA, 0, 0 ); /* create matrix A */
      gan_mat_form ( &mB, 0, 0 ); /* create matrix B */

      /* reallocate & initialise A */
      gan_mat_fill_va ( &mA, 2, 3, 11.0, 9.0, 7.0,
                                    5.0, 3.0, 1.0 );
      gan_mat_copy_q ( &mA, &mB ); /* set B = A, reallocating B */
\end{verbatim}
The last two lines reallocate first $A$ and then $B$, because both
were created with zero size. Note that $B$ may have previosly been
filled with other values, which are now lost.

There is also a version that creates a copy of a matrix from scratch:
\begin{verbatim}
      Gan_Matrix *pmB; /* declare matrix B */

      pmB = gan_mat_copy_s ( &mA ); /* create B and set B = A */
\end{verbatim}

For special square matrices, use one of the functions
\begin{verbatim}
      Gan_SquMatrix smA, smB, *psmB; /* declare matrices A & B */

      /* ... create A & B using e.g. gan_diagmat_form(), and initialise A using
             e.g. gan_diagmat_fill_va() ...  */
      gan_squmat_copy_q ( &smA, &smB ); /* set B = A, reallocating B if necessary, OR */
      psmB = gan_squmat_copy_s(&smA); /* set B = A, creating B */
\end{verbatim}

{\bf Error detection:} The matrix copy routines return {\tt NULL} and
invoke the Gandalf error handler upon failure.

\subsubsection{Transposing a general size matrix}
Gandalf supports implicit matrix transpose across all relevant routines,
so it is not often necessary to explicitly transpose a matrix.
Nonetheless, like matrix inverse it sometimes cannot be avoided.
To transpose a matrix $A$ into another matrix $B$, both matrices must have
been created, and $A$ should be filled with values. $B$ can be
created with arbitrary initial dimensions (for instance zero),
since Gandalf will
if necessary reallocate $B$ to the same size as $A$.
So for instance the following code is perfectly valid:
\begin{verbatim}
      Gan_Matrix mA, mB; /* declare matrices A & B */

      gan_mat_form ( &mA, 0, 0 ); /* create matrix A */
      gan_mat_form ( &mB, 0, 0 ); /* create matrix B */

      /* reallocate & initialise A */
      gan_mat_fill_va ( &mA, 2, 3, 11.0, 9.0, 7.0,
                                    5.0, 3.0, 1.0 );
      gan_mat_tpose_q ( &mA, &mB ); /* set B = A^T, reallocating B */
\end{verbatim}
The last two lines reallocate first $A$ and then $B$, because both
were created with zero size. Note that $B$ may have previosly been
filled with other values, which are now lost.

There is also a version that creates the transpose of a matrix from scratch:
\begin{verbatim}
      Gan_Matrix *pmB; /* declare matrix B */

      pmB = gan_mat_tpose_s ( &mA ); /* create B and set B = A */
\end{verbatim}
If in this case matrix $A$ happens to be square, Gandalf supports in-place
transpose:
\begin{verbatim}
      /* A have the same number of rows and columns */
      gan_mat_tpose_i ( &mA ); /* replace A = A^T */
\end{verbatim}

There is no explicit transpose implemented in Gandalf for special square
matrices. With the current matrix types supported by Gandalf, it would only
be relevant anyway for triangular matrices. Implicit transpose can handle
every practical situation.

{\bf Error detection:} The matrix transpose routines return {\tt NULL} and
invoke the Gandalf error handler upon failure.

\subsubsection{General size matrix addition}
To add two matrices $A$ and $B$ together, obtaining the sum
$C = A + B$, use the routine
\begin{verbatim}
      Gan_Matrix mA, mB, mC; /* declare matrices A, B and C */

      /* ... create and fill mA & mB, create mC ... */
      gan_mat_add_q ( &mA, &mB, &mC ); /* compute C = A + B */
\end{verbatim}
Again matrix $C$ is reallocated if necessary. Matrices $A$ and
$B$ must of course be the same size, or the error handler is invoked
and {\tt NULL} is returned. The sum matrix $C$ may be create from scratch
using
\begin{verbatim}
      Gan_Matrix *pmC; /* declare matrix C as pointer */

      /* ... create and fill mA & mB ... */
      pmC = gan_mat_add_s ( &mA, &mB ); /* compute C = A + B */
\end{verbatim}
Another way of computing matrix addition is to replace one of the input
matrix $A$ or $B$ with the result, using one of the in-place routines
\begin{verbatim}
      gan_mat_add_i1 ( &mA, &mB ); /* replace A = A + B */
      gan_mat_add_i2 ( &mA, &mB ); /* replace B = A + B */
\end{verbatim}
An alternative to {\tt gan\_mat\_add\_i1()} is the more explicit routine
\begin{verbatim}
      gan_mat_increment ( &mA, &mB ); /* replace A = A + B */
\end{verbatim}

There is also a set of routines for adding a general size matrix to the
transpose of another:
\begin{verbatim}
      Gan_Matrix mA, mB, mC, *pmC; /* declare matrices A, B and C */

      /* ... create and fill A & B, create C ... */

      /* B must have the same number of columns as A has rows, and vice versa */
      gan_mat_addT_q ( &mA, &mB, &mC ); /* compute C = A + B^T, OR */
      pmC = gan_mat_addT_s ( &mA, &mB ); /* compute C = A + B^T, OR */
      gan_mat_incrementT ( &mA, &mB ); /* replace A = A + B^T */
\end{verbatim}

Another set of routines allows you to add two matrices and generate a
symmetric matrix, on the assumption that the result is indeed symmetric.
Either matrix may be implicitly transposed for the purpose of the operation:
\begin{verbatim}
      Gan_Matrix mA, mB; /* declare matrices A, B */
      Gan_SquMatrix smS, *psmS; /* declare result matrix S */

      /* ... create and fill A & B, create S ... */

      /* for these functions, B must have the same number of columns and rows as A */
      gan_mat_add_sym_q ( &mA, &mB, &smS ); /* S = A + B, OR */
      psmS = gan_mat_add_sym_s ( &mA, &mB ); /* S = A + B */
      gan_matT_addT_sym_q ( &mA, &mB, &smS ); /* S = A^T + B^T, OR */
      psmS = gan_matT_addT_sym_s ( &mA, &mB ); /* S = A^T + B^T */

      /* here B must have the same number of columns as A has rows, and vice versa */
      gan_mat_addT_sym_q ( &mA, &mB, &smS ); /* S = A + B^T, OR */
      psmS = gan_mat_addT_sym_s ( &mA, &mB ); /* S = A + B^T */
      gan_matT_add_sym_q ( &mA, &mB, &smS ); /* S = A^T + B, OR */
      psmS = gan_matT_add_sym_s ( &mA, &mB ); /* S = A^T + B */
\end{verbatim}

Finally we have some routines for adding a matrix to its own transpose,
producing a symmetric matrix:
\begin{verbatim}
      Gan_Matrix mA; /* declare matrix A */
      Gan_SquMatrix smS, *psmS; /* declare result matrix S */

      /* ... create and fill A, create S ... */

      gan_mat_saddT_sym_q ( &mA, &smS ); /* S = A + A^T, OR */
      psmS = gan_mat_saddT_sym_s ( &mA ); /* S = A + A^T */
\end{verbatim}

There are equivalent functions for square matrices. Firstly the simple
routines for adding two matrices:
\begin{verbatim}
      Gan_SquMatrix smA, smB, smC, *psmC; /* declare matrices A, B & C */

      /* ... create and fill smA & smB, create smC ... */
      gan_squmat_add_q ( &smA, &smB, &smC ); /* compute C = A + B, OR */
      gan_squmat_add_i1 ( &smA, &smB ); /* replace A = A + B, OR */
      gan_squmat_add_i2 ( &smA, &smB ); /* replace B = A + B, OR */
      gan_squmat_increment ( &smA, &smB ); /* replace A = A + B, OR */
      psmC = gan_squmat_add_s ( &smA, &smB ); /* compute C = A + B as new matrix */
\end{verbatim}
Other routines implicitly transpose one of the input matrices:
\begin{verbatim}
      Gan_SquMatrix smA, smB, smC, *psmC; /* declare matrices A, B & C */

      /* ... create and fill smA & smB, create smC ... */
      gan_squmat_addT_q ( &smA, &smB, &smC ); /* compute C = A + B^T, OR */
      gan_squmat_incrementT ( &smA, &smB ); /* replace A = A + B^T, OR */
      psmC = gan_squmat_addT_s ( &smA, &smB ); /* compute C = A + B^T as new matrix */
\end{verbatim}

{\bf Error detection:} {\tt NULL} is returned and the Gandalf error handler
invoked if the matrix addition fails.
The most likely failure modes are failing to create/set the result matrix,
or size/type incompatibility between the input matrices.

\subsubsection{General size matrix subtraction}
The routines for matrix subtraction follow the scheme of those for
matrix addition, leading to the options
\begin{verbatim}
      Gan_Matrix mA, mB, mC; /* declare matrices x, y and z */
      Gan_Matrix *pmC; /* declare matrix z alternatively as pointer */

      /* ... create and fill mA & mB, create mC ... */
      gan_mat_sub_q ( &mA, &mB, &mC ); /* compute C = A - B */
      pmC = gan_mat_sub_s ( &mA, &mB ); /* compute C = A - B */
      gan_mat_sub_i1 ( &mA, &mB ); /* replace A = A - B */
      gan_mat_sub_i2 ( &mA, &mB ); /* replace B = A - B */
      gan_mat_decrement ( &mA, &mB ); /* replace A = A - B */
\end{verbatim}
If one of the input matrices is to be implicitly transposed, use instead
\begin{verbatim}
      Gan_Matrix mA, mB, mC; /* declare matrices x, y and z */
      Gan_Matrix *pmC; /* declare matrix z alternatively as pointer */

      /* ... create and fill mA & mB, create mC ... */

      /* here B must have the same number of columns as A has rows, and vice versa */
      gan_mat_subT_q ( &mA, &mB, &mC ); /* compute C = A - B^T */
      pmC = gan_mat_subT_s ( &mA, &mB ); /* compute C = A - B^T */
      gan_mat_decrementT ( &mA, &mB ); /* replace A = A - B^T */
\end{verbatim}

There are equivalent functions for square matrices. Firstly the simple
routines for subtracting two matrices:
\begin{verbatim}
      Gan_SquMatrix smA, smB, smC, *psmC; /* declare matrices A, B & C */

      /* ... create and fill smA & smB, create smC ... */
      gan_squmat_sub_q ( &smA, &smB, &smC ); /* compute C = A - B, OR */
      gan_squmat_sub_i1 ( &smA, &smB ); /* replace A = A - B, OR */
      gan_squmat_sub_i2 ( &smA, &smB ); /* replace B = A - B, OR */
      gan_squmat_decrement ( &smA, &smB ); /* replace A = A - B, OR */
      psmC = gan_squmat_sub_s ( &smA, &smB ); /* compute C = A - B as new matrix */
\end{verbatim}
Other routines implicitly transpose one of the input matrices:
\begin{verbatim}
      Gan_SquMatrix smA, smB, smC, *psmC; /* declare matrices A, B & C */

      /* ... create and fill smA & smB, create smC ... */
      gan_squmat_subT_q ( &smA, &smB, &smC ); /* compute C = A - B^T, OR */
      gan_squmat_decrementT ( &smA, &smB ); /* replace A = A - B^T, OR */
      psmC = gan_squmat_subT_s ( &smA, &smB ); /* compute C = A - B^T as new matrix */
\end{verbatim}

{\bf Error detection:} {\tt NULL} is returned and the Gandalf error handler
invoked if the matrix addition fails.
The most likely failure modes are failing to create/set the result matrix,
or size/type incompatibility between the input matrices.

\subsubsection{Rescaling a general size matrix}
Multiplying or dividing a matrix by a scalar value follows the scheme of the
above copy, addition and subtraction operations. To multiply a matrix $A$
by a scalar $s$, $B=sA$, use for example
\begin{verbatim}
      Gan_Matrix mA, mB; /* declare matrices A & B */

      /* ... create & fill mA, create (& optionally fill) mB ... */
      gan_mat_scale_q ( &mA, 5.0, &mB ); /* B = 5*A */
\end{verbatim}
to multiply all the elements in matrix $A$ by five, writing the result
into matrix $B$.
Alternatively you can create the rescaled matrix from scratch as in
\begin{verbatim}
      Gan_Matrix *pmB; /* declare matrix B */

      /* ... create & fill mA ... */
      pmB = gan_mat_scale_s ( &mA, 5.0 ); /* B = 5*A */
\end{verbatim}
or overwrite $A$ with the result
\begin{verbatim}
      gan_mat_scale_i ( &mA, 5.0 ); /* replace A = 5*A */
\end{verbatim}

There are similar routines for dividing a general size matrix by a scalar
value:
\begin{verbatim}
      gan_mat_divide_q ( &mA, 5.0, &mB ); /* B = A/5 */
      pmB = gan_mat_divide_s ( &mA, 5.0 ); /* B = A/5 */
      gan_mat_divide_i ( &mA, 5.0 ); /* replace A = A/5 */
\end{verbatim}

There are specific routines to negate a matrix, i.e. multiply it by -1,
as follows:
\begin{verbatim}
      gan_mat_negate_q ( &mA, &mB ); /* B = -A */
      pmB = gan_mat_negate_s ( &mA ); /* B = -A */
      gan_mat_negate_i ( &mA ); /* replace A = -A */
\end{verbatim}

The equivalent routines for square matrices are
\begin{verbatim}
      Gan_SquMatrix smA, smB, *psmB; /* declare matrices A & B */

      /* ... create & fill smA, create (& optionally fill) smB ... */

      /* scale a square matrix */
      gan_squmat_scale_q ( &smA, 5.0, &smB ); /* B = 5*A, B an existing matrixOR */
      psmB = gan_squmat_scale_s ( &smA, 5.0 ); /* B = 5*A, B a new matrix, OR */
      gan_squmat_scale_i ( &smA, 5.0 ); /* replace A = 5*A */

      /* divide a square matrix by a scalar */
      gan_squmat_divide_q ( &smA, 5.0, &smB ); /* B = A/5, B an existing matrix, OR */
      psmB = gan_squmat_divide_s ( &smA, 5.0 ); /* B = A/5, B a new matrix, OR */
      gan_squmat_divide_i ( &smA, 5.0 ); /* replace A = A/5 */

      /* negate a square matrix */
      gan_squmat_negate_q ( &smA, &smB ); /* B = -A, B an existing matrix, OR */
      psmB = gan_squmat_negate_s ( &smA ); /* B = -A, B a new matrix, OR */
      gan_squmat_negate_i ( &smA ); /* replace A = -A */
\end{verbatim}
Passing zero as the scalar value to the {\tt gan\_mat\_divide\_[qsi]()}
or {\tt gan\_squmat\_divide\_[qsi]()} routines invokes the error handler,
with a division by zero error
(error code {\tt GAN\_ERROR\_DIVISION\_BY\_ZERO}),
and {\tt NULL} is returned.

{\tt Error detection:} The Gandalf error handler is invoked and {\tt NULL}
is returned if an error occurs. The most likely failure modes are
(i) failing to create the result matrix; (ii) division by zero.

\subsubsection{General size matrix/vector multiplication}
The general size matrix/vector multiplication, with optional implicit
transpose of the matrix, computes one of the operations
\[ \yvec = A\xvec\;\;\;\;\mbox{OR} \;\;\;\;\yvec = A\tr\xvec
\]
for vectors $\xvec$, $\yvec$ and matrix $A$. They are implemented in Gandalf
as follows.
\begin{verbatim}
      Gan_Matrix mA; /* matrix A */
      Gan_Vector vx, vy; /* vectors x & y */

      /* ... create and fill matrix A and vector x, create (and optionally
             fill) vector y ... */
      gan_mat_multv_q ( &mA, &vx, &vy ); /* set y = A*x, OR */
      gan_matT_multv_q ( &mA, &vx, &vy ); /* set y = A^T*x */
\end{verbatim}
with the alternative forms
\begin{verbatim}
      Gan_Matrix mA; /* matrix A */
      Gan_Vector vx, *pvy; /* vectors x & y */

      /* ... create and fill matrix A and vector x ... */
      pvy = gan_mat_multv_s ( &mA, &vx ); /* set y = A*x, y a new vector, OR */
      pvy = gan_matT_multv_s ( &mA, &vx ); /* set y = A^T*x, y a new vector */
\end{verbatim}

If $A$ is a special square matrix, more options are available. If $A$ is
a triangular matrix, multiplication with a vector can be implemented as
an in-place operation, whether or not $A$ is (implicitly) inverted or
transposed, in any combination. This gives rise to the following Gandalf
routines.
\begin{verbatim}
      Gan_SquMatrix smA; /* matrix A */
      Gan_Vector vx, vy; /* vectors x & y */

      /* ... create and fill matrix A and vector x, create (and optionally
             fill) vector y ... */
      gan_squmat_multv_q ( &smA, &vx, &vy ); /* set y = A*x, OR */
      gan_squmatT_multv_q ( &smA, &vx, &vy ); /* set y = A^T*x, OR */
      gan_squmatI_multv_q ( &smA, &vx, &vy ); /* set y = A^-1*x, OR */
      gan_squmatIT_multv_q ( &smA, &vx, &vy ); /* set y = A^-T*x */
\end{verbatim}
with in-place versions
\begin{verbatim}
      Gan_SquMatrix smA; /* matrix A */
      Gan_Vector vx; /* vector x */

      /* ... create and fill matrix A and vector x ... */
      gan_squmat_multv_i ( &smA, &vxy ); /* replace x = A*x, OR */
      gan_squmatT_multv_i ( &smA, &vx ); /* replace x = A^T*x, OR */
      gan_squmatI_multv_i ( &smA, &vx ); /* replace x = A^-1*x, OR */
      gan_squmatIT_multv_i ( &smA, &vx ); /* replace x = A^-T*x */
\end{verbatim}
and also the routines to create the result vector from scratch:
\begin{verbatim}
      Gan_SquMatrix smA; /* matrix A */
      Gan_Vector vx, *pvy; /* vectors x & y */

      /* ... create and fill matrix A and vector x ... */
      pvy = gan_squmat_multv_s ( &smA, &vxy ); /* set y = A*x, OR */
      pvy = gan_squmatT_multv_s ( &smA, &vx ); /* set y = A^T*x, OR */
      pvy = gan_squmatI_multv_s ( &smA, &vx ); /* set y = A^-1*x, OR */
      pvy = gan_squmatIT_multv_s ( &smA, &vx ); /* set y = A^-T*x */
\end{verbatim}
Note that the implicit inverse and in-place features are not available when
$A$ is of symmetric type; Gandalf will invoke the error handler and return
an error condition {\tt NULL} if {\tt mA} has type
{\tt GAN\_SYMMETRIC\_MATRIX}.

{\bf Error detection:} If implicit inverse is used (the
{\tt gan\_squmatI\_multv\_[qsi]()} or {\tt gan\_squmatIT\_multv\_[qsi]()}
routines), the matrix must be non-singular. If the matrix is singular
then {\tt NULL} is returned and the Gandalf error handler is invoked.
Other failure modes are failing to create the result vector and incompatibility
between the sizes of the input matrix and vector.

\subsubsection{General size matrix/matrix multiplication}
Similar options are available to matrix/matrix multiplication as with
matrix/vector multiplication, with the added complication that either or
both if the matrices may have an implicit transpose or (for square matrices)
inverse applied to them. Firstly we list the routines available when
both input matrices are general rectangular matrices. In this case only
implicit transpose is of relevance to us, and we can write the operations
we need to implement as
\[ C = A B,\;\;\;\;\;C = A\tr B,\;\;\;\;C = A B\tr,\;\;\;\;C = A\tr B\tr
\]
To right-multiply a matrix $A$ by another matrix $B$, with all the above
transpose combinations, the Gandalf routines are
\begin{verbatim}
      Gan_Matrix mA, mB, mC; /* matrices A, B & C */

      /* ... create and fill matrices A, B, create matrix C ... */
      gan_mat_rmult_q ( &mA, &mB, &mC ); /* C = A*B, OR */
      gan_mat_rmultT_q ( &mA, &mB, &mC ); /* C = A*B^T, OR */
      gan_matT_rmult_q ( &mA, &mB, &mC ); /* C = A^T*B, OR */
      gan_matT_rmultT_q ( &mA, &mB, &mC ); /* C = A^T*B^T, OR */
\end{verbatim}
with similar routines to create the result matrix $C$ from scratch:
\begin{verbatim}
      Gan_Matrix mA, mB, *pmC; /* matrices A, B & C */

      /* ... create and fill matrices A, B ... */
      pmC = gan_mat_rmult_s ( &mA, &mB ); /* C = A*B, OR */
      pmC = gan_mat_rmultT_s ( &mA, &mB ); /* C = A*B^T, OR */
      pmC = gan_matT_rmult_s ( &mA, &mB ); /* C = A^T*B, OR */
      pmC = gan_matT_rmultT_s ( &mA, &mB ); /* C = A^T*B^T, OR */
\end{verbatim}
The next set of routines deals with the case where it is known that the
result of multiplying matrices $A$ and $B$ is symmetric. In that case we
can use the routines
\begin{verbatim}
      Gan_Matrix mA, mB; /* matrices A & B */
      Gan_SquMatrix smC; /* matrix C */

      /* ... create and fill matrices A, B, create matrix C ... */
      gan_mat_rmult_sym_q ( &mA, &mB, &mC ); /* C = A*B, OR */
      gan_mat_rmultT_sym_q ( &mA, &mB, &mC ); /* C = A*B^T, OR */
      gan_matT_rmult_sym_q ( &mA, &mB, &mC ); /* C = A^T*B, OR */
      gan_matT_rmultT_sym_q ( &mA, &mB, &mC ); /* C = A^T*B^T */
\end{verbatim}
with the alternatives
\begin{verbatim}
      Gan_Matrix mA, mB; /* matrices A & B */
      Gan_SquMatrix *psmC; /* matrix C */

      /* ... create and fill matrices A, B ... */
      psmC = gan_mat_rmult_sym_s ( &mA, &mB ); /* C = A*B, OR */
      psmC = gan_mat_rmultT_sym_s ( &mA, &mB ); /* C = A*B^T, OR */
      psmC = gan_matT_rmult_sym_s ( &mA, &mB ); /* C = A^T*B, OR */
      psmC = gan_matT_rmultT_sym_s ( &mA, &mB ); /* C = A^T*B^T */
\end{verbatim}
In the case that $A$ and $B$ are the same matrix, we can be sure that both
$AA\tr$ and $A\tr A$ are symmetric, and there are special Gandalf routines
for these cases, distinguished according to whether $A$ is multiplied by its
own transpose on the {\bf r}ight or {\bf l}eft:
\begin{verbatim}
      Gan_Matrix mA; /* matrix A */
      Gan_SquMatrix smC; /* matrix C */

      /* ... create and fill matrix A, create matrix C ... */
      gan_mat_srmultT_q ( &mA, &mC ); /* C = A*A^T, OR */
      gan_mat_slmultT_q ( &mA, &mC ); /* C = A^T*A */
\end{verbatim}
with the alternatives
\begin{verbatim}
      Gan_Matrix mA; /* matrix A */
      Gan_SquMatrix *psmC; /* matrix C */

      /* ... create and fill matrix A ... */
      psmC = gan_mat_srmultT_s ( &mA ); /* C = A*A^T, OR */
      psmC = gan_mat_slmultT_s ( &mA ); /* C = A^T*A */
\end{verbatim}

If one or both of the input matrices is a special square matrix, there are
many more combinations available. First consider a square matrix $A$ being
multiplied on left or right by a general rectangular matrix $B$, giving
a result matrix $C$. Given the possibility of both implicit transpose and
inverse of the square matrix, we need to consider the operations
\[ C = A B,\;\;\;C = A B\tr,\;\;\;C = A\tr B,\;\;\;C = A\tr B\tr,
\]
\[ C = A\inv B,\;\;\;C = A\inv B\tr,\;\;\;C = A\trinv B,\;\;\;C = A\trinv B\tr
\]
\[ C = B A,\;\;\;C = B\tr A,\;\;\;C = B A\tr,\;\;\;C = B\tr A\tr,
\]
\[ C = B A\inv,\;\;\;C = B\tr A\inv,\;\;\;C = B A\trinv,\;\;\;C = B\tr A\trinv
\]
These operations are implemented by the following Gandalf routines:
\begin{verbatim}
      Gan_SquMatrix smA; /* square matrix A */
      Gan_Matrix mB, mC; /* matrices B & C */

      /* ... create and fill matrices A, B, create matrix C ... */

      /* routines right-multipling A by B */
      gan_squmat_rmult_q ( &smA, &mB, &mC ); /* C = A*B, OR */
      gan_squmat_rmultT_q ( &smA, &mB, &mC ); /* C = A*B^T, OR */
      gan_squmatT_rmult_q ( &smA, &mB, &mC ); /* C = A^T*B, OR */
      gan_squmatT_rmultT_q ( &smA, &mB, &mC ); /* C = A^T*B^T, OR */
      gan_squmatI_rmult_q ( &smA, &mB, &mC ); /* C = A^-1*B, OR */
      gan_squmatI_rmultT_q ( &smA, &mB, &mC ); /* C = A^-1*B^T, OR */
      gan_squmatIT_rmult_q ( &smA, &mB, &mC ); /* C = A^-T*B, OR */
      gan_squmatIT_rmultT_q ( &smA, &mB, &mC ); /* C = A^-T*B^T */

      /* routines left-multipling A by B */
      gan_squmat_lmult_q ( &smA, &mB, &mC ); /* C = B*A, OR */
      gan_squmat_lmultT_q ( &smA, &mB, &mC ); /* C = B^T*A, OR */
      gan_squmatT_lmult_q ( &smA, &mB, &mC ); /* C = B*A^T, OR */
      gan_squmatT_lmultT_q ( &smA, &mB, &mC ); /* C = B^T*A^T, OR */
      gan_squmatI_lmult_q ( &smA, &mB, &mC ); /* C = B*A^-1, OR */
      gan_squmatI_lmultT_q ( &smA, &mB, &mC ); /* C = B^T*A^-1, OR */
      gan_squmatIT_lmult_q ( &smA, &mB, &mC ); /* C = B*A^-T, OR */
      gan_squmatIT_lmultT_q ( &smA, &mB, &mC ); /* C = B^T*A^-T */
\end{verbatim}
These routines have the alternative form
\begin{verbatim}
      Gan_SquMatrix smA; /* square matrix A */
      Gan_Matrix mB, *pmC; /* matrices B & C */

      /* ... create and fill matrices A, B ... */

      /* routines right-multipling A by B */
      pmC = gan_squmat_rmult_s ( &smA, &mB ); /* C = A*B, OR */
      pmC = gan_squmat_rmultT_s ( &smA, &mB ); /* C = A*B^T, OR */
      pmC = gan_squmatT_rmult_s ( &smA, &mB ); /* C = A^T*B, OR */
      pmC = gan_squmatT_rmultT_s ( &smA, &mB ); /* C = A^T*B^T, OR */
      pmC = gan_squmatI_rmult_s ( &smA, &mB ); /* C = A^-1*B, OR */
      pmC = gan_squmatI_rmultT_s ( &smA, &mB ); /* C = A^-1*B^T, OR */
      pmC = gan_squmatIT_rmult_s ( &smA, &mB ); /* C = A^-T*B, OR */
      pmC = gan_squmatIT_rmultT_s ( &smA, &mB ); /* C = A^-T*B^T */

      /* routines left-multipling A by B */
      pmC = gan_squmat_lmult_s ( &smA, &mB ); /* C = B*A, OR */
      pmC = gan_squmat_lmultT_s ( &smA, &mB ); /* C = B^T*A, OR */
      pmC = gan_squmatT_lmult_s ( &smA, &mB ); /* C = B*A^T, OR */
      pmC = gan_squmatT_lmultT_s ( &smA, &mB ); /* C = B^T*A^T, OR */
      pmC = gan_squmatI_lmult_s ( &smA, &mB ); /* C = B*A^-1, OR */
      pmC = gan_squmatI_lmultT_s ( &smA, &mB ); /* C = B^T*A^-1, OR */
      pmC = gan_squmatIT_lmult_s ( &smA, &mB ); /* C = B*A^-T, OR */
      pmC = gan_squmatIT_lmultT_s ( &smA, &mB ); /* C = B^T*A^-T */
\end{verbatim}
The in-place versions will overwrite the contents of matrix $B$ with the
result, and work fine unless $A$ is of symmetric type (in which case the
error handler is invoked and {\tt NULL} returned):
\begin{verbatim}
      Gan_SquMatrix smA; /* square matrix A */
      Gan_Matrix mB; /* matrix B */

      /* ... create and fill matrices A, B ... */

      /* routines right-multipling A by B */
      gan_squmat_rmult_i ( &smA, &mB ); /* replace B = A*B, OR */
      gan_squmat_rmultT_i ( &smA, &mB ); /* replace B = A*B^T, OR */
      gan_squmatT_rmult_i ( &smA, &mB ); /* replace B = A^T*B, OR */
      gan_squmatT_rmultT_i ( &smA, &mB ); /* replace B = A^T*B^T, OR */
      gan_squmatI_rmult_i ( &smA, &mB ); /* replace B = A^-1*B, OR */
      gan_squmatI_rmultT_i ( &smA, &mB ); /* replace B = A^-1*B^T, OR */
      gan_squmatIT_rmult_i ( &smA, &mB ); /* replace B = A^-T*B, OR */
      gan_squmatIT_rmultT_i ( &smA, &mB ); /* replace B = A^-T*B^T */

      /* routines left-multipling A by B */
      gan_squmat_lmult_i ( &smA, &mB ); /* replace B = B*A, OR */
      gan_squmat_lmultT_i ( &smA, &mB ); /* replace B = B^T*A, OR */
      gan_squmatT_lmult_i ( &smA, &mB ); /* replace B = B*A^T, OR */
      gan_squmatT_lmultT_i ( &smA, &mB ); /* replace B = B^T*A^T, OR */
      gan_squmatI_lmult_i ( &smA, &mB ); /* replace B = B*A^-1, OR */
      gan_squmatI_lmultT_i ( &smA, &mB ); /* replace B = B^T*A^-1, OR */
      gan_squmatIT_lmult_i ( &smA, &mB ); /* replace B = B*A^-T, OR */
      gan_squmatIT_lmultT_i ( &smA, &mB ); /* replace B = B^T*A^-T */
\end{verbatim}
Now we consider multiplying a square matrix $A$ by its own transpose, producing
a symmetric matrix $B$. This operation will most often be applied to triangular
matrices, and in Gandalf implicit transpose and inverse of the input
matrix are supported, giving rise to the operations
\[ B = A A\tr,\;\;\;\;B = A\tr A,\;\;\;\;B = A\inv A\trinv,\;\;\;\;B = A\trinv A\inv
\]
which are implemented by the routines
\begin{verbatim}
      Gan_SquMatrix smA, smB; /* declare matrices A & B */

      /* ... create & fill matrix A, create (& optionally fill) matrix B ... */
      gan_squmat_srmultT_squ_q ( &smA, &smB ); /* set B = A*A^T, OR */
      gan_squmatT_srmult_squ_q ( &smA, &smB ); /* set B = A^T*A, OR */
      gan_squmatI_srmultIT_squ_q ( &smA, &smB ); /* set B = A^-1*A^-T, OR */
      gan_squmatIT_srmultI_squ_q ( &smA, &smB ); /* set B = A^-T*A^-1 */
\end{verbatim}
There are also routines to build the result matrix $B$ from scratch:
\begin{verbatim}
      Gan_SquMatrix smA, *psmB; /* declare matrices A & B */

      /* ... create & fill matrix A ... */
      psmB = gan_squmat_srmultT_squ_s ( &smA ); /* create B = A*A^T, OR */
      psmB = gan_squmatT_srmult_squ_s ( &smA ); /* create B = A^T*A, OR */
      psmB = gan_squmatI_srmultIT_squ_s ( &smA ); /* create B = A^-1*A^-T, OR */
      psmB = gan_squmatIT_srmultI_squ_s ( &smA ); /* create B = A^-T*A^-1 */
\end{verbatim}
and in-place versions of these operations are also available:
\begin{verbatim}
      Gan_SquMatrix smA; /* declare matrix A */

      /* ... create & fill matrix A ... */
      gan_squmat_srmultT_squ_i ( &smA ); /* replace A = A*A^T, OR */
      gan_squmatT_srmult_squ_i ( &smA ); /* replace A = A^T*A, OR */
      gan_squmatI_srmultIT_squ_i ( &smA ); /* replace A = A^-1*A^-T, OR */
      gan_squmatIT_srmultI_squ_i ( &smA ); /* replace A = A^-T*A^-1 */
\end{verbatim}

Finally, there is a set of routines that multiply a symmetric matrix on left
and right by a rectangular matrix and its transpose, producing another
symmetric matrix. The operations implemented are
\[ S' = A S A\tr,\;\;\;\;\;\;S' = A\tr S A.
\]
This triple product is implemented as two matrix multiplications, and the
matrix to hold the intermediate result is also passed in to the routines,
so that it is also available on output. The routines are
\begin{verbatim}
      Gan_SquMatrix smS, smSp; /* declare matrices S & S' */
      Gan_Matrix mA, mB; /* declare matrices A & B */

      /* ... create & fill matrices S & A, create (& optionally fill) matrices B & Sp ... */
      gan_symmat_lrmult_q ( &smS, &mA, &mB, &smSp ); /* set B = S*A^T and Sp = A*S*A^T, OR */
      gan_symmat_lrmultT_q ( &smS, &mA, &mB, &smSp ); /* set B = S*A and Sp = A^T*S*A */
\end{verbatim}
with alternative versions that create the result matrix $S'$ from scratch:
\begin{verbatim}
      Gan_SquMatrix smS, *psmSp; /* declare matrices S & S' */
      Gan_Matrix mA, mB; /* declare matrices A & B */

      /* ... create & fill matrices S & A, create (& optionally fill) matrix B ... */
      psmSp = gan_symmat_lrmult_s ( &smS, &mA, &mB ); /* set B = S*A^T and Sp = A*S*A^T, OR */
      psmSp = gan_symmat_lrmultT_s ( &smS, &mA, &mB ); /* set B = S*A and Sp = A^T*S*A */
\end{verbatim}
It is allowable to pass {\tt NULL} for the $B$ matrix ({\tt \&mB} in the
above function calls). In that case the intermediate result is computed and
thrown away.

\subsubsection{Inverting a general size matrix}
If a general rectangular matrix $A$ happens to be square, it can be
inverted using the routine
\begin{verbatim}
      Gan_Matrix mA, mB; /* declare matrices A & B */

      /* ... create and fill matrix A, which must be square, create B ... */
      gan_mat_invert_q ( &mA, &mB ); /* B = A^-1 */
\end{verbatim}
There is also a routine to create the inverse matrix $A\inv$ from scratch:
\begin{verbatim}
      Gan_Matrix mA, *pmB; /* declare matrix A */

      /* ... create and fill matrix A, which must be square ... */
      pmB = gan_mat_invert_s ( &mA ); /* B = A^-1 */
\end{verbatim}
The routines for special square matrices are similar
\begin{verbatim}
      Gan_Matrix mA, mB, *pmB; /* declare matrices A & B */

      /* ... create and fill matrix A, which must be square, create B ... */
      gan_squmat_invert_q ( &mA, &mB ); /* B = A^-1, OR */
      pmB = gan_squmat_invert_s ( &mA ); /* B = A^-1 */
\end{verbatim}
The type of the output $B$ is in this case set to the appropriate type given
the input. For all the square matrix types supported by Gandalf (symmetric,
triangular, diagonal, scaled identity), the matrix type of the inverse $B$
is the same as that of the input matrix $A$.

{\bf Error detection:} If implicit inverse is used (the
{\tt ...\_squmatI\_...()} or {\tt ...\_squmatIT\_...()} routines),
the square matrix involved must be non-singular. If the matrix is singular
then {\tt NULL} is returned and the Gandalf error handler is invoked.
Other failure modes are failing to create the result matrix and incompatibility
between the sizes of the input matrices.

\subsubsection{Cholesky factorising a general size symmetric matrix}
The Cholesky factor of a symmetric positive definite matrix $S$ is
a lower triangular matrix $L$ such that
\[ S = L L\tr
\]
and we write $L={\rm chol}(S)$.
The Gandalf Cholesky factorisation routines apply to all symmetric types
of matrix, i.e. {\tt GAN\_SYMMETRIC\_MATRIX} itself,
{\tt GAN\_DIAGONAL\_MATRIX} and {\tt GAN\_SCALED\_IDENT\_MATRIX}.
Routines are available to compute the factorisation, with the usual options,
as follows:
\begin{verbatim}
      Gan_SquMatrix smS, smL, *psmL; /* declare matrices S & L */

      /* ... create and fill matrix S, which must be symmetric and positive definite,
             create L ... */
      gan_symmat_cholesky_q ( &smS, &smL ); /* L = chol(S), OR */
      psmL = gan_symmat_cholesky_s ( &smS ); /* L = chol(S) */
      gan_symmat_cholesky_i ( &smS ); /* replace S = chol(S) */
\end{verbatim}
The last option {\tt gan\_symmat\_cholesky\_i()} replaces $S$ in-place
by ${\rm chol}(S)$.

{\bf Error detection:} If $S$ is not either symmetric or positive definite
in the above routines, {\tt NULL} is returned and the Gandalf error handler
is invoked. Another failure mode is failing to create the result matrix.

\subsubsection{Symmetric matrix eigendecomposition}
\begin{verbatim}
      #include <gandalf/linalg/mat_symmetric.h>
\end{verbatim}
Gandalf has a routine for computing the real eigenvalues and eigenvectors of
a general size symmetric matrix, based on either the
CLAPACK routine {\tt dspev()} or the CCMath routine {\tt eigval()}.
A symmetric matrix $S$ can be written as
\[ S Z = Z W
\]
where $W$ is a diagonal matrix of real eigenvalues and $Z$ is a square
matrix of orthognal eigenvectors, unique if the eigenvalues are distinct.
If the matrix is positive definite (or semi-definite) then all the eigenvalues
will be $> 0$ (or $\geq 0$). Here is an example code fragment using the
Gandalf routine to compute $W$ and (optionally) $Z$.
\begin{verbatim}
      Gan_SquMatrix smS; /* declare symmetric matrix */
      Gan_SquMatrix smW; /* declare matrix of eigenvalues W */
      Gan_Matrix mZ; /* declare matrix of eigenvectors */

      /* create and fill S */
      gan_symmat_form ( &smS, 5 );
      gan_symmat_fill_va ( &smS, 5,
                            1.0,
                            2.0,  3.0,
                            4.0,  5.0,  6.0,
                            7.0,  8.0,  9.0, 10.0,
                           11.0, 12.0, 13.0, 14.0, 15.0 );

      /* create Z and W */
      gan_mat_form ( &mZ, 5, 5 );
      gan_diagmat_form ( &smW, 0 );

      /* compute sigenvalues and eigenvectors of S */
      gan_symmat_eigen ( &smS, &smW, &mZ, GAN_TRUE, NULL, 0 );
\end{verbatim}
After calling this routine {\tt smW} will contain the computed eigenvalues,
and {\tt mZ} the eigenvectors. If the eigenvector matrix is passed as
{\tt NULL}, the eigenvectors are not computed. The boolean fourth argument
indicates whether the eigenvectors should be sorted into ascending order.
The fifth and sixth arguments define a workspace array of {\tt double}s,
and the size of the array, which can be used by LAPACK. If passed as
{\tt NULL, 0} as above, the workspace is allocated inside the function.

\subsubsection{Accumulated symmetric matrix eigendecomposition}
\label{acc-symeigen-sec}
\begin{verbatim}
      #include <gandalf/linalg/symmat_eigen.h>
\end{verbatim}
There is also a specific module in Gandalf to compute the eigenvalues and
eigenvectors of a positive semi-definite matrix accumulated as a sum of
outer products of vectors. This is useful for instance when solving
homogeneous linear equations; for an example see the computation of the
fundamental \& essential matrix in Section~\ref{fun-ess-sec}.
The symmetric matrix $S$ is constructed as
\begin{equation}
 S = \sum_{i=1}^n \sigma_i \xvec_i \xvec_i\tr
 \label{acc-eigen-sum}
\end{equation}
given the $n$ vectors $\xvec_i$ and weighting factors $s_i$, $i=1,\ldots,n$.

There is a structure to hold the accumulated matrix and the resulting
eigendecomposition matrices:
\begin{verbatim}
      /* structure to hold accumulated symmetric matrix and resulting
       * eigendecomposition of a sum of vector outer products
       */
      typedef struct
      {
         Gan_SquMatrix SxxT; /* accumulated sum of vector outer products */
         Gan_SquMatrix W;    /* diagonal matrix of eigenvalues */
         Gan_Matrix    Z;    /* matrix of eigenvectors */
         Gan_Vector    work; /* workspace vector for computing eigendecomposition */

         /* whether this structure was dynamically allocated */
         Gan_Bool alloc;
      } Gan_SymMatEigenStruct;
\end{verbatim}
To start the computation, create an instance of the structure using
\begin{verbatim}
      Gan_SymMatEigenStruct SymEigen;

      /* create structure for computing eigenvalues and eigenvectors,
         initialising accumulated matrix S (here 5x5) to zero */
      gan_symeigen_form ( &SymEigen, 5 );
\end{verbatim}
This routine creates the structure and its constituent matrices, and
initialises the outer product sum matrix, in this case a $5\times 5$ matrix,
to zero. You can then add a term to the sum in Equation~\ref{acc-eigen-sum}
using the routine
\begin{verbatim}
      /* increment matrix S by 
      gan_symeigen_increment ( &SymEigen, 2.0, /* weighting factor s */
                               3.4, 1.0, 9.7, 3.4, 2.1 ); /* elements of vector x */
\end{verbatim}
Here the first value $2.0$ is the weighting factor $\sigma$ for this vector,
and the elements of the vector $\xvec$ are passed into a variable argument
list (and hence have to be explicitly {\tt double} type).
You should call {\tt gan\_symeigen\_increment()} once for each of the
$n$ vectors. Then solve for the eigenvalues \& eigenvectors using the routine
\begin{verbatim}
      /* solve for eigenvalues and eigenvectors */
      gan_symeigen_solve ( &SymEigen );
\end{verbatim}
after which the eigenvalues can be read back from {\tt SymEigen.W} and the
eigenvectors from {\tt SymEigen.Z}.
If you want to reuse the structure on a new eigendecomposition computation,
call the routine
\begin{verbatim}
      /* reset accumulated symmetric to zero, optionally changing size of matrix */
      gan_symeigen_reset ( &SymEigen, 5 );
\end{verbatim}
where the last argument allows you to change the size of the matrix if
desired. Finally to free the structure use
\begin{verbatim}
      /* free eigensystem structure and constituent matrices */
      gan_symeigen_free ( &SymEigen );
\end{verbatim}

{\bf Error detection:} {\tt sym\_symeigen\_form()} returns a pointer to
the eigensystem structure ({\tt \&SymEigen}), and so returns {\tt NULL} on
error. {\tt sym\_symeigen\_increment()}, {\tt sym\_symeigen\_solve()} and
{\tt sym\_symeigen\_reset()} return boolean values, so {\tt GAN\_FALSE}
indicates an error. In all cases the Gandalf error handler is invoked.

\subsection{Single precision general size matrices \& vectors}
Note that for the routines to fill a matrix/vector with values, described
in Sections~\ref{fill-vec-gen-sec} and~\ref{fill-mat-gen-sec}, the values
provided in the single precision case must actually still be double precision,
because C has the restriction that floating point variable argument list
values must be double precision. So for instance, this won't work:
\begin{verbatim}
      Gan_Vector_f vx;

      gan_vecf_form ( &vx, 6 );
      gan_vecf_fill_va ( &vx, 6, 1.0F, 2.0F, 3.0F, 4.0F, 5.0F, 6.0F ); /* WRONG */
\end{verbatim}
Instead use
\begin{verbatim}
      gan_vecf_fill_va ( &vx, 6, 1.0, 2.0, 3.0, 4.0, 5.0, 6.0 ); /* RIGHT! */
\end{verbatim}
\label{single-gen-sec}
If necessary use a {\tt (double)} cast in front of each value.

