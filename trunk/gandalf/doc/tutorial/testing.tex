\chapter{The Test Framework} \label{testing-chapter}
There is extensive testing code in the {\tt gandalf/TestFramework}
directory. A simple way to test your installation of Gandalf is to
execute the following commands:-
\begin{verbatim}
      cd gandalf/TestFramework
      make
      ./cUnit -all
\end{verbatim}
This compiles the main Gandalf test program {\tt cUnit} and runs through
all the tests. There is at least one test program for each Gandalf package.

You can also build test programs individually by {\tt cd}'ing into a
Gandalf package and typing the command {\tt make all}. For instance,
the commands
\begin{verbatim}
      cd gandalf/common
      make all
      ./list_test
\end{verbatim}
will compile all the test programs in the Common package, and run the linked
list test program.
The main test program {\tt TestFramework/cUnit.c} compiles and links with
all the individual test programs in the Gandalf packages.

Input files are all in the {\tt gandalf/TestInput} directory. Output files
are written into the {\tt gandalf/TestOutput} directory. Test programs are
designed to purge any old {\tt TestOutput} files before running the test
program.

\section{Adding new tests}
New code written for Gandalf should come with a test harness that tests
the functionality of the module. Currently there is only one module-specific
test harness in Gandalf, that for linked lists in
{\tt common/linked\_list.[ch]}, the rest of the test programs combining
tests over several modules in a package.
In the future we wish to push Gandalf
in the direction of Extreme Programming, with test harnesses for each module
and for testing interactions between modules.

Let us say we have written a new module {\tt histogram.[ch]} in a
(currently fictitious) {\tt statistics} package. We now wish
to add a test harness {\tt statistics/histogram\_test.[ch]}
(actually we should write the test harness {\em first} according to
Extreme Programming principles). The first thing is to copy another test
program, say {\tt common/list\_test.[ch]}. Remove the test code and change
the names of all the definitions and strings to correspond to the new
test program, leaving the following template files.
Firstly {\tt histogram\_test.h} should like like this:
\begin{verbatim}
      /**
       * File:          $RCSfile: testing.tex,v $
       * Module:        Histogram test program
       * Part of:       Gandalf Library
       *
       * Revision:      $Revision: 1.3 $
       * Last edited:   $Date: 2003/02/24 10:06:15 $
       * Author:        $Author: pm $
       * Copyright:     (c) 2002 YOUR INSTITUTION
       *
       * Notes:         Tests the histogram functions
       */

      /* This library is free software; you can redistribute it and/or
         modify it under the terms of the GNU Lesser General Public
         License as published by the Free Software Foundation; either
         version 2.1 of the License, or (at your option) any later version.

         This library is distributed in the hope that it will be useful,
         but WITHOUT ANY WARRANTY; without even the implied warranty of
         MERCHANTABILITY or FITNESS FOR A PARTICULAR PURPOSE.  See the GNU
         Lesser General Public License for more details.

         You should have received a copy of the GNU Lesser General Public
         License along with this library; if not, write to the Free Software
         Foundation, Inc., 59 Temple Place, Suite 330, Boston, MA  02111-1307  USA
      */

      #include <gandalf/TestFramework/cUnit.h>

      #ifndef CUNIT_HISTOGRAM_TEST_H
      #define CUNIT_HISTOGRAM_TEST_H

      cUnit_test_suite * histogram_test_build_suite(void);

      #endif
\end{verbatim}
Make sure you keep the header and license sections. The {\tt histogram\_test.c}
file should be:
\begin{verbatim}
      /**
       * File:          $RCSfile: testing.tex,v $
       * Module:        Histogram test program
       * Part of:       Gandalf Library
       *
       * Revision:      $Revision: 1.3 $
       * Last edited:   $Date: 2003/02/24 10:06:15 $
       * Author:        $Author: pm $
       * Copyright:     (c) 2002 YOUR INSTITUTION
       *
       * Notes:         Tests the histogram functions
       */

      /* This library is free software; you can redistribute it and/or
         modify it under the terms of the GNU Lesser General Public
         License as published by the Free Software Foundation; either
         version 2.1 of the License, or (at your option) any later version.

         This library is distributed in the hope that it will be useful,
         but WITHOUT ANY WARRANTY; without even the implied warranty of
         MERCHANTABILITY or FITNESS FOR A PARTICULAR PURPOSE.  See the GNU
         Lesser General Public License for more details.

         You should have received a copy of the GNU Lesser General Public
         License along with this library; if not, write to the Free Software
         Foundation, Inc., 59 Temple Place, Suite 330, Boston, MA  02111-1307  USA
      */

      #include <stdlib.h>
      #include <stdio.h>

      #include <gandalf/TestFramework/cUnit.h>
      #include <gandalf/statistics/histogram_test.h>
      #include <gandalf/statistics/histogram.h>

      static Gan_Bool setup_test(void)
      {
         printf("\nSetup for histogram_test completed.\n\n");
         return GAN_TRUE;
      }

      static Gan_Bool teardown_test(void)
      {
         printf("\nTeardown for histogram_test completed.\n\n");
         return GAN_TRUE;
      }

      /* Tests all the histogram functions */
      static Gan_Bool run_test(void)
      {  
         return GAN_TRUE;
      }

      #ifdef HISTOGRAM_TEST_MAIN

      int main ( int argc, char *argv[] )
      {
         /* set default Gandalf error handler without trace handling */ 
         gan_err_set_reporter ( gan_err_default_reporter );
         gan_err_set_trace ( GAN_ERR_TRACE_OFF );

         setup_test();
         if ( run_test() )
            printf ( "Tests ran successfully!\n" );
         else
            printf ( "At least one test failed\n" );

         teardown_test();
         gan_heap_report(NULL);
         return 0;
      }

      #else

      /* This function registers the unit tests to a cUnit_test_suite. */
      cUnit_test_suite *histogram_test_build_suite(void)
      {
         cUnit_test_suite *sp;

         /* Get a new test session */
         sp = cUnit_new_test_suite("histogram_test suite");
      
         cUnit_add_test(sp, "histogram_test", run_test);

         /* register a setup and teardown on the test suite 'histogram_test' */
         if (cUnit_register_setup(sp, setup_test) != GAN_TRUE)
            printf("Error while setting up test suite histogram_test");

         if (cUnit_register_teardown(sp, teardown_test) != GAN_TRUE)
            printf("Error while tearing down test suite histogram_test");

         return( sp );
      }

      #endif /* #ifdef HISTOGRAM_TEST_MAIN */
\end{verbatim}
There are now three functions, {\tt setup\_test()}, {\tt teardown\_test()}
and {\tt run\_test()} for you to fill with your test code.
{\tt setup\_test()} should create any data structures to be used multiple
times by the tests. Then {\tt run\_test()} performs the tests, and
{\tt teardown\_test()} frees the memory allocated by {\tt setup\_test()}.
You can leave {\tt setup\_test()} and {\tt teardown\_test()} blank if you
like, and allocate \& free the memory in {\tt run\_test()}.
It is up to you.

The next stage is to add a rule in the package {\tt Makefile.in} program to
make your test program. Add {\tt histogram-test} as a target to the
{\tt all:} line in {\tt statistics/Makefile.in}. Then add the following lines
to {\tt statistics/Makefile.in}:
\begin{verbatim}
      histogram-test:
              $(LIBTOOL) $(CC) -I$(TOPLEVEL)/.. $(CFLAGS) -DHISTOGRAM_TEST_MAIN histogram_test.c $(PATH_BUILDER_C) -o histogram_test $(LDFLAGS) $(LIB) $(LIBS)
\end{verbatim}
Remember that there must be a tab character at the start of the
{\tt \$(LIBTOOL)} line. Note the predefined symbol {\tt HISTOGRAM\_TEST\_MAIN}.
This is to make sure that the section of {\tt histogram\_test.c} with the
{\tt main()} function is compiled in. The other section of the code is
for when the test functions are linked against the Gandalf test harness.
For now, rerun {\tt ./configure} from the {\tt gandalf} directory to
recreate {\tt statistics/Makefile} with the new rules, and make the
test program with the commands:
\begin{verbatim}
      cd statistics
      make histogram-test
      ./histogram_test
\end{verbatim}
(or {\tt make all}). The tests should be designed so that if the data is
successfully allocated and all the tests pass, {\tt setup\_test()},
{\tt run\_test()} and {\tt teardown\_test()} should all return {\tt GAN\_TRUE}.
There is a special macro {\tt cu\_assert()}, which operates like {\tt assert()}
in the sense that it tests an expression and fails if zero is returned.
In the {\tt cu\_assert()} either {\tt GAN\_TRUE} (one, success) is returned
if the expression is non-zero, and {\tt GAN\_FALSE} (zero, failure) is returned
if the expression is zero. In the latter case an error message is also
printed, providing the line at which failure occurs. This provides a
convenient shorthand for testing the results of the tests.

The next stage is to incorporate the test into the main Gandalf test
harness. To do this, first edit {\tt gandalf/TestFramework/Makefile.in}
and add the following:
\begin{enumerate}
  \item Add {\tt histogram\_test.o} to the {\tt OBJS} list.
  \item Add the line
\begin{verbatim}
      HISTOGRAM_TEST_C  = $(TOPLEVEL)/statistics/hisogram_test.c
\end{verbatim}
	to the list below the {\tt OBJS} list.
  \item Add the rule
\begin{verbatim}
      histogram_test.o: $(HISTOGRAM_TEST_C)
              $(LIBTOOL) $(CC) -I$(TOPLEVEL)/.. $(CFLAGS) -c $(HISTOGRAM_TEST_C)
\end{verbatim}
(remember the tab character again before {\tt \$(LIBTOOL)}).
\end{enumerate}
Now you will need to edit {\tt TestFramework/cUnit.c}. Add the header file
declaration
\begin{verbatim}
      #include <gandalf/statistics/histogram_test.h>
\end{verbatim}
among the other {\tt \#include} declarations. Find the line
which has {\tt \#define maxAutoTests} in it and add one to the number
you see there. You will also need to add a line
\begin{verbatim}
         auto_tests[iIndex++] = "histogram_test";
\end{verbatim}
in the list of similar lines below, and finally the lines
\begin{verbatim}
         pSuite = cUnit_add_test_suite(auto_tests[iIndex++],
                                       histogram_test_build_suite);
         gan_list_insert_last(glstAutoSuiteList, pSuite);
\end{verbatim}
at the corresponding place in the next set of similar lines.
You will need to run {\tt configure} again to recreate the
{\tt TestFramework/Makefile} file, and then typing
\begin{verbatim}
      cd TestFramework
      make
      ./cUnit -menu
\end{verbatim}
should give you the extended menu of test programs with your new test
as one of the options.
